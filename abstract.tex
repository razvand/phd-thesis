\documentclass[11pt,a4paper]{article}

% character encoding
\usepackage{ucs}
\usepackage[utf8x]{inputenc}
\usepackage[english,romanian]{babel}

% adjust the page margins
\usepackage[scale=0.87]{geometry}

% no indent
\usepackage{parskip}

\title{Abtract}
\date{}

\begin{document}

\thispagestyle{empty}
\pagestyle{empty}

\section*{Abstract}

Since their inception in the late '90s, Peer-to-Peer protocols have been the
focus of the scientific community. This work focuses on Peer-to-Peer protocol
analysis and measurements with the objective of providing improvements to
existing protocols and designing new ones. The \textit{BitTorrent} protocol,
one of the most heavily used protocol in the Internet, is at the core of the
research activities.

With the goal of providing improvements to Peer-to-Peer protocols, steps
undertaken have been setting up a realistic Peer-to-Peer system deployment
environment, collecting and measuring protocol parameters, providing analysis
and interpretation for measured data and enhancing and extending existing
protocols. Considering the BitTorrent protocol as an example of protocol
``done right'' and taking into account the large array of implementations, the
focus is to improve speed and performance of Peer-to-Peer protocols.

Virtualization has been used to providing realism and scalability of the
experimental setup. Most experiments and scenarios made use of virtualization
to deploy realistic Peer-to-Peer networks. Results from experiments have been
collected and interpreted at protocol level parameters. Protocol level
parameters are monitored, stored and analysed with the goal of providing
insights for the enhancement phase.

Considering the recent research and commercial focus, Peer-to-Peer streaming
support has also been the subject of experimentation, measurements and
analysis. This work uses real world experiments and the \textit{NextShare}
technology to provide a comparison between Peer-to-Peer streaming
characteristics and Peer-to-Peer classical distribution.

As a protocol level improvement, the \textit{TSUP} protocol has been designed
and implemented. It provides swarm unification through a tracker overlay
network allowing for larger and healthier swarms. Inspired by the design of
the \textit{swift} protocol, a draft design and implementation of a multiparty
protocol has been undertaken. The goal is to create a complete implementation
of a Transport layer protocol in the Linux kernel networking stack.

\hrulefill

Încă de la apariția lor la începutul anilor 90, protocoalele Peer-to-Peer au
captat atenția comunității științifice. Această lucrare prezintă analiza și
măsurătorile protocoalelor Peer-to-Peer cu obiectivul furnizării de
îmbunătățiri pentru protocoale existente și proiectarea unora noi. Protocolul
\textit{BitTorrent}, unul dintre cele mai folosite protocoale din Internet,
este în centrul activităților de cercetare.

Având țelul de a furniza îmbunătățiri protocoalelor Peer-to-Peer, pașii urmați
au fost: crearea unui mediu realist de lansare realist pentru sisteme
Peer-to-Peer, colectarea și măsurarea parametrilor de protocol, analizarea și
interpretarea datelor existente și extinderea protocoalelor existente. Cu
BitTorrent pe post de exemplu de protocol ,,bine făcut'' și având în vedere
numărul mare de implementări, obiectivul este îmbunătățirea vitezei și
performanței protocoalelor Peer-to-Peer.

Pentru a furniza un mediu de experimentare scalabil și realist, au fost
utilizate tehnologii de virtualizare. Cea mai mare parte a experimentelor și
scenariilor au folosit virtualizarea pentru a lansa rețele Peer-to-Peer
realiste. Rezultatele experimentelor au fost colectate și interpretate prin
parametri de nivel de protocol. Parametrii de nivel de protocol sunt
monitorizați, stocați și analizați cu obiectivul de a furniza indicii în faza
de îmbunătățire.

Ținând cont de atenția recentă acordată facilităților de streaming în
Peer-to-Peer, acestea au devenit subiectul experimentelor, măsurătorilor și
analizei. Lucrarea de față s-a folosit de experimente cu utilizatori reali și
de tehnologia \textit{NextShare} pentru a furniza comparație între
caracteristicile streaming-ului Peer-to-Peer și a formei de distribuție
clasică.

Protocolul \textit{TSUP} a fost proiectate și implementat ca îmbunătățire la
nivel de protocol. Acesta furnizează unificarea swarm-urilor printr-o rețea de
trackere de tip overlay care permite swarm-uri mai mari și mai fiabile.
Folosind modelul protocolului \textit{swift}, a fost proiectat și implementat,
în fază preliminară, un protocol multiparty. Scopul este de creare a unei
implementări complete a unui protocol la nivel Transport în stiva de rețea a
nucleului Linux.

\end{document}
