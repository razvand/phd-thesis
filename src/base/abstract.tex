% This file contains the abstract of the thesis

In the ever evolving Internet, Peer-to-Peer systems have grown from a
fan-based file sharing solution to account for the highest percentage of
Internet bandwidth. The development of hardware systems and network links
ensured powerful resources for edge nodes in the Internet. Peer-to-Peer
systems took advantage of these newly developed resources and provided the
means to provide new services to users. While traditionally being used for
file sharing, Peer-to-Peer solutions have been developed for streaming (both
live and video-on-demand), social networking, collaborative work, anonymity.

Peer-to-Peer protocols in general, and the BitTorrent protocol in particular
form the subject of this work. We've undertaken a thorough analysis of current
protocols and their particularities and provided enhancements. We consider the
current design to be a good one, gaining experience for about 10 years of
continuous research and development. As we are focused on research, our aim is
to take protocols that are doing things right and work to make that better.
This generally means making them faster.

Protocol design contributions are present in the design and implementation of
the Tracker Swarm Unification Protocol (\textit{TSUP}) and contributions to a
multiparty protocol. The former provides an enhancement to an existing
protocol -- BitTorrent, while the other provides updates to a newly designed
Peer-to-Peer protocol, challenging current status of data distribution.

Peer-to-Peer streaming has been tackled in the local LivingLab. The LivingLab
has witnessed multiple experiments using the P2P-Next framework. These
experiments have provided information and measurement data for analysis and
comparison between classical Peer-to-Peer distribution and streaming.

As formal results, metrics for virtualization solutions and Peer-to-Peer
measurements are proposed. The aim is to provide the context for analyzing and
improving Peer-to-Peer protocols and node distribution.

In order to run experiments and to provide insight to the inner protocol
workings and network, measurement frameworks have been designed and
implemented. Our approach means using logging information provided by
Peer-to-Peer implementations and gather protocol performance data. This has
been possible with the help of a designing, building and using a lightweight
virtualized infrastructure. The infrastructure is used to configure and deploy
a complete Peer-to-Peer network and gather information afterwards.

Considering the sheer size of research and development in Peer-to-Peer
systems, this work couldn't possibly present an exhaustive approach to
improving Peer-to-Peer protocols. We have constructed a Peer-to-Peer
infrastructure stack for experimentation and provided protocol improvements
and updates. We hope that we've set up new bricks in the right direction and
are anxious to see new challenges being taken.
