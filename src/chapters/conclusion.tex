% vim: set tw=78 tabstop=4 shiftwidth=4 aw ai:

\chapter{Conclusion}
\label{chapter:conclusion}

In the ever evolving context of Internet technologies in general, and
Peer-to-Peer technologies in particular, this work presented approaches on
measuring protocol parameters and providing improvements with particular focus
on performance. Considering the past decade evolution of Peer-to-Peer
technology, we have been using lessons learned in the past years to augment
existing protocols or design and implement new ones, while keeping the grip on
running realistic trials and collecting and interpreting protocol parameters.

Our measurements analysis has been centered at protocol level, with focus on
peer protocols. We didn't focus on providing an overall swarm view, but rather
a client-centric analysis. Information provided by clients had bee subject to
analysis -- protocol parameters such as download rate, upload rate, peer
connections, protocol events are at the center of the analysis. An
``individualistic'' client-centric approach was preferred against an overall
swarm-centric one.

Recognizing the immense impact of the BitTorrent protocol, most of the
measurements and improvements presented have made use of BitTorrent
architecture and specifics. BitTorrent protocol message analysis, protocol
parameters, streaming updates, tracker overlays, automating client actions
have formed the primary set of activities used in this work.

The approach used in this work, as highlighted in the chronology of the
chapters. We followed a deploy, run, analyze, evaluate, improve approach
concerning Peer-to-Peer clients and protocol messages. An infrastructure is
deployed forming the basis for nodes running within a swarm. Automation had
been employed to easily configure and run various implementations on the
infrastructure. Information is gathered in interpreted as protocol parameters,
that are subsequently subject to evaluation. With the evaluation and advices
in mind, improvements are implemented and proposed.

The infrastructure that has been used as basis for trials and experimentation
uses virtualization for efficiency reasons and easy configuration. Various
swarms and topologies have been deployed on top of the infrastructure and run.
The client-centric approach used in this work required data provided from
peers; this data has been processed into protocol parameters that at their
turn had been stored and then used for analysis, evaluation and comparison.
Various approaches on data collection and processing had been considered and
employed.

Considering the recent deep focus on streaming updates into P2P systems,
measurements, analysis and evaluation of streaming technologies had also been
presented. The local LivingLab, as part of the P2P-Next project, has been the
testing ground for live trials, involving users and streaming technologies. As
the NextShare technology provides good support for both video-on-demand and
live streaming it has been chosen as the etalon for our evaluation.

Improvements to Peer-to-Peer systems have been providing in the light of the
design and implementation of two new protocols. The first one is a tracker
overlay protocol on top of an existing BitTorrent swarm. The other is
designing a multiparty protocol at Transport level in the Linux kernel
networking stack -- relying heavily on the previous swift implementation.
These updates provide extended features or improved performance to
Peer-to-Peer systems.

The greater part of this work had been part of the P2P-Next project. The
author is thankful and appreciative towards the P2P-Next team members, that
have been working tirelessly and enthusiastically towards providing the next
generation Peer-to-Peer content delivery platform.

\section{Summary}
\label{sec:conclusion:summary}

This thesis presented the work concerning measurements and improvements at
protocol level for Peer-to-Peer systems with particular focus on the
BitTorrent protocol. Various approaches, techniques and updates are shown
throughout the chapters while generally trying to keep concepts and ideas in
order, both chronologically and logically. First chapters are concerned with
measurements and parameter evaluation, while the last chapters are focused on
actual improvements.

Thesis objective, scope and context are presented in
Chapter~\ref{chapter:intro}. The chapter is an overview of the work described
in the rest of the thesis and highlights the most important keywords and
primary focus.

Chapter~\ref{chapter:p2p-systems} is a state of the art regarding the
evolution and current challenges of Peer-to-Peer systems. The Peer-to-Peer
paradigm is presented, along with possible P2P topologies and implementations.
Focus is given to the BitTorrent protocol and streaming capabilities of P2P
systems, paving the way for contributions highlighted in later chapters.

At the basis of every work dealing with real world systems, there lies an
environment or set of environments used as testing ground. The virtualized
network infrastructure deployed for various trials is described in
Chapter~\ref{chapter:virt-infra}. Advantages and benefits of virtualization
are presented, with focus on OpenVZ -- the solution chosen as the basis for
the infrastructure implementation. Other tools and approaches are also
presented with a general focus on efficiency, automation and ease of
deployment.

Chapter~\ref{chapter:proto-measure} introduces the approaches for collecting
Peer-to-Peer measured information and the parameters at use and approaches to
parameter analysis. The two approaches used (hooking into client source code)
and collecting logs are presented within a generic logging library and client
instrumentation or context for providing log files. Parameters are stored in
an easy to access database which is easily accessed by a rendering engine.

One of the improvements highlighted in this work is the development of a
tracker overlay protocol, dubbed \textit{Tracker Swarm Unification Protocol}
(TSUP), detailed in Chapter~\ref{chapter:unified-tracker}. The new protocol is
a used between tracker such that, if coordinating swarms centered around the
same .torrent files, these swarms would be unified. Unification means that
peers in the initially two different swarms would be able to communicate with
each other and transfer data. All updates and extra communication are
tracker-centric; no modifications are required to client code.

Chapter~\ref{chapter:multiparty} presents a novel design for a multiparty
protocol at kernel level heavily based on the swift protocol. The aim is to
provide a true running multiparty protocol at Transport level; the Linux
kernel networking stack is the perfect place for such an operation. An
intermediate approach, relieving the burden of kernel programming, but still
providing the same programming interface is a raw socket-based implementation.

Trials and evaluations centered around the local LivingLab and content
distribution using Peer-to-Peer technology is presented in
Chapter~\ref{chapter:multimedia-dist}. It presents a contribution to the
libtorrent-rasterbar engine for providing streaming, and then focuses on the
inner workings and features of the NextShare technology used within the
P2P-Next project. We present the work and results involved in deploying live
trials in the LivingLab and evaluation results.

The final chapter (\textbf{Conclusion}) presents ending remarks and
highlights contributions of this work.

\section{Contributions}
\label{sec:conclusion:contributions}

The work described in this thesis provided significant contributions in
Internet technology and Peer-to-Peer technology. The research and engineering
activities described provide updates to measurement and evaluation methods and
existing protocols used in Peer-to-Peer systems. With focus on client-centric
behavior and performance evaluation, the contributions highlighted below
amount to valuable progress in analysing and improving Peer-to-Peer protocols
and applications.

Contributions of this thesis range from infrastructure building and automation
to protocol designs and implementations and streaming experiments. Main
contributions are:
\begin{itemize}
  \item development of a fully automated and scalable virtualized
  infrastructure for Peer-to-Peer experiments;
  \item proposal of metrics for evaluation of virtualiation solutions;
  \item design, development and incorporation of a BitTorrent logging library;
  \item design and development of an automated framework for extracting,
  processing and analysing protocol data for BitTorrent implementations;
  \item proposal of a formal context for evaluation of BitTorrent clients
  based on measured protocol parameters;
  \item design and implementation of a novel protocol for swarm unification in
  BitTorrent environment;
  \item design and implementation of a multiparty protocol in the Linux
  kernel;
  \item evaluation of Peer-to-Peer streaming technologies through user-centric
  experiments.
\end{itemize}

The creation of a \textbf{fully automated and scalable infrastructure} stands
at the basis of extensive trials and experiments involving Peer-to-Peer
systems. The infrastructure is able to integrated a variety of Peer-to-Peer
clients and offer a in depth view of their behavior and performance
parameters.

The use of \textbf{virtualization technology in the context of Peer-to-Peer
systems} offers an original approach towards providing a rapid deployment of
experimental setups and trials. Virtualization offers the benefit of deploying
a medium-sized swarm (consisting of several hundred nodes) on top of a much
lower number (as low as 10) of modest commodity hardware systems. At the same
time it provides an easy to use interface ensuring extensibility,
configurability and automation.

We have proven the \textbf{scalable use of OpenVZ solutions} for providing a
realistic infrastructure. The use of OpenVZ, a lightweight virtualization
solution, meant we ccould easily deploy a high number of virtualized systems
and thus, Peer-to-Peer nodes. Operating system-level virtualization solutions
are to be taken seriously in any situation where scalability and realism are
of importance.

The advantages and disadvantages of virtualization solutions have been
explored and formalized as a set of \textbf{virtualization metrics}. These
metrics provide mainly a comparison view of different virtualization
solutions. While a complete numerical formalism is improbable to be created,
we do provide an insight on how a virtualization solution compares against the
other and what are the major factors that influence their performance.

We undertook a set of approaches to \textbf{simulating connection dropouts in
BitTorrent environments}. We considered several approaches to simulating
connection dropouts such as stopping/suspending the process, terminating the
process, filtering the connection through the use of a firewall. We drew
several conclusions on the matter, presenting the advantages and disadvantages
of each of the analysed approaches.

In the process of providing realistic and relevant trials with existing
Peer-to-Peer clients, \textbf{updates and patches have been applied to
existing BitTorrent} clients. Most of these have been applied to the clients
with the heaviest use in our trials, namely Tribler/NextShare and
hrktorrent/libtorrent-rasterbar. Updates have been sent to developers and
integrated upstream. Instrumentation has also made use of hidden or
non-obvious interfaces to existing clients, mostly in the process of gathering
logging information.

We isolated and defined \textbf{protocol messages types and protocol
parameters} specific to Peer-to-Peer systems. We defined two types of
messages: status messages and verbose messages; these messages consist of
valuable information that is translated by a logging engine in protocol
parameters such as download rate, upload rate, number of peer connections,
protocol events and others.

A \textbf{generic logging library} has been implemented to provide an API that
may be hooked in existing clients. The library provides API both for status
and verbose messages. It has been integrated and tested against Transmission
and rtorrent clients. It offers configurability of the API to use and the type
of logging information provided (either plain text or XML format).

\textbf{Client-centric logging updates and configuration} have formed the
basis of the most heavily used approach towards collecting protocol
information from running clients. The logging facility in place, used in
conjunction with the virtualized infrastructure, instruments and/or configures
BitTorrent clients to provide extensive logging information. The information
is collected and subject to analysis.

\textbf{A protocol parameter parsing and analysis engine} has been developed
to provide an in depth look at information provided by client log files. It
consists of several components such as parsers, storage engines and rendering
engines. Parsers may be \textbf{post-processing parsers or real-time parsers},
with the ability of the latter to provide monitoring of clients and swarms.
Storage engines, typically databases, are ideal for providing and easy to
access, low overhead interface to protocol parameters. The rendering engine
provides a GUI and graphics-centered interface of protocol information.

Within protocol measurement activities, we've defined \textbf{a formal context
for performance evaluation} of Peer-to-Peer protocols. It takes into account
parameters extracted from logging messages and builds up a model for analysis
and interpretation. The formalism is centered around performance, mainly
transfer speed, and thus transfer time.

During our investigation of Peer-to-Peer streaming technologies, we provided
\textbf{streaming support to the popular libtorrent implementation}. Several
streaming algorithms have been put into place; the piece picker algorithm
used by libtorrent had to be significantly updated in order to provide a
realistic streaming experience.

Within the context of the P2P-Next project, the deployment and maintenance of
the local LivingLab was essential towards \textbf{providing live user centric
experiments regarding Peer-to-Peer technology}. The LivingLab provides access
to a plethora of video asset files delivered through the use of Peer-to-Peer
technology, in particular NextShare, in the form of browser plugins, easy to
use and deploy by users. Various experiments have taken place involving users
and allowing them to feedback the current technology and provide suggestions.

A significant part of the LivingLab activities was and is dedicated to
\textbf{analysis and monitoring of BitTorrent streaming}. We created the test
ground for streaming trials involving users and Peer-to-Peer streaming
technology. This has provided preliminary results in analysing BitTorrent
based streaming and providing a comparison between classical BitTorrent
distribution and streaming.

A novel overlay network protocol on top of BitTorrent, aiming at integrating
peers in different swarms, has been presented. Dubbed TSUP (Tracker Swarm
Unification Protocol), the protocol is used for \textbf{creating and
maintaining a tracker network enabling peers in swarms to converge in a single
swarm}. Each initial swarm is controlled by a different tracker; trackers use
the overlay protocol to communicate with each other and, thus, take part in a
greater swarm. Implementation has been tested against the XBT Tracker
implementation.

The TSUP protocol has been subject to \textbf{experimentation within the XBT
tracker}. These experiments have used a variety of topologies testing and
evaluating the protocol: from simple one to one tracker networks to complex
networks involving multiple trackers in dedicated tracker networks. It was
noticed, from the experiments, the reduced overhead incurred and the overall
behavior of the new swarm.

We have proposed and designed an optimization of the currently \textit{swift}
protocol. The \textbf{integration the kernel space as a multiparty transport
protocol} that is solely responsible for getting the bits moving improves the
over all protocol performance. It ensures maximum efficiency of data transfer
by decreasing switches between user and kernel space and eliminating some
performance penalties due to context switches. Currently in draft phase,
there is an ongoing standardisation effort to provide swift and, thus, the
current multiparty kernel implementation as an IETF approved standard.

In order to provide a rapid development environment, we've created a
\textbf{user-space implementation and test suite for the multiparty protocol}.
Considering the harsh kernel development environment, this implementation
allows rapid design, implementation and testing. After features are tested
against the user-space implementation they would be ``ported'' to the
kernel-space implementation.

Most of the code required for this work has made use of Python, the C
programming language, and shell scripting. Code metrics for languages and
tools that have been used throughout the project result in the below list:

\begin{itemize}
  \item \textbf{Python}: more than 20,000 lines of code
  \item \textbf{C/C++}: around 5,000 lines of code
  \item \textbf{shell scripting} and related (sed, awk): around 5,000 lines
  of code
  \item \textbf{R scripts} for processing: more than 500 lines of code
\end{itemize}

\section{Future Work}
\label{sec:conclusion:future}

As is the case with improvements, updates and extensions, no work could ever
be complete. Continuous protocol evolution, the ever expanding Internet, new
user request request further updating. At the same time, current approaches
must be pushed towards a larger crowd, high quality formal evaluations and
increased reliability.

The multiparty protocol design and implementation in the Linux kernel has to
be completed, tested, evaluated. Current goal is to push for a mainline
integration, albeit in connection with the standardisation effort taking
place. The implementation presented, together with swift are part of a draft
that is to be submitted to IETF for standardisation, as a PPSP (Peer-to-Peer
Streaming Protocol). An efficient implementation, together with a standard
back up, would simplify the work of providing the current implementation as an
important contribution to the Linux kernel.

The current infrastructure (both from a virtualized point of view and from
measurement capabilities), needs enhancing. This means the addition of further
clients, such as powerful and popular ones: uTorrent and swift, and the
addition of other virtualization solutions (such as KVM, LXC, Xen). Scripts
should be packed to provide an easily installable base on top of other
hardware configuration.

A draft model for virtualization adequacy had been presented at the end of
Chapter~\ref{chapter:virt-infra} proposing metrics such as isolation,
efficiency and performance. This model hasn't been applied to any
virtualization solution and forms one of the desired directions for further
work. Providing new virtualization solutions to the infrastructure would
provide the experimental testing ground for properly evaluating and verifying
the model.

Formal evaluation of Peer-to-Peer parameters, as highlighted in the final part
of Chapter~\ref{chapter:proto-measure} should be improved. Currently the
defined model provides little insight on what are the bottleneck and
problematic points where the experimenter should insist and focus his/her
actions. Enhancing the evaluation implies providing the means through which
the experimenter may detect anomalies or performance loss and also understand
the factors influencing them, factor which should be tuned.

On par with the formal evaluation of Peer-to-Peer parameters is the
Peer-to-Peer streaming versus classical distribution activity mentioned in
Chapter~\ref{chapter:multimedia-dist}. The focus of the LivingLab is
providing sound comparison between the two updates and, based on that, provide
advice regarding improvement of streaming extensions. Preliminary trials have
only taken into account status message parameters from seeders in swarms;
complete trials should take into account all parameters available and do
analysis on top of that.

The exciting and ever-evolving context of Peer-to-Peer systems offers a
plethora of possibilities for further work on measuring and improving
protocols. While primarily aiming to enhance current efforts, new research
directions and challenges may be paved.

\section{Publications and Talks}
\label{sec:conclusion:publications}

\subsection{Books}

\begin{itemize}
  \item Răzvan Rughiniș, Răzvan Deaconescu, George Milescu and Mircea Bardac.
  Introducere în sisteme de operare. Editura Printech, 2009, Bucharest. ISBN:
  978-606-521-386-9
  \item Răzvan Rughiniș, Răzvan Deaconescu, Andrei Ciorba and Bogdan Doinea.
  Rețele locale. Editura Printech, 2008, Bucharest. ISBN: 978-606-521-092-9
\end{itemize}

\subsection{Talks}

\begin{itemize}
  \item Performance of P2P Implementations. \textit{P2P'08 Workshop}. Aachen,
  September 2008
  \item Peer-to-Peer Systems. Evolution and Challenges. \textit{Ixia HiTech
  Presentations}. Bucharest, April 2011
\end{itemize}

\subsection{Papers}

\begin{itemize}
  \item Mircea Bardac, George Milescu, and Răzvan Deaconescu. Monitoring a
  BitTorrent Tracker for Peer-to-Peer System Analysis. In \textit{Intelligent
  Distributed Computing}, pages 203--208, 2009 (\textbf{ISI indexed})
  \item Călin-Andrei Burloiu, Răzvan Deaconescu, and Nicolae Țăpuș. Design and
  Implementation of a BitTorrent Tracker Overlay for Swarm Unification. In
  \textit{International Conference on Network Services}, 2011 (\textbf{ISI
  indexed})
  \item Răzvan Deaconescu, George Milescu, Bogdan Aurelian, Răzvan Rughiniș,
  and Nicolae Țăpuș. A Virtualized Infrastructure for Automated BitTorrent
  Performance Testing and Evaluation. \textit{International Journal on
  Advances in Systems and Measurements}, 2(2\&3):236--247, 2009
  \item Răzvan Deaconescu, George Milescu, and Nicolae Țăpuș. Simulating
  Connection Dropouts in BitTorrent Environments. In \textit{EUROCON --
  International Conference on Computer as a Tool}, 2011, IEEE, pages 1-4, 2011
  (\textbf{ISI indexed})
  \item Răzvan Deaconescu, Răzvan Rughiniș, and Nicolae Țăpuș. A BitTorrent
  Performance Evaluation Framework. \textit{Proceedings of Fifth International
  Conference of Networking and Services}, 2009, \textbf{Best Paper Award}
  (\textbf{ISI Indexed})
  \item Răzvan Deaconescu, Răzvan Rughiniș, and Nicolae Țăpuș. A Virtualized
  Testing Environment for BitTorrent Applications. \textit{Proceedings of
  CSCS'17}, 2009
  \item Răzvan Deaconescu, Marius Sandu-Popa, Adriana Drăghici, and Nicolae
  Țăpuș. Using Enhanced Logging for BitTorrent Swarm Analysis. In
  \textit{Proceedings of the 9th RoEduNet IEEE International Conference},
  Sibiu, 2010 (\textbf{ISI Indexed})
  \item Răzvan Deaconescu, Marius Sandu-Popa, Adriana Drăghici, and Nicolae
  Țăpuș. BitTorrent Swarm Analysis through Automation and Enhanced Logging.
  \textit{International Journal of Computer Networks \& Communications},
  3(1):53--65, 2011
  \item Andreea Leța, Răzvan Deaconescu and Răzvan Rughiniș. Extending Packet
  Altering Capacities in Simulated Large Networks. In \textit{Proceedings of
  the 17th International Conference on Control Systems and Computer Science
  (CSCS17)}, Bucharest, 2009
  \item Marius Sandu-Popa, Adriana Drăghici, Răzvan Deaconescu, and Nicolae
  Țăpuș. A Peer-to-Peer Swarm Creation and Management Framewor. In
  \textit{Proceedings of the 1st Workshop on Software Services: Frameworks and
  Platforms}, Timișoara, Romania, 2010 (\textbf{ISI indexed})
  \item George Milescu, Răzvan Deaconescu, and Nicolae Țăpuș. Versatile
  Configuration and Deployment of Realistic Peer-to-Peer Scenarios. In
  \textit{International Conference on Network Services}, 2011 (\textbf{ISI
  indexed})
  \item Răzvan Rughiniș and Deaconescu, Răzvan. Analysis of a QoS-based
  Traffic Engineering Solution in GMPLS Grid Networks. \textit{17th
  International Conference on Control Systems and Computer Science},
  Bucharest, 2009
  \item Răzvan Rughiniș and Răzvan Deaconescu. Optimization Strategies in MPLS
  Traffic Engineering. \textit{UPB Scientific Bulletin, Series C}, 1/ 2009,
  ISSN 1454-234x, pp. 91-102
  \item Răzvan Rughiniș and Răzvan Deaconescu. Methods of Adjusting MPLS
  Network Policies. \textit{UPB Scientific Bulletin, Series C}, 3/2009, ISSN
  1454-234x, pp. 121-132
  \item Mircea Bardac, Răzvan Deaconescu and Adina Magda Florea. Scaling
  Peer-to-Peer testing using Linux Containers. \textit{9th RoEduNet IEEE
  International Conference}, 2010, June 24-26, 2010, Sibiu, Romania, pag.
  287-292, ISSN: 2068-1038, ISBN 978-1-4244-7335-9 (\textbf{ISI indexed})
  \item Laura Gheorghe, Răzvan Rughiniș, Răzvan Deaconescu, and Nicolae Țăpuș.
  Authentication and Anti-replay Security Protocol for Wireless Sensor
  Networks. \textit{The Fifth International Conference on Systems and Networks
  Communications}, 2010 (\textbf{ISI indexed})
  \item Laura Gheorghe, Răzvan Rughiniș, Răzvan Deaconescu and Nicolae Țăpuș.
  Reliable Authentication and Anti-replay Security Protocol for Wireless
  Sensor Networks. \textit{The Second International Conferences on Advanced
  Service Computing}, 2010 (\textbf{ISI indexed})
  \item Laura Gheorghe, Răzvan Rughiniș, Răzvan Deaconescui and Nicolae Țăpuș.
  Adaptive Trust Management Protocol Based on Fault Detection for Wireless
  Sensor Networks. \textit{The Second International Conferences on Advanced
  Service Computing}, 2010 (\textbf{ISI indexed}):
\end{itemize}
