% vim: set tw=78 tabstop=4 shiftwidth=4 aw ai:

\chapter{A Novel BitTorrent Tracker Overlay for Swarm Unification}
\label{chapter:unified-tracker}

While Gnutella is a fully decentralized Peer-to-Peer protocol, BitTorrent
rests on a dedicated service, called tracking, to ensure bootstrapping. The
tracker is a metadata storage facility that provides ``who's sharing what''
information to peers. The tracker is contacted by peers to provide information
about other peers sharing the same file. Based on this information, peers
subsequently contact other peers and transfer pieces. The tracker only fills
the role of piece possession mediator among peers.

The prime disadvantage of this architecture is the single point of failure
role of the tracker. If the tracker fails further contacts among peers are
disabled -- peers may still communicate with peers they are already acquainted
to. A simple solution to this problem is the introduction of multiple trackers
in a swarm; this is easily achieved by adding multiple tracker announce URLs
to the .torrent file. In case a tracker fails, the peer may contact another
tracker. Note that tracker are note synchronized among themselves. Still,
new contacts may be provided through the new tracker. As a further update, the
.torrent file may be simplified through the use of magnet
links\footnote{http://news.softpedia.com/news/BitTorrent-Magnet-Links-Explained-132536.shtml}.

The role of the tracker, and the problem of it being a single point of
failure, may be circumvented through the use of PXE (\textit{Peer Exchange})
and DHT (\textit{Distributed Hash Table})~\cite{dht-paper}. These new
technologies are new ways of discovering peers, without having to rely on the
BitTorrent tracker. There needs to be an initial bootstrap phase; control
information is provided from one peer to another peer. Most implementations
nowadays provide PXE or DHT, coupled with the use of a tracker. In many cases,
such as libtorrent-rasterbar, the implementation may simultaneously use both
DHT and the classical tracker data.

The role of extending tracker functionality is to provide redundancy and
generally simplify the architecture. Our goal of building a tracker overlay
for swarm unification aims to enhance swarms, and allow the presence of
multiple peers and seeders in a swarm, with the potential benefit of increased
swarm speed. As the health of a swarm not only relies on the number of peers,
but also on the ratio of seeders versus leechers, approaches such as building
private swarms or providing moderation are important. These approaches provide
the means for extending seeder availability within the swarm and providing
a constant flow of data.

Trackers are server-only entities. They are contacted by peers to update
internal information and provide information to clients. The tracker database
stores information about peers taking part in a swarm and statistical
information about peers, such as download size. This information have been
used heavily by Bardac~et~al.~\cite{tracker-mon} in building a tracker
monitoring service. Our proposal is to enhance existing tracker messages and
allow trackers to interconnect -- that is to communicate with each other, on
top of the existing communication with peers in the swarm.

\section{Context and Objective}
\label{sec:unified-tracker:context}

In this chapter, we address the issue of unifying separate swarms that take
part in a session sharing the same file. We propose a tracker unification
protocol that enables disparate swarms, using different .torrent files, to
converge. We define swarm unification as enabling clients from different
swarms to communicate with each other. The basis for the unification is a
``tracker-centric convergence protocol'' through which trackers form an
overlay network and send peer information to each other.

Communication of peers in a swarm is typically mediated by a BitTorrent
\textbf{tracker} or several trackers which are defined in the .torrent file.
It is periodically contacted by peers to provide information regarding
piece distribution within the swarm. A peer would receive a list of peers from
the tracker and then connect to these peers in a decentralized manner. As it
uses a centralized tracker, the swarm may suffer if the tracker is
inaccessible or crashes. Several solutions have been devised, such as PEX
(Peer EXchange)\footnote{\url{http://wiki.vuze.com/w/Peer\_Exchange}} or DHT
(Distributed Hash Table)~\cite{dht-paper}. The tracker swarm unification
protocol presented in this article enables redundancy by integrating multiple
trackers in a single swarm.

The goal of the tracker swarm unification protocol is the integration of peers
taking part in different swarms that share the same file. These swarms, named
\textbf{common swarms}, use the same content but different trackers.

We have designed and implemented a tracker network overlay that enables
trackers to share information and integrate peers in their respective swarms.
The overlay is based on the Tracker Swarm Unification Protocol (TSUP) that
allows update notification and delivery to trackers from the overlay. The
protocol design is inspired by routing protocols in computer networks.

At this point, as proof of concept, the tracker overlay is defined statically.
All trackers know beforehand the host/IP addresses and port of the neighboring
trackers and contact them to receive required information. The integration of
dynamic tracker discovery is set as future work. Each tracker may act as a
``router'', sending updates to neighboring trackers that may themselves send
them to other trackers.

The protocol proposed in this article, named \textbf{Tracker Swarm Unification
Protocol (TSUP)}, renders possible the unification of swarms which share
the same files, by employing a tracker network overlay. A tracker which
implements this protocol will be referred here as an \textbf{unified tracker}.

\begin{figure}[h]
  \begin{center}
    \includegraphics[width=0.7\textwidth]{src/img/unified-tracker/unified-trackers}
  \end{center}
  \caption{Unified trackers}
  \label{fig:unified-tracker:unified-trackers}
\end{figure}

Torrent files created for the same content have the same \textit{info_hash}.
As such, in swarms that share the same file(s) (common swarms), peers will announce
to the tracker the same \textit{info_hash}. Therefore, TSUP capable trackers
can ``unify'' them by communicating with each other in order to change
information about peers which contribute to shared files with the same
\textit{info_hash}.  In order to accomplish this, unified trackers send
periodic updates to each other, containing information about the peers from
the network.

\section{Swarm Unification Protocol}
\label{sec:unified-tracker:swarm-unification}

As mentioned above, \textit{TSUP} is the acronym for \textit{Tracker Swarm
Unification Protocol}. TSUP is responsible with the communication between
trackers for peer exchange information in common swarms.

\subsection{Protocol Overview}
\label{subsec:unified-tracker:protocol-overview}

For transport layer communication, the protocol uses UDP (User Datagram
Protocol) to reduce resource consumption. A tracker already possesses a lot of
TCP (Transport Control Protocol) connections with other peers. Adding
more TCP connections to each neighboring tracker would increase TSUP overhead
too much for a big tracker overlay. The messages passed from one tracker to
another do not need TCP's flow control and need a lower level of reliability
than TCP as it will be explained below. The simplicity of the UDP protocol
gives the advantage of a smaller communication overhead.

In TSUP's operation the following processes may be identified:
\begin{itemize}
    \item \textbf{Virtual connections establishment process:} A
    three-way-handshake responsible with establishing a UDP ``connection''
    between two linked trackers at the application layer. The process is
    started by a SYN packet (synchronization packet).
    \item \textbf{Unification process:} The trackers exchange unification
    packets (named \textit{SUMMARY} packets) during a three-way-handshake in
    order to find out which are the common swarms.
    \item \textbf{Updating process:} The trackers exchange peers from common
    swarms, encapsulated in \textit{UPDATE} packets.
    \item \textbf{Election process:} The establishment of a \textit{swarm
    leader} which is responsible with receiving all updates from the
    neighboring linked trackers, aggregating them and sending the results
    back.
\end{itemize}

The \textit{unification process} includes an updating process in its
three-way-handshake, such that the two operations, unification and update, are
run in pipeline. Whenever a new torrent file is registered to the tracker, a
new swarm is created. The unification process is triggered and a SUMMARY
packet is immediately sent to each neighboring tracker, informing the others of the new swarm.

The \textit{updating process} is started periodically, such that UPDATE
packets are sent at a configurable interval of time to each tracker in a
common swarm. A typical update interval is 30 seconds.

In order to maintain the virtual connections between trackers, HELLO packets
are sent periodically, acting as a keep-alive mechanism. A typical HELLO
interval is 10 seconds, but its value may be changed from protocol
configuration. If no HELLO packet is received during a configurable interval,
called disconnect interval, the virtual connection is dropped and the virtual
connection establishment process is restarted for that link by sending a SYN
packet.

Some packets, such as UPDATE packets, must be acknowledged. If no answer or
acknowledgement is received, the packet is retransmitted. For example, UPDATE
packets are resent at each HELLO interval until an acknowledgement is received.

It is not a problem if some UPDATE packets are lost and arrive later to
destination. However they need to be acknowledged and they are retransmitted
in order to increase the probability of their arrival. TCP, by offering
reliability, provides a faster delivery of updates in case of a network
failure which is not needed in the case of TSUP. Lower overhead is considered
here more important that fast retransmission. Thus TSUP implements a
timer-driver retransmission, as opposed to data-driven used by TCP.

Periodically sent packets, the keep-alive mechanism, acknowledgements and
retransmissions contribute to the low overhead needed in TSUP. They help
exceed the drawbacks of the UDP transport protocol, and also give a more
efficient communication than a TCP one.

\subsection{Tracker Awareness}
\label{subsec:unified-tracker:tracker-awareness}

Tracker communication is conditioned by awareness. For this purpose, in the
current version of the protocol, each tracker is configured statically with a
list of other communicating trackers. Each element of the list represents a
\textit{link} which is identified by the tracker host name (URL or IP address)
and port. Other parameters for the link may be configured; some of them may be
specific to the implementation. If the virtual connection establishment
process is successful, the link becomes a virtual connection, which is
conserved with keep-alive packets (\texttt{HELLO} packets).

A future version of the protocol will incorporate the design of a tracker
discovery mechanism capable of generating the list of communicating trackers
for each tracker dynamically, with the benefit of scalability and reduction of
the administrative overhead.

\subsection{Tracker Networks}
\label{subsec:unified-tracker:tracker-networks}

To improve TSUP's scalability, trackers may be grouped together in networks
named \textbf{tracker networks}. Connections in all tracker networks are full
mesh. Two networks are connected with the aid of \textbf{border trackers} (see
Figure~\ref{fig:unified-tracker:tracker-networks}).

\begin{figure}[h]
  \begin{center}
    \includegraphics[width=0.7\textwidth]{src/img/unified-tracker/tracker-networks}
  \end{center}
  \caption{Tracker Networks}
  \label{fig:unified-tracker:tracker-networks}
\end{figure}

To configure a topology which contains multiple networks, each link of each
tracker must be set as an \textbf{internal link} or an \textbf{external link}
(see Figure~\ref{fig:unified-tracker:tracker-networks}). Trackers connected
with internal links are part of the same \textit{tracker network}; trackers
connected with external links are part of other networks. However, the ones
from the first category may also be classified as belonging to an
\textbf{internal network} and the ones from the second category as being from
an \textbf{external network}.

A complete graph, using internal links as edges is an \textit{internal
network}, and a complete graph with external links as edges is an
\textit{external network} (see
Figure~\ref{fig:unified-tracker:tracker-networks}). A tracker which has both
internal and external links is a border tracker. Peer information received
from an internal network originates from \textbf{internal peers} while
information received from an external network originates from \textbf{external
peers}. Peers connected to a tracker with TCP, via the BitTorrent protocol,
are called \textbf{own peers}. The links between trackers in a network,
whether internal or external, must be full mesh.

In order to use a scalable and low resource consuming communication within a
network, trackers are organized in groups depending on the unified swarm.
Therefore a tracker may belong to more than one group at the same time, the
number of groups it belongs being equal with the number of swarms present on
that tracker. Each group contains a \textbf{swarm leader} responsible for
sending peer updates to peers in the group. The other group members, instead
of sending updates to other members on a full mesh graph, it sends updates to
the swarm leader on a tree graph, reducing updating overhead. These updates
propagate to other peers in the group.

In accordance to graph theory, the number of updates sent in a swarm without
the swarm leader mechanism (full mesh) is computed by using the formula below:

%\vspace{3mm}

\begin{equation}\label{updates-fullmesh}
UPDATES_{full mesh} = 2\cdot\frac{n(n-1)}{2} = n(n-1)
\end{equation}

%\vspace{3mm}

The number of updates sent within a swarm using the swarm leader mechanism
(tree) are:

%\vspace{3mm}

\begin{equation}\label{updates-swarmleader}
UPDATES_{swarm leader} = 2(n-1)
\end{equation}

%\vspace{3mm}

As may be observed from the formulas above the complexity decreases from
$O(n^{2})$ in a full mesh update scheme to $O(n)$ with the swarm leader
scheme.

Each swarm contains two swarm leaders, one for the internal network (which
sends updates through internal links), called \textit{internal swarm leader}
and one for the external network (which communicates updates through external
links), named \textit{external swarm leader}.

As connections are full mesh in an internal network, the internal peers
(received from other trackers from the internal network) are distributed to
other internal trackers only by the internal swarm leader and in no other
circumstance by another tracker. Through analogy, in an external network,
peers (received from other trackers from the external network) are distributed
to other external trackers only through the external swarm leader. On the
other hand, internal peers are distributed to external trackers and external
peers to the internal trackers. Own tracker peers are distributed both to the
internal and the external network.

Swarm leaders are automatically chosen by trackers during the election process
which is started periodically. There are metrics used in order to choose the
most appropriate leader. The first and most important one prefers as swarm
leader a tracker which possesses the smallest \textit{number of swarm leader
mandates}. The \textit{number of mandates} is the number of swarms where a
tracker is swarm leader. This balances the load of the trackers -- as the
number of mandates of a tracker increases, its load also increases. In the
current version of the protocol the grouping of trackers into networks and the
selection of border trackers is done manually (statically) by the system
administrator.

When two network trackers $A_{1}$ and $A_{2}$ are connected indirectly through
other network trackers $B_{j}$, if $A_{1}$ and $A_{2}$ use a common swarm and
$B_{j}$ doesn't use this swarm, then the $A_{i}$ trackers cannot unify unless
the border trackers are specified in the configuration. This happens because
the configured border trackers must unify with any swarm, although they do not
have peers connected from that swarm.

Grouping trackers in networks increases system scalability, but also network
convergence time. The update timers can be set to a lower value for border
trackers to limit convergence overhead.  The system administrator should opt
between scalability and convergence and adapt the protocol parameters to the
specific topology.

\section{Protocol Implementation}
\label{sec:unified-tracker:implementation}

TSUP is currently implemented in the popular \textit{XBT
Tracker}\footnote{\url{http://xbtt.sourceforge.net/tracker/}}, implemented in C++.
The extended TSUP capable tracker was dubbed \textit{XBT Unified Tracker}.

The original tracker implements an experimental UDP BitTorrent protocol
known as \textit{UDP Tracker}. Because TSUP also uses UDP and communication
takes place using the same port, TSUP-specific packets use the same header
structure as the UDP Tracker, enabling compatibility.

XBT Tracker uses a MySQL database for configuration parameters and for
communication with a potential front end. XBT Unified Tracker adds parameters
for configurations that are specific to TSUP and uses a new table in order to
remember links with other trackers and their parameters.  \textit{Tracker
awareness}, as described in
Section~\ref{subsec:unified-tracker:tracker-awareness}, is implemented in the
database.

Besides the HTTP \textit{announce} and \textit{scrape} URLs, the original
tracker uses other web pages for information and debugging purpose. The
unified tracker adds two extra information web pages for \textit{monitoring}.
The \textit{trackers web page} shows details about every link and the state of
the connection for that link. For every swarm, the \textit{swarms web page}
shows the list of peers and the list of trackers connected to that swarm.

As any server, XBT tracker uses an infinite loop waiting for the arrival of
packets from clients. Depending on the host system configuration, the server
uses \texttt{select} or \texttt{epoll}. In order to receive UDP packets, the
\texttt{recv} function is being used. This function is called when and event
is generated through \texttt{select} or \texttt{epoll}. Our methods are used
to process UDP packets; they are dubbed \texttt{process\_\ldots} and
\texttt{send\_\ldots}.

The protocol sends packets to other trackers either as a reply to some packets
received, or events generated, such as timer expiry. Both \texttt{select} and
\texttt{epoll} are unblocked after 5 seconds, in case of no event being
generated on the socked. Due to this, any protocol interval may consume at
most 5 seconds, adding extra possible delays of the running code.

A sample of the TSUP timer code is shown in
Listing~\ref{lst:unified-tracker:tsup-timer}.

\lstset{language=C,caption=TSUP
Timer,label=lst:unified-tracker:tsup-timer}
\begin{lstlisting}
/**
 * Check TSUP timers and if some of them expired take the appropriate actions.
 */
void Cserver::tsup_timers()
{
    ...

    //*_ On update timeout build the updates and mark this event
    if(time() - m_update_time > m_config.m_update_interval) {
        b_update_timeout = true;
        m_update_time = ::time(NULL); //*_ Reset update timer
    }
    else
        b_update_timeout = false;
    ...
}
\end{lstlisting}

The Cserver class, the main class of the implementation, had been used to
implement the protocol. The structure in
Listing~\ref{lst:unified-tracker:tracker-structure} defines components of
another trakcer to which there is a connection and stores the state of the
link in the \texttt{status} field.

\lstset{language=C,caption=Tracker Structure,label=lst:unified-tracker:tracker-structure}
\begin{lstlisting}
/**
 * Information about a tracker.
 */
struct t_tracker
{
    t_tracker();
    void clean_up_sent_peers();

    void clear();

    // From DB

    int tid;
    char name[256];
    in_addr_t host;
    string str_host;
    int port;
    int nat_port;
    char description[256];
    bool nat;
    bool external;
    int status;
    time_t recv_time;
    int retry_times;
    int reconnect_time;
    long long connection_id;
    ...
    int mandates;
};
\end{lstlisting}

A list of such structures forms the set of links to other trackers. Original
data structures have veen update to satisfy the new protocol needs. For
example, the \texttt{t\_file} structure, shown in
Listing~\ref{lst:unified-tracker:file-structure} is used to define information
about a \texttt{.torrent} file. It stores a list of pointers to
\texttt{t\_tracker} files. This structure also stores swarm leaders: the
internal network (\texttt{swarm\_leader}) and the external network
(\texttt{swarm\_leader\_external}).

\lstset{language=C,caption=File Structure,label=lst:unified-tracker:file-structure}
\begin{lstlisting}
struct t_file
{
    //* TSUP members:
    std::set<t_tracker *> trackers;
    t_tracker *swarm_leader;
    t_tracker *swarm_leader_external;
    bool new_arrivals;
};
\end{lstlisting}

Another important structure that was updated was \texttt{t\_peer} consisting
of information about a swarm peer. New peers had been added when announced
from another tracker.

The \texttt{CServer} class has also implemented a set of methods required for
the protocol. Such methods are \texttt{unify\_swarm} and
\texttt{unify\_swarms} for enablingunifications, \texttt{update},
\texttt{build\_update}, \texttt{build\_updates},
\texttt{append\_update\_peer}.

\section{Experimental Setup}
\label{sec:unified-tracker:setup}

TSUP testing activities used a virtualized infrastructure and a
Peer-to-Peer testing framework running on top of it. We were able to deploy
scenarios employing various swarms, ranging from a 4-peer and 1-tracker swarm
to a 48-peer and 12-tracker swarm. Apart from testing and evalution, the
infrastructure has been used to compare the proposed tracker overlay network
with classical swarms using a single tracker and the same number of peers. We
will show that a unified swarm has similar performance when compared to a
single tracker (classical) swarm.

In order to deploy a large number of peers we have used a thin virtualization
layer employing OpenVZ\footnote{\url{http://wiki.openvz.org}}. OpenVZ is a
lightweight solution that allows rapid creation of virtual machines (also
called containers). All systems are identical with respect to hardware and
software components. The deployed experiments used a single OpenVZ container
either for each tracker or peer taking part in a swarm. A virtualized network
has been build allowing direct link layer access between systems -- all
systems are part of the same network; this allows easy configuration and
interraction.

The experiments made use of an updated version of hrktorrent, a lightweight
application built on top of libtorrent-rasterbar. Previous
experiments~\cite{bt-vi} have shown libtorrent-rasterbar outperforming other
BitTorrent experiments leading to its usage in the current experiments. The
experiments we conducted used a limitation typical to ADSL2+ connections (24
Mbit download speed limitation, 3 Mbit upload speed limitation).

\begin{figure}[h]
  \begin{center}
    \includegraphics[width=0.7\textwidth]{src/img/unified-tracker/tsup-sample-run-48peers}
  \end{center}
  \caption{Sample Run Graphic}
  \label{fig:unified-tracker:tsup-sample-run}
\end{figure}

An automatically-generated sample output graphic, describing a 48 peer session
(12 seeders, 36 leechers, 12 trackers) sharing a 1024 MB file is shown in
Figure~\ref{fig:unified-tracker:tsup-sample-run}. The image presents download
speed evolution for all swarm peers. All of them are limited to 24 Mbit
download speed and 3 Mbit upload speed.

All peers use download speed between 2 Mbit and 5 Mbit on the first 2000
seconds of the swarm's lifetime. As the leechers become seeders, the swarm
download speed increases rapidly as seen in the last part of the swarm's
lifetime, with the last leechers reaching the top speed of 24 Mbit.

\section{Experimental Results and Evaluation}
\label{sec:unified-tracker:results}

In order to test the overhead added by TSUP to BitTorrent protocol, we have
made a set of test scenarios which compare the average download speed for a
swarm with unified trackers and for another swarm with just one non-unified
tracker, but the same number of leechers and seeders. Each test scenario is
characterized by the shared file sizes, the number of peers and, in the case
of tests with unified trackers, by the number of trackers. We shared 3 files
of sizes 64MB, 256MB and 1024MB.

In the test scenarios with unified trackers for each file we tested the swarm
with 1, 2, 4, 8 and 12 trackers. On each tracker there were connected 4 peers,
from which 1 is a seeder and 3 are leechers. So, for example, in a scenario
with 8 trackers there are 8 seeders and 24 leechers, totalizing 32 peers.
Having 3 files and 12 trackers in the biggest scenario we needed to create 36
.torrent files, because for each shared file we made a .torrent file for each
tracker. In the corresponding test scenarios with non-unified trackers, there
is just one .torrent file for each shared file. We varied the numbers of
seeders and leechers connected to the tracker so that they have the same
cardinality with the corresponding unified trackers scenarios.

Each test scenario has been repeated \textit{20 times} in order to allow
statistical relevance. The average download speed was calculated as an average
value from the 20 sessions.

Results may be seen in Table~\ref{tab:unified-tracker:comparison}, which
depicts the results for each file size, in the top (64MB), middle (256MB) and
bottom part of it (1024MB), respectively. For each of this two situations the
mean download speed (``mean dspeed'') and relative standard deviation
(``rel.st.dev.'') is depicted. In the right part, titled ``perf. Decrease''
(performance decrease), shows the percent of download speed decrease induced
by the overhead of the TSUP. In the left side of the table the number of
seeders and leechers for each scenario is shown. The percentage value for
download speed decrease is computed using the standard formula:

%\vspace{3mm}

\begin{equation}\label{updates-swarmleader}
d = 100\% \cdot \frac{ds_{SingleTracker} - ds_{UnifiedTrackers}}{ds_{SingleTracker}} ,
\end{equation}

%\vspace{2mm}

where \textit{ds} is an abbreviation from mean download speed.

%\vspace{3mm}

All positive percentage values from the ``perf. decrease'' header mark a
decrease of performance caused by TSUP overhead. A negative download speed
decrease percentage shows that there is an increase instead of a decrease.

The unification which takes place in XBT Unified Tracker introduces an
overhead in the BitTorrent protocol in comparison to XBT Tracker. In theory
the performance decrease must always be positive. But there are situations
where percentages are negative, which could suggest that TSUP increases
the speed, thus reducing the download time. But this performance increase is
not due to TSUP, but is caused by another fact. In all scenarios, in the
Single Tracker experiments, at the beginning, all peers are started almost
simultaneously, creating a flash crowd. In this situation, communication
between peers will start immediately, but when multiple trackers are unified,
the TSUP imposes a delay before each peer finds out of all the others, because
of the convergence time.  It is known that sometimes it is better when peers
enter the swarm later~\cite{bt-analysis}, explaining the presence of negative
values for the performance decrease.

%\begin{figure}[h]
%  \begin{center}
%    \includegraphics[width=\columnwidth]{src/img/unified-tracker/table-results.png}
%  \end{center}
%  \caption{Tracker Networks}
%  \label{fig:unified-tracker:table-results}
%\end{figure}

\def \singlesods {337.73}
\def \singlestds {472.27}
\def \singlesfds {475.13}
\def \singleseds {476.10}
\def \singleswds {470.53}
\def \singlesorsd {1.28}
\def \singlestrsd {0.87}
\def \singlesfrsd {1.87}
\def \singlesersd {6.06}
\def \singleswrsd {9.61}

\def \singletods {358.72}
\def \singlettds {476.59}
\def \singletfds {477.56}
\def \singleteds {492.40}
\def \singletwds {486.04}
\def \singletorsd {0.45}
\def \singlettrsd {0.26}
\def \singletfrsd {0.45}
\def \singletersd {6.64}
\def \singletwrsd {8.33}

\def \singleoods {365.27}
\def \singleotds {477.47}
\def \singleofds {477.54}
\def \singleoeds {423.71}
\def \singleowds {407.71}
\def \singleoorsd {0.19}
\def \singleotrsd {0.15}
\def \singleofrsd {0.18}
\def \singleoersd {6.19}
\def \singleowrsd {4.47}

\def \multiplesods {335.11}
\def \multiplestds {396.55}
\def \multiplesfds {463.42}
\def \multipleseds {497.35}
\def \multipleswds {496.48}
\def \multiplesorsd {2.87}
\def \multiplestrsd {16.64}
\def \multiplesfrsd {12.60}
\def \multiplesersd {11.10}
\def \multipleswrsd {11.48}

\def \multipletods {356.55}
\def \multiplettds {407.55}
\def \multipletfds {477.45}
\def \multipleteds {500.56}
\def \multipletwds {494.93}
\def \multipletorsd {1.07}
\def \multiplettrsd {14.67}
\def \multipletfrsd {10.73}
\def \multipletersd {8.49}
\def \multipletwrsd {12.38}

\def \multipleoods {365.57}
\def \multipleotds {437.09}
\def \multipleofds {466.03}
\def \multipleoeds {418.69}
\def \multipleowds {418.76}
\def \multipleoorsd {0.31}
\def \multipleotrsd {9.12}
\def \multipleofrsd {5.93}
\def \multipleoersd {8.73}
\def \multipleowrsd {6.27}


\begin{sidewaystable}
  \centering
  \caption{Single Tracker vs. Unified Trackers Comparison}
  \label{tab:unified-tracker:comparison}
  \begin{tabular}{@{}rrcccrccc@{}}
    \toprule
       & & \multicolumn{2}{c}{\textbf{Single Tracker}} & &
       \multicolumn{3}{c}{\textbf{Unified
       Trackers}} & \\
    \cmidrule{3-4} \cmidrule{6-8}
      \textit{seeders} & \textit{leechers} & \textit{mean dspeed (KB/s)} &
      \textit{rel.st.dev. (\%)} & & \textit{trackers} & \textit{mean dspeed (KB/s)} &
      \textit{rel.st.dev. (\%)} & \textit{perf. decrease (\%)} \\
    \midrule
      $size = 64MB$\\
      1 & 3 & \singlesods & \singlesorsd & & 1 & \multiplesods &
      \multiplesorsd & 0.78 \\  %01
      2 & 6 & \singlestds & \singlestrsd & & 6 & \multiplestds &
      \multiplestrsd & 16.03 \\  %04
      4 & 12 & \singlesfds & \singlesfrsd & & 4 & \multiplesfds &
      \multiplesfrsd & 2.46 \\  %07
      8 & 24 & \singleseds & \singlesersd & & 8 & \multipleseds &
      \multiplesersd & -4.46 \\  %10
      12 & 36 & \singleswds & \singleswrsd & & 12 & \multipleswds &
      \multipleswrsd & -5.52 \\ %13
      $size = 256MB$\\
      1 & 3 & \singletods & \singletorsd & & 1 & \multipletods &
      \multipletorsd & 0.6 \\   %02
      2 & 6 & \singlettds & \singlettrsd & & 6 & \multiplettds &
      \multiplettrsd & 14.49 \\   %05
      4 & 12 & \singletfds & \singletfrsd & & 4 & \multipletfds &
      \multipletfrsd & 0.02 \\  %08
      8 & 24 & \singleteds & \singletersd & & 8 & \multipleteds &
      \multipletersd & -1.66 \\  %11
      12 & 36 & \singletwds & \singletwrsd & & 12 & \multipletwds &
      \multipletwrsd & -1.83 \\ %14
      $size = 1024MB$\\
      1 & 3 & \singleoods & \singleoorsd & & 1 & \multipleoods &
      \multipleoorsd & -0.08 \\   %03
      2 & 6 & \singleotds & \singleotrsd & & 6 & \multipleotds &
      \multipleotrsd & 8.46 \\   %06
      4 & 12 & \singleofds & \singleofrsd & & 4 & \multipleofds &
      \multipleofrsd & 2.41 \\  %09
      8 & 24 & \singleoeds & \singleoersd & & 8 & \multipleoeds &
      \multipleoersd & 1.18 \\  %12
      12 & 36 & \singleowds & \singleowrsd & & 12 & \multipleowds &
      \multipleowrsd & -2.71 \\ %15
    \bottomrule
  \end{tabular}
\end{sidewaystable}


From results in Table~\ref{tab:unified-tracker:comparison} several
conclusions are drawn.  The TSUP overhead decreases, on the
first hand, when the number of peers increases (and proportionally the number
of seeders) and, on the other hand, when the file size increases. When the
overhead is insignificant, the percentages have lower values. TSUP convergence
time causes the avoidance of a flash crown at the beginning of each scenario,
thus inducing a small performance increase. Starting from 8 seeders and more
TSUP performance decrease becomes smaller than the performance increase caused
by avoiding the flash crowd. That is why some performance decrease values are
negative. The relative standard deviation is generally increasing with the
number of peers, but is decreasing when the file size increases. The values
can be considered normal, taking into account the number of peers that are
part of a swarm.

Due to the small values of performance decrease and relative standard
deviation, we concluded that TSUP overhead is insignificant for small to
medium-sized swarms (less than 50 peers) which share big files (1GB).
BitTorrent is generally used for sharing large files and TSUP allows the
increase of swarms size; these two factors come as an advantage for this
technology.

Swarm unification increases the number of peers for a shared file, but this
fact does not always grant a larger download speed, as it can be seen in
Table~\ref{tab:unified-tracker:comparison}. However, having a swarm with a
bigger number of peers has three advantages. First of all, it increases the chance
that more seeders will later be available and a big proportion of stable
seeders increase download speed. The second reason is that bigger swarms
increase a shared file's availability by creating redundancy. The third reason
is
that a bigger swarm is more attractive for new users, giving the possibility
of creating a big network which may be used for features such as social
networking.

\section{Conclusion}
\label{sec:unified-tracker:conclusion}

A novel overlay network protocol on top of BitTorrent, aiming at integrating
peers in different swarms, has been presented. Dubbed TSUP (Tracker Swarm
Unification Protocol), the protocol is used for creating and maintaining a
tracker network enabling peers in swarms to converge in a single swarm. Each
initial swarm is controlled by a different tracker; trackers use the overlay
protocol to communicate with each other and, thus, take part in a greater
swarm.

We have used an OpenVZ-based Peer-to-Peer testing infrastructure to create a
variety of scenarios employing an updated version of the XBT Tracker, dubbed
XBT Unified Tracker. The protocol incurs low overhead and overall performance.
The unified swarm is close to the performance of single-tracker swarms
consisting of the same number of seeders and leechers with the benefit of
increased number of peers, which boosts download speed. The increased number of peers provides the basis
for improved information for various overlays (such as social networks) and
allows a healthier swarm -- given enough peers, if some of them decide to
leave the swarm, some peers will still take part in the transfer session.

Experiments have involved different swarms, with respect to the number of
peers and trackers, and different file sizes using simulated asymmetric links.
The table at the end of the chapter shows a summary of results, providing
insight to the fact that the protocol has a low overhead.

At this point each tracker uses a statically defined pre-configured list of
neighboring trackers. One of the main goals for the near future is to enable
dynamic detection of neighboring trackers and ensure extended scalability. We
are currently considering two approaches: the use of a tracker index where
trackers' IP/host addresses and ports are stored or the use of a completely
decentralized tracker discovery overlay similar to DHT's discovery methods.

As proof of concept, our test scenarios have focused on homogeneous swarms.
All peers in swarms are using the same implementation and the same bandwidth
limitation. All peers enter the swarm at about the same time, with some delay
until swarm convergence, in case of the unified tracker protocol. We plan to
create heterogeneous swarms that use different clients with different
characteristics. The number of seeders and leechers in initial swarms are also
going to be altered and observe the changes incurred by using the unification
protocol.
