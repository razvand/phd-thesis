% vim: set tw=78 tabstop=4 shiftwidth=4 aw ai:

\chapter{Introduction}
\label{chapter:intro}

\textit{``Make it run, then make it right, then make it fast.'' -- Kent Beck}

Technology is strongly related to progress. As time goes, technology is the
prime ingredient for progress and progress forces technology to adapt itself,
to become better and more useful, to surpass its original design and provide
new outcomes.

Computer Science and Computer Engineering are among the most dynamic fields of
human knowledge with continuous transformation of the technology, principles
and scope it enables.

Among the myriad of inventions, discoveries and contributions to the
development of human civilization, the Internet stands out as one of the
most impressive and extraordinary. With only a few decades of existence, the
Internet has managed to provide the means for uniting the entire planet, for
ensuring access to knowledge, information and digital resources. The Internet
is also a catalyst for all major technological advances in many other fields
of study where researchers and engineers alike have been challenged to provide
better services, better connectivity, better user experience, better
performance and increased diversity.

On the frontier between two millennia, when the Internet had become a main
stream and de facto service, a new technology emerged, fulfilling the need for
data exchange among edge users: Peer-to-Peer Systems.

The exploding evolution of the Internet had provided users everywhere with
higher and higher bandwidth. The similarly explosive evolution of computer
technology ensured faster CPUs, larger hard disks, more memory. As even fringe
users had access to a good quality resources, the need to easily interconnect
and share information became preeminent. Peer-to-Peer Systems thus filled the
vacuum technological space enabled by the ever progressing Internet
technology.

Peer-to-Peer technology is as early as the Internet. In the beginnings of the
Internet, all stations had limited power and limited bandwidth and had to
cooperate to achieve a common goal. A prime example of this is the email
service. Peer-to-Peer have reemerged in late 90s when Napster became a true
powerful service in use by multiple users. Through more than 10 years of
research, engineering and updates, Peer-to-Peer concepts, applications and
protocols have managed to become common things nowadays.

As one of the most highly successful and most heavily used protocol in the
Internet, the BitTorrent protocol has captured the attention of research and
commercial entities alike. With a ``tit-for-tat'' algorithm at its base, the
BitTorrent protocol is nowadays the de facto protocol used for large file
distribution, offering advantages such as high speed, uniform bandwidth
consumption, low load on stations and extensibility.

The BitTorrent protocol is, in this thesis author's opinion, an example of a
protocol that had been made to work, and then has been made to work right.
Simple, yet powerful, BitTorrent shines when used for large content
distribution.

It is around the BitTorrent protocol that this thesis is centered around. Our
aim, as described in the thesis chapters is to take this protocol, run
measurements, analyze it and make it better: make it faster, enhance its
performance, its usability, broaden its scope. We do not aim to fill every
possible spot of improvement, but rather signal relevant aspects to be taken
into account to ensure increased performance.

\section{Thesis Objectives}
\label{sec:intro:objectives}

When involved in Peer-to-Peer systems, updates, enhancements and improvements
have to take into account the rules, diversity and particularities of such
systems. Carefully crafted experiments, sensible measurements, reasonable
conclusions, useful proposals are important components of research activities
in this domain. As such the objectives of the thesis are:

\begin{itemize}
  \item provide a state of the art of Peer-to-Peer technology: protocols,
  applications, scope;
  \item describe the process of bringing up a complete infrastructure for
  deploying realist scenarios;
  \item advice on relevant trials that may be deployed and subsequently
  analyzed;
  \item provide information regarding metrics, measured data, collecting and
  interpreting information provided by peers in P2P networks;
  \item provide insight on the inner workings of content streaming swarm when
  using P2P protocols, BitTorrent in particular;
  \item describe the particularities of deploying streaming trials and the
  technologies to use;
  \item offer proposals for possible enhancements to P2P protocols and
  architecture, with design, implementation and evaluation components.
\end{itemize}

Providing proper evaluation of deployed trials information is critical to the
relevance of collected data. An important part of the thesis is dedicated to
evaluating architectures, protocols, actual implementations, internal
messages, network topologies with the help of versatile infrastructures.
Monitoring, log processing, internal processing are all employed to allow
in-depth view and analysis of P2P protocols.

As much of the work highlighted in this paper has been part of the
P2P-Next project, extensive focus has been given to using P2P technology for
providing content streaming. With BitTorrent at the focus of the research in
place, various adaptations and extensions have been analyzed. BitTorrent's
rarest-piece-first strategy is not suitable for streaming and adaptations to
the piece-picker policy have to be taken into account. The NextShare
technology, used in the P2P-Next project is the core of analysis and
measurements.

Measurements, evaluation and analysis were driven at providing better network
architectures and protocols for improved performance in P2P systems. A tracker
overlay that ensures swarm unification is a proposal for providing healthy
swarms and content unification. Providing a proper inter-tracker communication
protocol, marking relevant measurements and evaluation is part of the objective
of providing enhancements to existing solution. As is the kernel-based
implementation of a BitTorrent-alternative dubbed swift, designed from ground
up as a Transport-layer protocol.

\section{Thesis Scope}
\label{sec:intro:scope}

\todo{P2P, BitTorrent, motivation, run - collect - analyze - bring advice, low
level information, protocol messages, protocol improvements, comparison}

\section{Related Work}
\label{sec:intro:related}

\todo{measurements, streaming, swift}

\section{Context}
\label{sec:intro:context}

\todo{P2P-Next, BitTorrent, swift, multiparty, LivingLab, streaming,
virtualization, logging}

\section{Contents}
\label{sec:intro:contents}

\todo{present chapters}
