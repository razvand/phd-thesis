% vim: set tw=78 tabstop=4 shiftwidth=4 aw ai:

\chapter{Introduction}
\label{chapter:intro}

\textit{``Make it run, then make it right, then make it fast.'' -- Kent Beck}

Technology is strongly related to progress. As time goes by, technology is the
prime ingredient for progress. Progress, in its turn, forces technology to
adapt itself, to become better and more useful, to surpass its original design
and provide new outcomes.

Computer Science and Computer Engineering are among the most dynamic fields of
human knowledge with continuous transformations of the technology, scope and
principles it enables.

Among the myriad of inventions, discoveries and contributions to the
development of human civilization, the Internet stands out as one of the
most impressive and extraordinary. With only a few decades of existence, the
Internet has managed to provide the means for uniting the entire planet, for
ensuring access to knowledge, information and digital resources. The Internet
is also a catalyst for all major technological advances in many other fields
of study where researchers and engineers alike have been challenged to provide
better services, better connectivity, better user experience, better
performance and increased diversity.

On the frontier between two millennia, when the Internet had become a
mainstream technology and de facto service, a new technology emerged,
fulfilling the need for data exchange among edge users: Peer-to-Peer Systems.

The exploding evolution of the Internet had provided users everywhere with
higher and higher bandwidth. The similarly explosive evolution of computer
technology ensured faster CPUs, larger hard disks, more memory. As even fringe
users had access to a good quality resources, the need to easily interconnect
and share information became preeminent. Peer-to-Peer Systems thus filled the
vacuum technological space enabled by the ever progressing Internet
technology.

Peer-to-Peer technology is as early as the Internet. In the beginnings of the
Internet, all stations had limited power and limited bandwidth and had to
cooperate to achieve a common goal. A prime example of this is the email
service. Peer-to-Peer have reemerged in late 90s when
Napster\footnote{\url{http://www.napster.com/}}
became a true powerful service in use by multiple users. Through more than 10
years of research, engineering and updates, Peer-to-Peer concepts,
applications and protocols have managed nowadays to become common knowledge.

As one of the most highly successful and most heavily used protocol in the
Internet\footnote{\url{http://www.sandvine.com/news/global\_broadband\_trends.asp}},
the BitTorrent protocol~\cite{bittorrent-cohen} has captured the
attention of research and commercial entities alike. With a
``tit-for-tat''\footnote{\url{http://www.gametheory.net/dictionary/TitforTat.html}} algorithm at its base, the BitTorrent protocol
is nowadays the de facto protocol used for large file distribution, offering
advantages such as high speed, uniform bandwidth consumption, low load on
stations and extensibility.

The BitTorrent protocol is, in this thesis author's opinion, an example of a
protocol that had been made to work, and then has been made to work right,
though, as every protocol, has its own downsides~\cite{bittorrent-trade-offs}.
Simple, yet powerful, BitTorrent shines when used for large content
distribution and has subsequently been adapted for other features such as
streaming~\cite{bittorrent-streaming}.

It is around the BitTorrent protocol that this thesis is centered around. Our
aim, as highlighted in the thesis, is to take the protocol, run measurements,
analyze it and make it better: make it faster, enhance its performance, its
usability, broaden its scope. We do not aim to fill every possible spot of
improvement, but rather signal relevant aspects to be taken into account to
ensure increased performance.

\section{Thesis Objective}
\label{sec:intro:objective}

When involved in Peer-to-Peer systems, updates, enhancements and improvements
have to take into account the rules, diversity and particularities of such
systems. Carefully crafted experiments, sensible measurements, reasonable
conclusions, useful proposals are important components of research activities
in this domain.

The objective of this thesis is providing a series of improvements to
Peer-to-Peer systems at protocol level, either through updating and enhancing
existing protocols or designing and implementing new ones. Formal and
experimental evaluations are employed to provide arguments regarding the
advantages of these updates. Providing improvements is based on careful
trial deployment and analysis of Peer-to-Peer protocols. The use of testing
environments and the employment of measurements and evaluation techniques are
critical to the successful providing of protocol updates and improvements. As
a secondary objective, the thesis highlights the methods and mechanisms for
creating realistic, scalable and automated trial environments and the
approaches available for collecting, measuring and interpreting Peer-to-Peer
protocol parameters.

Providing proper evaluation of deployed trials information is critical to the
relevance of collected data. An important part of the thesis is dedicated to
evaluating architectures, protocols, actual implementations, internal
messages, network topologies with the help of versatile infrastructures.
Monitoring, log processing, internal processing are all employed to allow
in-depth view and analysis of P2P protocols.

As much of the work highlighted in this paper has been part of the
P2P-Next Project\footnote{\url{http://www.p2p-next.org}}, extensive focus has been given to using P2P
technology for providing content streaming. With BitTorrent at the focus of
the research in place, various adaptations and extensions have been analyzed.
BitTorrent's rarest-piece-first strategy is not suitable for streaming and
adaptations to the piece-picker policy have to be taken into account. The
NextShare technology, used in the P2P-Next project is the core of analysis and
measurements.

Measurements, evaluation and analysis were driven at providing better network
architectures and protocols for improved performance in P2P systems. A tracker
overlay that ensures swarm unification is a proposal for providing healthy
swarms and content unification. Providing a proper inter-tracker communication
protocol, marking relevant measurements and evaluation is part of the objective
of providing enhancements to existing solution. As is the kernel-based
implementation of a BitTorrent-alternative dubbed swift, designed from ground
up as an OSI-stack Transport-layer protocol.

\section{Thesis Scope}
\label{sec:intro:scope}

In accordance with the stated objective of providing improvements and updates
to Peer-to-Peer systems (as also stated in the title), this thesis primarily
defines its scope in the field of Peer-to-Peer systems, with an overal view
present in Chapter~\ref{chapter:p2p-systems}. In particular, most of
the research and development of this work is centered around the BitTorrent
protocol, one of the most popular protocols in the Internet.

Motivated by the desire to provide new and useful updates to existing
applications and creating new ones, the approach employed is one that
carefully deploys and monitors Peer-to-Peer applications, collects protocol
parameters, puts them under scrutiny and analysis and provides the input
required for improvements, extensions and updates.

While multiple approaches may be considered for client, protocol, network and
swarm analysis, the approach chosen relies on low level information,
considered to be the basis of protocol parameters. We are not (directly)
concerned with the overall view of the Peer-to-Peer system, but rather with an
in depth view of the client behavior. Hook points in Peer-to-Peer client
implementations, client logging or network traffic analysis provide the means
to gather the valuable low level Peer-to-Peer protocol information.

Protocol messages contain valuable information such as client download rate,
client upload rate, number of connected peers, protocol events. These
parameters are analysed and conclusions are drawn allowing for protocol
improvements to be proposed, analyzed and then put to use. An important
approach is using collected parameters for comparison between various
implementations, protocols or behavior in diverse situations.

Protocol improvements may result in updates to existing protocols and
solutions as is the case with the tracker overlay presented in
Chapter~\ref{chapter:unified-tracker} or the design of a novel protocol as is
the case with the multiparty protocol implementation in the Linux kernel
detailed in Chapter~\ref{chapter:multiparty}.

\section{Related Work}
\label{sec:intro:related}

Measurements and improvements in Peer-to-Peer systems have been the focus of
commercial and scientific players alike. Thorough analysis, a diversity of
approaches, different interests have sparked a plethora of novel mechanisms,
architectures, updates and extensions to existing applications as well as new
and improved implementations to both old and new problems. Recent years have
shown particular interest in the streaming capabilities of Peer-to-Peer
solutions, resulting in a surge of updates, extensions and trials. Novel
protocols such as
swift~\footnote{\url{http://tools.ietf.org/html/draft-grishchenko-ppsp-swift-03}} are proof to the continuous birth of new
protocols in the ever evolving Internet.

Most measurements and evaluations involving the BitTorrent protocol and
applications are either concerned with the behavior of a real-world swarm or
with the internal design of the protocol. There has been little focus on
creating a self-sustained swarm management environment capable of deploying
hundreds of controlled peers, and subsequently gathering results and
interpreting them.

The PlanetLab\footnote{\url{http://www.planetlab.org}} infrastructure provides a
realistic testbed for Peer-to-Peer experiments. PlanetLab nodes are connected
to the Internet and experiments have a more realistic testbed where delays,
bandwidth and other are subject to change. Tools are also available to aid in
conducting experiments and data collection~\cite{bittorrent-planetlab}.

Pouwelse~et~al.~\cite{measurement-study} have done extensive analysis on the
BitTorrent protocol using large real-world swarms focusing on the overall
performance and user satisfaction. Guo~et~al.~\cite{guo} have modeled the
BitTorrent protocol and provided a formal approach to its functionality.
Bharambe~et~al.~\cite{bt-analysis} have done extensive studies on improving
BitTorrent's network performance.

Garbacki~et~al.~\cite{garbacki} have created a simulator for testing 2Fast, a
collaborative download protocol. The simulator is useful only for small swarms
that require control. Real-world experiments involved using real systems
communicating with real BitTorrent clients in the swarm.

A testing environment involving four major BitTorrent trackers for measuring
topology and path characteristics has been deployed by
Iosup~et~al.~\cite{corr-overlay}. They used nodes in PlanetLab. The
measurements were focused on geo-location and required access to a set of
nodes in PlanetLab. The authors' efforts are directed towards correlating
characteristics of BitTorrent and its Internet underlay, with focus on
topology, connectivity, and path-specific properties. For this purpose they
designed and implemented \textit{Multiprobe}, a framework for large-scale P2P
file sharing measurements. The main difference between their implementation
and our approach is that we focus on an in-depth client-level analysis and not
on the whole swarm.

Dragos~Ilie~et~al.~\cite{p2p-traf-meas} developed a measurement infrastructure
with the purpose of analyzing P2P traffic. The measurement methodology is
based on using application logging and link-layer packet capture.

Meulpolder~et~al.~\cite{p2p09} present a mathematical model for
bandwidth-inhomogeneous BitTorrent swarms. Based on a detailed analysis of
BitTorrent's unchoke policy for both seeders and leechers, they study the
dynamics of peers with different bandwidths, monitoring their unchoking and
uploading/downloading behavior. Their analysis showed that having only peers
with the same bandwidth is not enough to determine in-depth the peers'
behavior. In those experiments they split the peers into two bandwidth classes
- slow and fast - and they observed that slow ones usually unchoked other slow
peers, their data being transfered from fast peers. Although they do not offer
precise details about the experimental part of monitoring unchoking behavior
and transfers rates, their work relates to what we intend to do: put to use
the BitTorrent logging messages that a Peer-to-Peer network generates, parses
and stores.

While Meulpolder~et~al.~\cite{p2p09} provide a peer-level analysis, another
approach is to study BitTorrent at tracker level, as described by
Bardac~et~al.~\cite{tracker-mon}. This paper implements a scalable and
extensible BitTorrent tracker monitoring architecture, currently used in the
Ubuntu Torrent Experiment\footnote{\url{http://torrent.cs.pub.ro/}} experiment
at University POLITEHNICA of Bucharest (UPB), the Computer Science and
Engineering Department. The system analyses the peer-to-peer network
considering both the statistic data variation and the geographical
distribution of data. This study is based on a similar infrastructure with the
one we use for our client and protocol level analysis.

As video-based traffic has continuously filled the Internet backbone, effort
has been invested in providing streaming functionality to Peer-to-Peer
solutions.  As Peer-to-Peer file sharing solutions use the ``offline
playback'' model, protocol and overlay network topology updates had to be
considered.

Guo~el~al.~\cite{p2p-streaming-survey} have put up a survey of available P2P
streaming solutions in late 2007. They created a state of the art of streaming
solutions and took into consideration two basic overlay topologies: tree-based
systems and mesh-based systems (similar to swarm). For each of them they
surveyed existing Peer-to-Peer solutions for VoD and for live streaming. They
concluded that P2P streaming systems are capable of streaming video to a large
population at low server cost and with minimal dedicated infrastructure.
However, they signaled fundamental limitations such as reduced Quality of
Experience and the non-ISP friendly status of P2P applications and.

An important paper that was referred to by Guo et.~al, when considering VoD on
mesh-based systems was ``BiToS: Enhancing BitTorrent for Supporting Streaming
Applications''. In the latter article, Vlavianos~et~al.~\cite{bitos}, propose
updates to the popular BitTorrent protocol to turn in into a VoD-friendly
protocol, which they refer to \textbf{view-as-you-download}. They acknowledge
the fact that BitTorrent is an adequate protocol for large data distribution
but lack support for basic streaming due to its ``rarest-piece-first'' piece
selection policy. Therefore, they split the pieces that need to be collected
into two sets: a high priority set and a remaining piece set. They attach a
probability to a given piece and conclude that an updated priority-enhanced
rarest piece first-based policy retains good performance specific to the
BitTorrent protocol and provides VoD streaming support, allowing the
construction of a \textbf{view-as-you-download} service.

Lessons from BiToS were used and enhanced in ``Give-to-Get'', an updated form
of ``tit-for-tat'' within BitTorrent developed at TUDelft by
Mol~et~al.~\cite{give-to-get}. Give-to-Get assumes altruistic behavior for
peers -- that is it will voluntarily give out pieces to a peer in hope that
peer will be reciprocal. Give-to-Get takes into account that a peer's good
history is mostly irrelevant as a peer is concerned with data that needs to be
played out ``now''; peer ranking is built up accordingly. The solution using
Give-to-Get makes to important updates to BitTorrent: the neighbor selection
policy and the piece selection policy. Give-to-Get borrows heavily from BiToS,
and uses priority sets for altering the piece selection policy. Unlike BiToS,
however, Give-to-Get uses three sets: a high priority set, a medium priority
set and low-priority set. Give-to-Get had been designed with fairness in mind
and the issue of free riding is addressed heavily in the paper.
Xing~et~al.~\cite{malicious-p2p} proposed methods to detect malicious peers in
a streaming network that may be used to increse the health of the swarm.
Somewhat similar, Lehriender~et~al.~\cite{mitigating-unfairness} proposed
mechanisms to mitigate unfairness in locality-aware Peer-to-Peer networks.

Mol~et.~al~\cite{design-p2p-live} had also worked on creating a live streaming
update to BitTorrent, as described in his technical paper ``The Design and
Deployment of a BitTorrent Live Video Streaming Solution''. As with
Give-to-Get, their solution was a mesh/swarm-based one and requested important
updates to the non-sequential BitTorrent protocol. A ``.tstream'' file had
been generated to describe content that, being live stream, had no
pre-determined length. A sliding window had been added for each peer, such
that information prior and before and certain point would be requested. The
initial seeder, also dubbed injector, provides the stream to other peers that
may, in turn, provide it to others. Dealing with the constraints and specifics
of a live streaming network, a seeder is redefined to being a peer which is
always trusted an unchoked by the injector and thus has access to the whole
stream. Seeders are required to ensure stability of the stream.

The latter two approaches (designed and implemented by
Mol~et.~al~\cite{give-to-get}~\cite{design-p2p-live}) form the basis for the
NextShare P2P streaming solution described in
Section~\ref{sec:multimedia-dist:nextshare}.

A novel approach to Peer-to-Peer file sharing, the swift
protocol\footnote{\url{http://libswift.org}} aims to transform the virtually
infinite capacity of nodes in the Internet in the ground for continuous file
sharing. While still in its infancy, the swift protocol has proven to be a
serious candidate in the Peer-to-Peer protocol market.

Inspired by the BitTorrent protocol, swift has been dubbed \textit{BitTorrent
at Transport layer}. Current implementation is user space based on UDP, but
progress has been made toward a complete implementation of swift in kernel
space. swift discards options it considers obsolete such as the use of TCP,
and thus ordered delivery and reliable transfer, and proposes new approaches
to storing and exchanging metadata such as binmaps~\cite{binmaps}, integrated
tracker and the effective use of Merkle hashes and trees.

One major asset of swift is the current effort (as of the writing of this
thesis) to create an IETF standardisation. There has been some progress but
the road is still ahead to transform swift into a fully fledged
Transport layer protocol to serve as a multiparty Peer-to-Peer protocol. A
current draft has been submitted to the IETF PPSP team.

\section{Context}
\label{sec:intro:context}

The context of this work is defined by several key components, ranging from
the EU FP7 P2P-Next project to protocols such as BitTorrent or swift, from
the streaming updates and LivingLab used for trials to approaches such as
virtualization and logging information that formed the basis for the
infrastructure used as experimental setup.

The \textbf{EU FP7 P2P-Next project}\footnote{\url{http://www.p2p-next.org}} is the
major component in the context of this work. Much of the effort and results in
the presented work have been part of the P2P-Next project. The project has
stated an ambitious goal of ``providing the next generation content delivery
platform''. With high ranked commercial and academic partners in Europe, it
incorporates P2P technology to provide streaming (both Video-on-Demand and
live streaming) for PC and for CE such as set-top boxes. Technology used by
the P2P-Next project is dubbed NextShare technology.

The \textbf{BitTorrent protocol} may be considered the flagship of nowadays
P2P file sharing networks. A highly successful and powerful
protocol\footnote{\url{http://www.sandvine.com/news/global\_broadband\_trends.asp}},
BitTorrent stands at the basis of the NextShare core of the P2P-Next project.
BitTorrent's success story has spawned a wild number of implementations
leading to our analysis of its inner workings for a variety of applications and
topologies. The BitTorrent protocol is easily identified by the two core
algorithms used for sharing content: \textit{tit-for-tat} for fair piece
exchange and \textit{rarest-piece-first} for the piece picking policy.

\textbf{swift} is a radical new approach to content distribution as
Peer-to-Peer. It discards many of the non-useful characteristics of TCP and
proposes a Transport layer protocol (in the OSI stack) that would be able to
intrinsically communicate with peers. It would not follow the usual end-to-end
approach as other Transport layer protocols, but rather create a
``multiparty'' approach where a request is simultaneously sent to multiple
peers. Facilities such as in-order delivery or reliability are discarded,
making swift act as as ``BitTorrent protocol at Transport layer''.

A significant component of this work, mostly highlighted in
Chapter~\ref{chapter:multimedia-dist} has been based around the local
\textbf{LivingLab}\footnote{\url{http://p2p-next.cs.pub.ro}}. A LivingLab is a
user-centric testing ground for use within the P2P-Next project. Major
partners of the project possess LivingLabs as a real environment for
evaluating, testing and comparing existing applications. The focus of the
local UPB LivingLab is comparison of various Peer-to-Peer streaming
implementations and classical BitTorrent distribution implementations.

Withing the LivingLab, the major keyword is \textbf{Peer-to-Peer streaming}.
A major interest in recent years, streaming requires important (and sometimes
severe) updates to existing protocols. The BitTorrent protocol, for example,
has to be updated and its ``rarest piece first'' policy is replaced by a set
of priority queues\cite{bitos}, some of which use a sequential piece selection
policy in order to ensure proper delivery and streaming playback, especially
when considering live streaming.

Testbeds and experimental setups heavily employed throughout this work have
made use of \textbf{virtualization} technology. Virtualization (especially
through the use of OpenVZ) has provided the means for deploying medium to
large scale swarms in a realistic, automated and efficient manner. As it is
difficult to provide a high enough number of systems for a realistic swarm,
virtualization offers the benefit of providing a close-to-real environment
on top of a much lower number of systems (by the order of tens)~\cite{p2p-va}.

Information required for thorough analysis of protocol behavior, messages and
parameters is typically collected through \textbf{client-based logging} (and,
to some extent, tracker logging). Client logging contains all relevant
information for analysis, with parameters such as download rate, upload rate,
connected peers, peer IDs, protocol events and protocol message types. This
information may be filtered and parsed to provide an easy to access, use and
process storage of Peer-to-Peer parameters.

\section{Contents}
\label{sec:intro:contents}

Thesis contents follow both a chronological order, when considering the work
employed, and a logical order as latter chapters rely on information presented
by former chapters. Forward and backward references are provided where
necessary, while always keeping an eye out for providing the initial
information in a previous chapter.

Chapter~\ref{chapter:p2p-systems} -- \textbf{Peer-to-Peer Systems} presents a
state of the art of Peer-to-Peer systems and implementations. It provides
insight on the birth of P2P systems, their evolution and current challenges.
Particular focus is attributed to the BitTorrent Protocol and content
distribution as they collectively form the context of this work. We insist on
streaming features for P2P systems as it is a relevant aspect presented in
this thesis. Issues and challenges are also identified and described.

The backbone of most of the experiments and trials involved is the
virtualized network infrastructure and features presented in
Chapter~\ref{chapter:virt-infra}. We argument the benefits of using
virtualization for creating P2P environments and highlight the characteristics
of the employed infrastructure. Thorough information is provided about
automating the deployment of the infrastructure, Peer-to-Peer clients and
gathering results.

We have designed, developed and deployed a novel virtualization infrastructure
providing such features as: scalability, extensibility, automation. Based on
OpenVZ and using a plethora of tools, the infrastructure has been used to
create experimental setups for scenarios including performance measurements,
implementation comparison, streaming versus classical distribution. We propose
a set of metrics that provide performance information about various
virtualization solutions, including OpenVZ, Xen and LXC.

Chapter~\ref{chapter:proto-measure} is concerned with methods and approaches
of gathering information from Peer-to-Peer clients, parsing that information
in protocol parameters and subjecting it to analysis. Two approaches used for
collecting logging information are the implementation of a generic library
whose API is ``hooked'' in existing applications and the usage of client
logging data. A custom processing framework is provided to interpret data
provided by clients. Information may be analyzed either subsequent to the log
collection activity (post-processing) or in the same time (live/real-time).
The latter approach may be used to provide monitoring.

The generic library has been integrated in the libtorrent-rakshasa and
Transmission BitTorrent clients, while the custom processing framework has
been integrated with a variety of clients, due to its less intrusive nature.
Storage, parsing and analysis components allow analysis and measurements of
BitTorrent swarms. We propose a formal analysis framework based on client
speed parameters; it is focused on providing peer and swarm performance
information.

A first improvement to BitTorrent swarms is the design, implementation,
testing and evaluation of a novel protocol that allows swarm unification, as
described in Chapter~\ref{chapter:unified-tracker}. The protocol, dubbed TSUP
(\textit{Tracker Swarm Unification Protocol}) allows swarms that part the same
.torrent file, yet are different, to be unified, such that peers would be able
to communicate one with the other even if initially in different swarms.
Intermediation is enabled by trackers and the new protocol forms a tracker
overlay network -- each tracker is initially part of the single swarm.

TSUP has been designed from scratch and has been implemented into the XBT
tracker. The implementation has been evaluated both formally and
experimentally on top of the virtualized infrastructure. The experimental
setup made use of diverse swarms with respect to number of trackers, seeders
and leechers.

Chapter~\ref{chapter:multiparty} presents the design and initial
implementation of a multiparty protocol in the Linux kernel. Heavily based on
swift this implementation aims to provided a real Transport layer of a
multiparty protocol directly within the Linux networking stack. An
intermediate approach, using raw sockets, is employed to make use of the
favorable user space environment while still providing the same interface.

The streaming part of this work and description of trials and results in the
local LivingLab are presented in Chapter~\ref{chapter:multimedia-dist}. The
addition of streaming facilities in the popular libtorrent-rasterbar
implementation\footnote{\url{http://www.rasterbar.com/products/libtorrent/}}
is an important component, highlighting the steps required to provide
streaming. The major part of the chapter is dedicated to the deployment,
analysis and evaluation of the NextShare
technology\footnote{\url{http://www.p2p-next.org}}, an advanced Peer-to-Peer
implementation based on BitTorrent.

Results and analysis of Peer-to-Peer streaming deployment in the LivingLab
form an important contribution of this work. Both feedback from users and
collected information from clients are used to provide an insight on the
performance of video-on-demand streaming in a Peer-to-Peer network.

The last chapter (\textbf{Conclusion}) highlights the major contributions and
results of this work and also points out what should be undertaken in the near
future. It rounds up the scientific relevance of this work and sets the ground
for further actions on the explored directions.
