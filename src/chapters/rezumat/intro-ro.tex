% vim: set tw=78 tabstop=4 shiftwidth=4 aw ai:

\chapter{Introducere}
\label{chapter:intro}

\textit{``Make it run, then make it right, then make it fast.'' -- Kent Beck}

Printre multitudinea de invenții, descoperiri și contribuții la dezvoltarea
civizilizației umane, Internet-ul este una dintre cele mai impresionante. În
numai câteva decenii de existență, Internetul a reușit să furnizeze mijloacele
pentru unificarea întregii planete, asigurând accesul la cunoștințe,
informație și resurse digitale.

Evoluția explozivă a Internet-ului a furnizat utilizatorilor din toată lumea
conexiuni din ce în ce mai performante. Evoluția similară a tehnologiei a
asigurat procesoare mai rapide, discuri mai mari, mai multă memorie. În
momentul în care și utilizatorii de la marginea Internet-ului aveau acces la
resurse de bună calitate, au devenit dominante nevoile de a se conecta ușor și
de a partaja facil informații. Sistemele Peer-to-Peer au umplut astfel un vid
technologic prin intermediului tehnologiei Internet-ului aflate într-o
evoluție continuă.

Protocolul BitTorrent este, în opinia autorului acestei teze, un exemplu de
protocol care a fost proiectat să funcționeze, apoi îmbunătățit sa ruleze
bine; dar, ca orice alt protocol, are dezavantajele
sale~\cite{bittorrent-trade-offs}. Simplu, dar puternic, protocolul BitTorrent
strălucește când este folosit pentru distribuția de conținut de dimensiune
mare și a fost ulterior adoptat pentru facilități precum
streaming~\cite{bittorrent-streaming}.

Această teză este centrată în jurul protocolului BitTorrent. Scopul nostru,
precizat și în cadrul tezei, este preluarea protocolului, rularea de
măsurători, analizarea acestuia și îmbunătățirea protocolului sau a topologiei:
să fie mai rapid, mai performant, mai utilizabil, cu funcționalități extinse.
Nu ne propunem să atacăm fiecare ocazie de îmbunătățire, ci să semnalăm
aspecte relevante care să fie luate în considerare pentru a asigura
îmbunătățirea performanței.

În cadrul sistemelor Peer-to-Peer, actualizările și îmbunătățirile trebuie să
țină cont de regulile, diversitatea și particularitățile unor astfel de
sisteme. Experimente pregătite cu grijă, măsurători atente, concluzii
raționale și propuneri utile sunt componente importante ale activităților de
cercetare în acest domeniu.

Obiectivul acestei teze este să furnizeze o serie de îmbunătățiri pentru
sistemele Peer-to-Peer la nivel de protocol, fie prin actualizarea și
îmbunătățirea protocoalelor existente, fie prin proiectarea și implementarea
unora noi. Evaluările formale și experimentale sunt folosite pentru a furniza
argumente legate de avantajele unor astfel de actualizări. Furnizarea
îmbunătățirilor se bazează pe lansarea de experimente atente și analiza
protocoalelor Peer-to-Peer. Folosirea mediilor de test și utilizarea
măsurătorilor și technicilor de evaluare sunt critice pentru a putea furniza
cu succes actualizări și îmbunătățiri ale protocoalelor.

\section{Obiectivul tezei}
\label{sec:intro:objective}

În contextul sistemelor Peer-to-Peer, actualizările, îmbunătățilre și
extensiile trebuie să țină cont de regulile, diversitatea și
particularităților acestor sisteme. Experimente create cu grijă, măsurători
raționale, concluzii clare, propuneri utile sunt componente importante ale
activităților de cercetare în domeniu.

Obiectivul acestei teze este furnizarea unei serii de îmbunătățiri ale
sistemelor Peer-to-Peer la nivel de protocol, fie prin actualizarea și
îmbunătățirea protocoalelor existente sau proiectarea și implementarea unora
noi. Evaluările formale și experimentale sunt folosite pentru a furnizare
argumente legate de avajtale acestor actualizări. Furnizarea de îmbunătățiri
se bazează pe generarea de scenarii de test și analiza protocoalelor
Peer-to-Peer. Folosirea mediilor de text și a tehnicilor de măsurare și
evaluare sunt critice pentru furnizarea cu succes a actualizărilor și
îmbunătățirilor de protocoale. Ca obiectiv secundar, teza indică metodele și
mecanismele pentru crearea de medii de test realiste, scalabile și automate și
abordările disponibile pentru colectarea, măsurarea și interpretarea
parametrilor de protocoale Peer-to-Peer.

Furnizarea unei evaluări corecte a datelor din teste este necesară pentru
relevanța acestora. O parte importantă a acestei teze este dedicată evaluării
arhitecturilor, protocoalelor, implementărilor actuale, mesajelor interne,
topologiilor de rețea cu ajutorul infrastructurilor versatile. Monitorizarea,
prelucrarea jurnalelor și prelucrarea internă sunt toate disponibile pentru a
permite o vedere în adâncime și o analiză a protocoalelor Peer-to-Peer.

Întrucât o parte importantă a acestei lucrări a fost desfășurată în contextul
proiectului P2P-Next\footnote{\url{http://www.p2p-next.org}}, un accent
deosebit a fost acordat folosirii tehnologiei Peer-to-Peer pentru a asigura
distribuția de conținut (\textit{streaming}). Având BitTorrent în centrul
activității de cercetare, am analizat mai multe adaptări și extensii.
Strategia \textit{cea mai rară piesă prima} aplicată de BitTorrent nu este
potrivită pentru streaming; trebuie avute în vedere adaptări ale strategiei.
Tehnologia NextShare, folosită în proiectul P2P-Next se află în centrul
activităților de măsurare și analiză.

Activitățile de măsurare, evaluare și analiză sunt dirijate spre furnizarea de
arhitecturi și protocoale de rețea care să ofere performanță superioară în
sistemele Peer-to-Peer. O rețea de tip overlay de trackere, asigurând
unificarea swarm-urilor, este propunerea pentru a obține swarm-uri mai
sănătoase (mai mulți seederi) și pentru unificarea conținutului. Furnizarea
unui protocol de comunicare inter-trackere, marcarea măsurătorilor relevante
și evaluarea sunt parte din obiectivele furnizării de extensii pentru
soluțiile existente. La fel este cazul proiectării unui protocol multiparty ca
alternativă la BitTorrent, proiectat de la zero ca un protocol de nivel
Transport în stiva OSI.

\section{Contextul tezei}
\label{sec:intro:scope}

În concordanță cu obiectivul stipulat de a furniza îmbunătățiri și actualizări
de protocoale, teza curentă definește domeniul său de lucru în
cadrul sistemelor Peer-to-Peer. În particular, cea mai mare parte a
activităților de cercetare și dezvoltare din cadrul acestei teze se desfășoară
în jurul protocolului BitTorrent, unul dintre cele mai utilizate protocoale
din Internet.

Motivată de dorința de a furniza actualizări utile aplicațiilor existente și
de a crea noi aplicații, abordarea folosită este una care lansează și
monitorizează, colectează parametri de protocol, îi analizează și furnizează
informații de intrare pentru îmbunătățiri, extensii și actualizări.

Cu mai multe abordări posibile pentru analiza clienților, protocoalelor,
rețelei și a swarm-ului, abordarea aleasă se bazează pe informații de nivel
scăzut, considerate baza parametrilor protocolului. Nu suntem (direct)
interesați de viziunea completă asupra unui sistem Peer-to-Peer; mai degrabă
de o privire în adâncime în modul de funcționare a clientului. Punctele de
atașare în implementarea clienților Peer-to-Peer, jurnalizarea clienților și
analiza traficului de rețea furnizează mijloacele de colectare a informațiilor
de protocol Peer-to-Peer de nivel scăzut.

Mesajele de protocol conțin informație relevantă precum viteza de download a
clienților, viteza de upload, numărul de peeri conectați, evenimente de
protocol. Acești parametri sunt analizați și sunt trase concluzii care permit
ca îmbunătățirile de protocol să fie proiectate și apoi utilizate. O abordare
importantă este folosirea parametrilor colectați pentru comparații între
diverse implementări, protocoale sau comportamente în situații diverse.

\section{Cuprins}
\label{sec:intro:contents}

Cuprinsul tezei urmărește atât o ordine cronologică, luând în calcul efortul
depus cât și una logică -- capitolele finale se bazează pe informația
prezentată de capitolele anterioare. Referințele înainte și înapoi sunt
furnizate acolo unde este necesar, având continuu în vedere furnizarea de
informație inițială în capitolul anterior.

Capitolul~\ref{chapter:p2p-systems} -- \textbf{Sisteme Peer-to-Peer} prezintă
starea sistemelor Peer-to-Peer și a implementărilor curente. Prezintă câteva
date legate de apariția sistemelor Peer-to-Peer, evoluția acestora și
provocările curente. Accent deosebit se acordă protocolului BitTorrent și
distribuției de conținut, elemente care formează, în mod colectiv, contextul
acestei lucrări. Insistăm pe facilitățile de streaming ale sistemelor
Peer-to-Peer întrucât reprezintă aspecte relevante prezentate în această teză.

Baza celor mai multe experimente și scenarii este infrastructura de rețea
virtualizată și caracteristicile prezentate în
Capitolul~\ref{chapter:virt-infra}. Prezentăm beneficiile folosirii
tehnologiilor de virtualizare pentru crearea mediilor Peer-to-Peer și
accentuăm caracteristicile infrastructurii. Se prezintă informație detaliată
despre automatizarea lansării infrastructurii, a clienților Peer-to-Peer și
despre colectarea rezultatelor.

Am proiectat, dezvoltat și lansat o infrastructură de virtualizare ce
furnizează caracteristici precum scalabilitate, extensibilitate, automatizare.
Având la bază OpenVZ și suita de utilitare aferentă, infrastructura a fost
folosită pentru a crea scenarii experimentale pentru măsurători de
performanță, comparație între implementări, distribuție clasică față de
streaming. Propunem un set de metrici care furnizează informații de
performanță despre diverse soluții de virtualizare, incluzând OpenVZ, Xen și
LXC.

Capitolul~\ref{chapter:proto-measure} prezintă metodele și abordările de
colectarea a informației de la clienți Peer-to-Peer, parsarea acelei
informații în parametri de protocol și supunerea datelor spre analiză. Două
abordări folosite sunt implementarea unei biblioteci generice al cărei API
este ,,conectat'' în aplicații existente și folosirea de date jurnalizate. Un
framework dedicat de prelucrare este furnizat pentru a interpreta datele
clienților. Informația poate fi analizată fie în urma activității de colectare
de jurnale sau în același timp (în timp real). Abordarea secundară este
folosită pentru monitorizare.

Biblioteca generică a fost integrată în libtorrent-rakshasa și Transmission
(clienți BitTorrent), în vreme ce framework-ul de procesare a fost integrat
într-o diversitate de clienți, datorită naturii sale mai puțin intruzive.
Componentele de stocare, parsare și analiză permit măsurători și metrici
pentru swarm-uri BitTorrent. Propunem un framework de analiză formală bazat pe
parametri ce țin de viteza clientului; este folosit pentru a furniza
informații legate de performanța peer-ilor și a swarm-ului.

O primă îmbunătățire în swarm-urile BitTorrent este proiectarea,
implementarea, testarea și evaluarea unui nou protocol care permite unificarea
swarm-urilor, descris în Capitolul~\ref{chapter:unified-tracker}. Protocolul,
denumit TSUP
(\textit{Tracker Swarm Unification Protocol}) permite swarm-urilor care sunt
parte a aceluiași fișier \texttt{.torrent}, dar care sunt diferite, să fie
unificate, astfel încât peerii vor putea să comunice unul cu altul chiar dacă
se găsesc în swarm-uri diferite. Intermedierea este realizată de trackere,
astfel încât protocolul formează o rețea de tip overlay de trackere.

TSUP a fost proiectat de la zero și a fost implementat în \textit{XBT
Tracker}. Implementarea a fost evaluată atât formal cât și experimental peste
infrastructura virtualizată. Scenariile de test au însemnat diverse swarm-uri
în care s-a variat numărul de trackere, seederi și leecheri.

Capitolul~\ref{chapter:multiparty} prezintă proiectarea și prima fază a
implementării unui protocol multiparty în nucleul Linux. Bazată pe swift,
implementarea urmărește să furnizeze un protocol multiparty de nivel transport
în stiva de rețea a nucleului Linux. O abordare intermediară, care folosește
sockeți raw, folosește mediul utilizator, mai favorabil pentru testare,
păstrând însă aceeași interfață.

Componenta de streaming a acestei lucrări și descrierea scenariilor și
rezultatelor din LivingLab-ul local sunt prezentate în
Capitolul~\ref{chapter:multimedia-dist}. Adăugarea facilităților de streaming
în cadrul implementării libtorrent-rasterbar este o componentă importantă,
indicât pașii necesari pentru a furniza streaming. Componenta majoră a acestui
capitol este dedicată testării, analizei și evaluării tehnologiei NextShare, o
implementare Peer-to-Peer avansată bazată pe BitTorrent.

Rezultatele și analiza streaming-ului Peer-to-Peer în scenarii LivingLab sunt
o contribuție importantă a acestei lucrări. Atât feedback-ul utilizatorilor
cât și informația colectată de la clienți sunt folosite pentru a furniza o
vedere detaliată legată de performanța streaming-ului în rețele Peer-to-Peer.

Ultimul capitol (Capitolul~\ref{chapter:conclusion} --\textbf{Concluzie})
precizează contribuțiile importante și rezulatele acestei tezei; punctează de
asemenea acțiunile care trebuie avute în vedere în viitorul apropiat.
Coagulează relevanța științifică a acestei lucrări și pune bazele pentru
acțiuni ulterioare în direcțiile explorate.
