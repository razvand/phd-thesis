% vim: set tw=78 tabstop=4 shiftwidth=4 aw ai:

\chapter{Introducere}
\label{chapter:intro}

\textit{``Make it run, then make it right, then make it fast.'' -- Kent Beck}

Printre multitudinea de invenții, descoperiri și contribuții la dezvoltarea
civizilizației umane, Internet-ul este una dintre cele mai impresionante. În
numai câteva decenii de existență, Internetul a reușit să furnizeze mijloacele
pentru unificarea întregii planete, asigurând accesul la cunoștințe,
informație și resurse digitale.

Evoluția explozivă a Internet-ului a furnizat utilizator din toată lumea
conexiuni din ce în ce mai performante. Evoluția similară a tehnologiei a
asigurat procesoare mai rapide, discuri mai mari, mai multă memorie. În
momentul în care și utilizatorii de la marginea Internet-ului aveau acces la
resurse de bună calitate, au devenit dominante nevoile de a se conecta ușor și
de a partaja facil informații. Sistemele Peer-to-Peer au umplut astfel un vid
technologic prin intermediului tehnologiei Internet-ului aflate într-o
evoluție continuă.

Protocolul BitTorrent este, în opinia autorului acestei teze, un exemplu de
protocol care a fost proiectat să ruleze, apoi îmbunătățit sa ruleze bine,
deși, ca orice alt protocol, are dezavantajele
sale~\cite{bittorrent-trade-offs}. Simplu, dar puternic, protocolul BitTorrent
strălucește când este folosit pentru distribuția de conținut de dimensiune
mare și a fost ulterior adoptat pentru facilități precum
streaming~\cite{bittorrent-streaming}.

Această teză este centrată în jurul protocolului BitTorrent. Scopul nostru,
precizat și în cadrul tezei, este preluarea protocolului, rularea de
măsurător, analizarea acestuia și îmbunătățirea protocolului sau a topologiei:
să fie mai rapid, mai performant, mai utilizabil, cu funcționalități extinse.
Nu ne propunem să umplem fiecare ocazie de îmbunătățire, ci să semnalăm
aspecte relevante care să fie luate în considerare pentru a asigura
îmbunătățirea performanței.

În cadrul sistemelor Peer-to-Peer, actualizările și îmbunătățirile trebuie să
țină cont de regulile, diversitatea și particularitățile unor astfel de
sisteme. Experimente pregătite cu grijă, măsurători atente, concluzii
raționale și propuneri utile sunt componente importante ale activităților de
cercetare în acest domeniu.

Obiectivul acestei teze este să furnizeze o serie de îmbunătățiri pentru
sistemele Peer-to-Peer la nivel de protocol, fie prin actualizarea și
îmbunătățirea protocoalelor existente sau proiectarea și implementarea unora
noi. Evaluările formale și experimentale sunt folosite pentru a furniza
argumente legate de avantajele unor astfel de actualizări. Furnizarea
îmbunătățirilor se bazează pe lansarea de experimente atente și analiza
protocoalelor Peer-to-Peer. Folosirea mediilor de test și utilizarea
măsurătorilor și technicilor de evaluare sunt critice pentru a putea furniza
cu succes actualizări și îmbunătățiri de protocoalelor.

\section{Domeniul tezei}
\label{sec:intro:scope}

În concordanță cu obiectivul stipulat de a furniza îmbunătățiri și actualizări
în sistemele Peer-to-Peer, teza curentă definește domeniul său de lucru în
cadrul sistemelor Peer-to-Peer. În Particular, cea mai mare parte a
activităților de cercetare și dezvoltare din cadrul acestei teze se desfășoară
în jurul protocolului BitTorrent, unul dintre cele mai utilizate protocoale
din Internet.

Motivată de dorința de a furniza actualizări noi și utile aplicațiilor
existente și de a crea noi aplicații, abordarea folosită este una care
lansează și monitorizează, colectează parametri de procol, îi analizează și
furnizează informații de intrare pentru îmbunătățiri, extensii și actualizări.

Cu mai multe abordări posibile pentru analiza clienților, protocoalelor,
rețelei și a swarm-ului, abordarea aleasă se bazează pe informații de nivel
scăzută, considerate baza parametrilor protocolului. Nu sunt (direct)
interesați de viziunea completă asupra unui sistem Peer-to-Peer; mai degrabă
cu o privire în adâncime la comportarea clientului. Punctele de angrenare în
implementarea clienților Peer-to-Peer, jurnalizarea clienților și analiza
traficului de rețea furnizează mijloacele de colectare a informațiilor de
protocol Peer-to-Peer de nivel scăzut.

Mesajele de protocol conțin informație prețioasă precum viteza de download a
clienților, viteza de upload, numărul de peeri conectați, evenimente de
protocol. Acești parametri sunt analizați și sunt trase concluzii care permit
ca îmbunătățirile de protocol să fie propuse și apoi utilizate. O abordare
importantă este folosirea parametrilor colectați pentru comparații între
diverse implementări, protocoale sau comportamente în situații diverse.

\section{Cuprins}
\label{sec:intro:contents}

Cuprinsul tezei urmărește atât o ordine cronologică, luând în calcul efortul
depus cât și una logică -- capitolele finale se bazează pe informația
prezentată de capitolele anterioare. Referințele înainte și înapoi sunt
furnizate acolo unde este necesar, având continuu în vedere furnizarea de
informație inițială în capitolul anterio.

Capitolul~\ref{chapter:p2p-systems} -- \textbf{Sisteme Peer-to-Peer} prezintă
starea curentă a sistemelor Peer-to-Peer și a implementărilor existente.

Baza experimentelor și a scenariilor de teste este infrastructura virtualizată
de testare și facilitățile acesteia, prezentate în
Capitolul~\ref{chapter:virt-infra}.

Capitolul~\ref{chapter:proto-measure} este preocupat cu metode și abordări de
obținere a informațiilor de la clienți Peer-to-Peer, parsarea acelei
informații în parametri de protocol și plasarea acestora în scenarii de
analiză.

O primă îmbunătățire a swarm-urilor BitTorrent este proiectarea,
implementarea, testarea și evaluarea unui nou protocol care permite unificarea
swarm-urilor, așa cum este descris în Capitolul~\ref{chapter:unified-tracker}.

Capitolul~\ref{chapter:multiparty} prezintă proiectarea și implementarea
inițială a unui protocol multiparty în nucleul Linux.

Componenta de streaming din cadrul activităților întreprinse, precum și
descrierea scenariilor de test și a rezultatelor din LivingLab-ul local sunt
prezentate în Capitolul~\ref{chapter:multimedia-dist}.

Ultimul capitolul (\textbf{Conlusion}) precizează contribuții majore și
rezultate ale acestor activități și punctează, de asemenea, ce trebuie să fie
avut în vedere în viitorul apropiat. Realizează o analiză a relevanței
științifice a activităților și stabilește contextul pentru acțiuni ulteriore
în cadrul direcțiilor explorate.
