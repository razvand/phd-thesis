% vim: set tw=78 tabstop=4 shiftwidth=4 aw ai:

\chapter{Conclusion}
\label{chapter:conclusion}

In the ever evolving context of Internet technologies in general, and
Peer-to-Peer technologies in particular, this work presented approaches on
measuring protocol parameters and providing improvements with particular focus
on performance. Considering the past decade evolution of Peer-to-Peer
technology, we have been using lessons learned in the past years to augment
existing protocols or design and implement new ones, while keeping the grip on
running realistic trials and collecting and interpreting protocol parameters.

Our measurements analysis has been centered at protocol level, with focus on
peer protocols. We didn't focus on providing an overall swarm view, but rather
a client-centric analysis. Information provided by clients had bee subject to
analysis -- protocol parameters such as download rate, upload rate, peer
connections, protocol events are at the center of the analysis. An
``individualistic'' client-centric approach was preferred against an overall
swarm-centric one.

Recognizing the immense impact of the BitTorrent protocol, most of the
measurements and improvements presented have made use of BitTorrent
architecture and specifics. BitTorrent protocol message analysis, protocol
parameters, streaming updates, tracker overlays, automating client actions
have formed the primary set of activities used in this work.

\section{Contributions}
\label{sec:conclusion:contributions}

The work described in this thesis provided significant contributions in
Internet technology and Peer-to-Peer technology. The research and engineering
activities described provide updates to measurement and evaluation methods and
existing protocols used in Peer-to-Peer systems. With focus on client-centric
behavior and performance evaluation, the contributions highlighted below
amount to valuable progress in analysing and improving Peer-to-Peer protocols
and applications.

Contributions of this thesis range from infrastructure building and automation
to protocol designs and implementations and streaming experiments. Main
contributions are:
\begin{itemize}
  \item development of a fully automated and scalable infrastructure for
  Peer-to-Peer experiments;
  \item proposal of metrics for evaluation of virtualiation solutions;
  \item design, development and incorporation of a BitTorrent logging library;
  \item design and development of an automated framework for extracting,
  processing and analysing protocol data for BitTorrent implementations;
  \item proposal of a formal context for evaluation of BitTorrent clients
  based on measured protocol parameters;
  \item design and implementation of a novel protocol for swarm unification in
  BitTorrent environment;
  \item design and implementation of a multiparty protocol in the Linux
  kernel;
  \item evaluation of Peer-to-Peer streaming technologies through user-centric
  experiments.
\end{itemize}

The creation of a \textbf{fully automated and scalable infrastructure} stands
at the basis of extensive trials and experiments involving Peer-to-Peer
systems.

The use of \textbf{virtualization technology in the context of Peer-to-Peer
systems}

We have proven the \textbf{scalable use of OpenVZ solutions} for providing a
realistic infrastructure.

We undertook a set of approaches to \textbf{simulating connection dropouts in
BitTorrent environments}.

In the process of providing realistic and relevant trials with existing
Peer-to-Peer clients, \textbf{updates and patches have been applied to
existing BitTorrent} clients.

We isolated and defined \textbf{protocol messages types and protocol
parameters} specific to Peer-to-Peer systems.

A \textbf{generic logging library} has been implemented to provide an API that
may be hooked in existing clients.

\textbf{Client-centric logging updates and configuration} have formed the
basis of the most heavily used approach towards collecting protocol
information from running clients.

\textbf{A protocol parameter parsing and analysis engine} has been developed
to provide an in depth look at information provided by client log files.

During our investigation of Peer-to-Peer streaming technologies, we provided
\textbf{streaming support to the popular libtorrent implementation}.

Within the context of the P2P-Next project, the deployment and maintenance of
the local LivingLab was essential towards \textbf{providing live user centric
experiments regarding Peer-to-Peer technology}.

A significant part of the LivingLab activities was and is dedicated to
\textbf{analysis and monitoring of BitTorrent streaming}.

A novel overlay network protocol on top of BitTorrent, aiming at integrating
peers in different swarms, has been presented. Dubbed TSUP (Tracker Swarm
Unification Protocol), the protocol is used for \textbf{creating and
maintaining a tracker network enabling peers in swarms to converge in a single
swarm}.

The TSUP protocol has been subject to \textbf{experimentation within the XBT
tracker}.

The \textbf{integration the kernel space as a multiparty transport
protocol} that is solely responsible for getting the bits moving improves the
over all protocol performance.

In order to provide a rapid development environment, we've created a
\textbf{user-space implementation and test suite for the multiparty protocol}.

\section{Future Work}
\label{sec:conclusion:future}

As is the case with improvements, updates and extensions, no work could ever
be complete. Continuous protocol evolution, the ever expanding Internet, new
user request request further updating. At the same time, current approaches
must be pushed towards a larger crowd, high quality formal evaluations and
increased reliability.

The multiparty protocol design and implementation in the Linux kernel has to
be completed, tested, evaluated. Current goal is to push for a mainline
integration, albeit in connection with the standardisation effort taking
place. The implementation presented, together with swift are part of a draft
that is to be submitted to IETF for standardisation, as a PPSP (Peer-to-Peer
Streaming Protocol). An efficient implementation, together with a standard
back up, would simplify the work of providing the current implementation as an
important contribution to the Linux kernel.

Formal evaluation of Peer-to-Peer parameters, as highlighted in the final part
of Chapter~\ref{chapter:proto-measure} should be improved. Currently the
defined model provides little insight on what are the bottleneck and
problematic points where the experimenter should insist and focus his/her
actions. Enhancing the evaluation implies providing the means through which
the experimenter may detect anomalies or performance loss and also understand
the factors influencing them, factor which should be tuned.

On par with the formal evaluation of Peer-to-Peer parameters is the
Peer-to-Peer streaming versus classical distribution activity mentioned in
Chapter~\ref{chapter:multimedia-dist}. The focus of the LivingLab is
providing sound comparison between the two updates and, based on that, provide
advice regarding improvement of streaming extensions. Preliminary trials have
only taken into account status message parameters from seeders in swarms;
complete trials should take into account all parameters available and do
analysis on top of that.

The exciting and ever-evolving context of Peer-to-Peer systems offers a
plethora of possibilities for further work on measuring and improving
protocols. While primarily aiming to enhance current efforts, new research
directions and challenges may be paved.

\section{Publications and Talks}
\label{sec:conclusion:publications}

\subsection{Papers}

\begin{itemize}
  \item Mircea Bardac, George Milescu, and Răzvan Deaconescu. Monitoring a
  BitTorrent Tracker for Peer-to-Peer System Analysis. In \textit{Intelligent
  Distributed Computing}, pages 203--208, 2009 (\textbf{ISI indexed})
  \item Călin-Andrei Burloiu, Răzvan Deaconescu, and Nicolae Țăpuș. Design and
  Implementation of a BitTorrent Tracker Overlay for Swarm Unification. In
  \textit{International Conference on Network Services}, 2011 (\textbf{ISI
  indexed})
  \item Răzvan Deaconescu, George Milescu, Bogdan Aurelian, Răzvan Rughiniș,
  and Nicolae Țăpuș. A Virtualized Infrastructure for Automated BitTorrent
  Performance Testing and Evaluation. \textit{International Journal on
  Advances in Systems and Measurements}, 2(2\&3):236--247, 2009
  \item Răzvan Deaconescu, George Milescu, and Nicolae Țăpuș. Simulating
  Connection Dropouts in BitTorrent Environments. In \textit{EUROCON --
  International Conference on Computer as a Tool}, 2011, IEEE, pages 1-4, 2011
  (\textbf{ISI indexed})
  \item Răzvan Deaconescu, Răzvan Rughiniș, and Nicolae Țăpuș. A BitTorrent
  Performance Evaluation Framework. \textit{Proceedings of Fifth International
  Conference of Networking and Services}, 2009, \textbf{Best Paper Award}
  (\textbf{ISI Indexed})
  \item Răzvan Deaconescu, Răzvan Rughiniș, and Nicolae Țăpuș. A Virtualized
  Testing Environment for BitTorrent Applications. \textit{Proceedings of
  CSCS'17}, 2009
  \item Răzvan Deaconescu, Marius Sandu-Popa, Adriana Drăghici, and Nicolae
  Țăpuș. Using Enhanced Logging for BitTorrent Swarm Analysis. In
  \textit{Proceedings of the 9th RoEduNet IEEE International Conference},
  Sibiu, 2010 (\textbf{ISI Indexed})
  \item Răzvan Deaconescu, Marius Sandu-Popa, Adriana Drăghici, and Nicolae
  Țăpuș. BitTorrent Swarm Analysis through Automation and Enhanced Logging.
  \textit{International Journal of Computer Networks \& Communications},
  3(1):53--65, 2011
  \item Andreea Leța, Răzvan Deaconescu and Răzvan Rughiniș. Extending Packet
  Altering Capacities in Simulated Large Networks. In \textit{Proceedings of
  the 17th International Conference on Control Systems and Computer Science
  (CSCS17)}, Bucharest, 2009
  \item Marius Sandu-Popa, Adriana Drăghici, Răzvan Deaconescu, and Nicolae
  Țăpuș. A Peer-to-Peer Swarm Creation and Management Framewor. In
  \textit{Proceedings of the 1st Workshop on Software Services: Frameworks and
  Platforms}, Timișoara, Romania, 2010 (\textbf{ISI indexed})
  \item George Milescu, Răzvan Deaconescu, and Nicolae Țăpuș. Versatile
  Configuration and Deployment of Realistic Peer-to-Peer Scenarios. In
  \textit{International Conference on Network Services}, 2011 (\textbf{ISI
  indexed})
  \item Răzvan Rughiniș and Deaconescu, Răzvan. Analysis of a QoS-based
  Traffic Engineering Solution in GMPLS Grid Networks. \textit{17th
  International Conference on Control Systems and Computer Science},
  Bucharest, 2009
  \item Răzvan Rughiniș and Răzvan Deaconescu. Optimization Strategies in MPLS
  Traffic Engineering. \textit{UPB Scientific Bulletin, Series C}, 1/ 2009,
  ISSN 1454-234x, pp. 91-102
  \item Răzvan Rughiniș and Răzvan Deaconescu. Methods of Adjusting MPLS
  Network Policies. \textit{UPB Scientific Bulletin, Series C}, 3/2009, ISSN
  1454-234x, pp. 121-132
  \item Mircea Bardac, Răzvan Deaconescu and Adina Magda Florea. Scaling
  Peer-to-Peer testing using Linux Containers. \textit{9th RoEduNet IEEE
  International Conference}, 2010, June 24-26, 2010, Sibiu, Romania, pag.
  287-292, ISSN: 2068-1038, ISBN 978-1-4244-7335-9 (\textbf{ISI indexed})
  \item Laura Gheorghe, Răzvan Rughiniș, Răzvan Deaconescu, and Nicolae Țăpuș.
  Authentication and Anti-replay Security Protocol for Wireless Sensor
  Networks. \textit{The Fifth International Conference on Systems and Networks
  Communications}, 2010 (\textbf{ISI indexed})
  \item Laura Gheorghe, Răzvan Rughiniș, Răzvan Deaconescu and Nicolae Țăpuș.
  Reliable Authentication and Anti-replay Security Protocol for Wireless
  Sensor Networks. \textit{The Second International Conferences on Advanced
  Service Computing}, 2010 (\textbf{ISI indexed})
  \item Laura Gheorghe, Răzvan Rughiniș, Răzvan Deaconescui and Nicolae Țăpuș.
  Adaptive Trust Management Protocol Based on Fault Detection for Wireless
  Sensor Networks. \textit{The Second International Conferences on Advanced
  Service Computing}, 2010 (\textbf{ISI indexed}):
\end{itemize}

\subsection{Books}

\begin{itemize}
  \item Răzvan Rughiniș, Răzvan Deaconescu, George Milescu and Mircea Bardac.
  Introducere în sisteme de operare. Editura Printech, 2009, Bucharest. ISBN:
  978-606-521-386-9
  \item Răzvan Rughiniș, Răzvan Deaconescu, Andrei Ciorba and Bogdan Doinea.
  Rețele locale. Editura Printech, 2008, Bucharest. ISBN: 978-606-521-092-9
\end{itemize}

\subsection{Workshops}

\begin{itemize}
  \item Performance of P2P Implementations. \textit{P2P'08 Workshop}. Aachen,
  September 2008
  \item Peer-to-Peer Systems. Evolution and Challenges. \textit{Ixia HiTech
  Presentations}. Bucharest, April 2011
\end{itemize}
