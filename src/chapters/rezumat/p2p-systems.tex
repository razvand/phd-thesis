% vim: set tw=78 tabstop=4 shiftwidth=4 aw ai:

\chapter{Peer-to-Peer Systems and Protocols}
\label{chapter:p2p-systems}

Peer-to-Peer (P2P) Systems have become a ``buzzword'' in the early 21st
century and are nowadays one of the most common technologies used in the
Internet. Although until 2000 the most common paradigm in Internet was
client-server, the emergence of Napster\footnote{\url{http://www.napster.com/}} has pushed Peer-to-Peer
systems to become one of the most used (and controversial) technologies.

\section{The Peer-to-Peer Paradigm}
\label{sec:p2p-systems:paragigm}

As mentioned previously, the characteristics of P2P networks are sharing and
distribution of resources, decentralization and autonomy.

Sharing and distribution of resources refers to the functionality provided by
each node of the Peer-to-Peer network. A node can function as a server or as a
client and may also act as a provider and consumer of resources and
services (information, files, bandwidth, storage, processor cycles). These
nodes are also called ``servants''.

Decentralization refers to the absence of a central coordinating authority for
organizing the network or for resource usage and distribution or communication
between nodes. Communication takes place directly between nodes. There is a
frequent distinction between pure P2P networks and hybrid networks. In
hybrid P2P networks, certain functions (such as indexing and authentication)
are allocated to a subset of nodes that have coordinating role.

Autonomy means that each node independently decides what resources it provides
to other nodes.

In scientific and IT communities, P2P applications are often classified in
categories such as: instant messaging, file sharing, grid computing and
collaboration.  This classification has been developed over time and fails to
make clear distinctions between different applications. Nowadays there are
many situations where several categories may be considered equivalent. For
these reasons, another classification could take into account issues regarding
resource usage in P2P systems.

\section{Peer-to-Peer Systems Deployed in the Internet}
\label{sec:p2p-systems:p2p-internet}

Since early 2000s, a variety of
Peer-to-Peer solutions have emerged. Though made popular due to the diversity
of P2P file sharing solutions, other applications such as collaborative
software (as in JXTA), video streaming and chat (such as Jabber) have made
their way in the Internet.

Each type of application possesses specific features and approaches towards
Peer-to-Peer communication. Napster, similar to DirectConnect, provides a
central indexing server, while actual communication (sharing) is enabled
between peers. Gnutella is a fully decentralized approach to file sharing.
Peer connections and interactions require no central server, albeit for super
nodes who are required to bootstrap the network. BitTorrent, one of the most
successful protocols, isolates a sharing session within a swarm, itself
described by a metadata file, dubbed a \textit{.torrent} file.

\subsection{Napster}

Napster\footnote{\url{http://www.napster.com/}} was the first successful example of a Peer-to-Peer
system. Napster was permanently closed in June 2001, an event that marked the
end of an outstanding era, with 65 million users in only 20 months. Napster's
central control model was possible due to focusing user approach.

In a common architecture, communication between Napster components is mediated
by the server. Clients are connecting to a well known \textbf{meta-server}.
The meta-server associates a server with reduced load from one of the
clusters. A cluster consists of 5 servers, each of which can control about
15.000 users.

The client registers to the chosen server, providing information about its
identity and shared files. Afterwards, it receives information about other
online users and available files. Users are anonymous to each other and local
directory structure is not directly interrogated. The main interest is
searching for content and determine where the requested resource may be
downloaded from.  The server directory is only used to translate between
station identity, the desired resource and the IP address necessary to
initiate the connection.

\subsection{Gnutella}

Initially, Gnutella\footnote{\url{http://rfc-gnutella.sourceforge.net/}} was the name of a prototype client developed in just a few
weeks in March 2000 by the same team that created WinAmp. At its beta version
launching, almost everyone saw Gnutella as a competitor for Napster, designed
to overcome many of its limitations. However, AOL, who had just bought
Nullsoft, decided to immediately stop working on the client. Everything would
have ended if hadn't been Bryan Mayland who managed to deduce Gnutella
protocol and published its specifications on the Internet. Thus began the
development of Gnutella open-source project.

Nowadays Gnutella is a generic term with various meanings: the protocol, the
open-source technology and the Internet network used. There are several
clients for the Gnutella protocol and, although most of them focus on sharing
and searching files, many other operations may be enabled.

Gnutella is mainly a file sharing network that allows arbitrary types of
files. There is no central server and therefore there is no point of failure.
Public or private networks are defined only by clients who are currently in
communication with each other. Each user can build a local map of the network
capturing messages from other clients. There may be multiple networks,
depending on how clients are configured to interconnect.

The lack of a central control point in Gnutella means that legal
responsibility for file transfers remains in the user's hands. Depending on
the point of view, this may turn out to be a good or a bad thing. On
\textit{gNET}, one may find illegal copies of almost everything one can think
of.

\subsection{FastTrack}

One of the latest and most innovative  applications in Peer-to-Peer
architectures is the FastTrack network. FastTrack arrived as a solution to
Napster and Gnutella problems. FastTrack architecture is hybrid by nature,
which, as noted above, is an intersection of two or more basic network
topologies. In FastTrack we are talking about the intersection between
centralized and decentralized topologies.

\subsection{BitTorrent}

One of the disadvantages of P2P systems is the ``opening'' for \textit{free
riding}~\cite{free-riding} or \textit{free loading}. While peers are generally
considered to be altruistic, this behavior is not generally enforced. Free
riding is equivalent to an egoistic behavior, where a peer gets information
from other peers and gives nothing back. This behavior disables the very
nature of P2P systems: sharing information.

If a given network consists of a high number of free riders, then data load
among peer is unbalanced. When a new node shares a popular file in the native
P2P system, it will attract a large number of connections. In case of free
riding, this node must provide bandwidth for additional clients.

BitTorrent\footnote{\url{http://www.bittorrent.com/}} is a new protocol being used to solve the free riding problem,
though not completely~\cite{free-riding}. The idea behind the protocol is
fairly simple: the node that plays the role of the server breaks the file in
pieces. If the file is requested by more clients simultaneously, each client
will get a different piece of that file. When a client gets a complete piece
it will allow other clients to download that subfile, while it will continue
downloading the second piece. In other words, the node acts simultaneously as
a client and a server after receiving the first piece. The process continues
until the download completes.

The BitTorrent protocol is highly suitable for downloading large files where
the download process takes a long time. As a direct consequence, more
``server'' peers are available for a longer duration. More clients will not
result in a performance decrease of the whole network as the load is
distributed.

Consider a situation where the ``server'' or seeder possesses 4 pieces. Each
client (leecher) requests one piece from the server. As noticed, client \#1
gets all pieces but from different nodes. The order in which pieces reach the
client may not respect the initial piece succession. The BitTorrent algorithm
will try maximizing the number of pieces available for download at a certain
point in time -- that is balancing the availability of pieces among peers in a
network.

In order to download a file in BitTorrent network, the user would follow the
step below:

\begin{itemize}
  \item install a BitTorrent application;
  \item surf the web;
  \item find a file on the server;
  \item select location where the file would be saved;
  \item wait for completion of the process;
  \item instruct application to close (the upload process, also called
  ``seeding'' continues until the user closes the application).
\end{itemize}

\section{Content Distribution in Peer-to-Peer Systems}
\label{sec:p2p-systems:streaming}

Peer-to-Peer solutions have traditionally been used for file sharing among
users in the Internet. Generally, file transfer occurs in small chunks (also
called pieces) that are transferred or exchanged from one client to the other.
Two different pieces may be transferred from two different clients, depending
on their availability. Piece transfer may not be (and generally isn't)
sequential; that is, a piece is transferred according to its availability,
peer approval and implementation of the Peer-to-Peer protocol.

As multimedia content forms a large part of data that is distributed over the
Internet, recent years have witnessed the rise of streaming solutions and
their introduction to Peer-to-Peer solutions. Both live streaming and
on-demand video streaming have been added to existing Peer-to-Peer solutions
and new applications have been designed. We introduce the main methods for
providing streaming solutions for P2P systems, such as tree, multi tree and
mesh based systems.

\subsection{Peer-to-Peer Streaming}
\label{subsec:p2p-systems:p2p-streaming-p2p}

In a P2P network the user may not only download a video asset, but he/she may
also upload it. Thus a user becomes an active participant in the streaming
process. Recent years have seen the emergence of video streaming applications
based on P2P solutions.

P2P streaming solutions~\cite{p2p-streaming-survey} create an overlay network
topology for delivering content (that is, a virtual network topology over a
physical one).  The overlay network typically follows one the two structures:
\textit{tree-based} and \textit{mesh-based}.

In a tree-based topology data is pushed from its \textit{root node} to
children nodes, then to other children nodes and so on. Its main disadvantage
is peer departure (peer churn). A peer departure will temporarily
disrupt video delivery to child peers of the departed node.

In a mesh-based topology, peers are able to communicate with other peers without
having to adhere to static topologies. Peer relations are established based on
data availability and network bandwidth. A peer periodically initiates
connections to other peers in the network, exchanges information regarding
data availability and pulls data from neighboring peers. This has the
advantage of robustness to churning. It does, however, suffer from video
playback quality degradation when no clear data distribution path exists.

\section{Issues and Challenges in Peer-to-Peer Systems}
\label{sec:p2p-systems:issues}

The plethora of Peer-to-Peer solutions in the Internet makes it difficult to
choose the best one for a given job. Each application or application type is
specialized in a given kind of activity; variations of each kind of
application is itself an issue when aiming to use the best one. Measurements,
metrics and evaluations have to be considered to allow a formal classification
of these solutions.

Streaming (both VoD and live) has been integrated into many Peer-to-Peer
solutions.  However, performance has been reported to still lag behind a
medium-load HTTP streaming server and far behind a classical \textit{Content
Delivery Network} (CDN). Better bandwidth utilization, increased peer
incentive, new business models and hybrid P2P/CDN infrastructures may provide
increased performance for Peer-to-Peer applications when dealing with
streaming requests.

In order to provide an evaluation of various Peer-to-Peer solutions out there,
one would either turn to experimentation and validation or to formal
analysis. While formal analysis is achievable through a ``pen and paper''
approach, experimentation requires a high number of resources and carefully
placed ``information probes'' to gather information. A common approach
involves ``hooking'' probes in existing Peer-to-Peer
swarms~\cite{corr-overlay} and gather
information in those points. Another one involves creating an
infrastructure~\cite{bt-vi}
that is able to realistically simulate a Peer-to-Peer network; this approach
would, however, require a large number of systems, close to the size of an
average swarm -- in order to ensure a realistic trial environment.

Even after 10 years of continuous development, Peer-to-Peer systems raise
interesting questions and pose significant challenges to Internet scientists
and engineers. Thorough analysis and evaluation coupled with cleverly
designed and efficient solutions are key to improving Peer-to-Peer
applications and their usage in the Internet.
