% vim: set tw=78 tabstop=4 shiftwidth=4 aw ai:

\chapter{Conclusion}
\label{chapter:conclusion}

În contexul evoluției continue a tehnlogiilor de Internet și a tehnologiilor
Peer-to-Peer, această lucrare prezintă abordările pentru măsurarea
parametrilor de protocol și furnizarea de îmbunătățiri cu accent particular pe
performanță. Luând în considerare evoluția tehnologiei Peer-to-Peer din
ultimul deceniu, am folosit lecțiile expuse în ultimii ani pentru a augmenta
protocoale existent sau pentru a proiecta și implementa protocoale noi, ținând
însă cont de furnizarea unui set de scenarii realiste și de colectarea și
intepretarea parametrilor de protocol.

Analiza măsurătorilor noastre a fost centrată la nivelul protocolului, cu
accent pe protocoalele specifice clienților. Nu am insistat pe furnizarea unei
vederi de ansamblu a swarm-ului, ci mai degraba pe o analiză centrată pe
client. Informația furnizată de clienți a fost supuză analizei -- parametri de
protocol precum viteza de download, viteza de upload, conexiunile între peeri,
evenimentele de tip protocol se găsesc în centrul analizei. O abordare
,,individualistă'', centrată pe client, a fost preferată în fața unei centrată
pe swarm.

Ținând cont de impactul potențial imens al protocolului BitTorrrent, cea mai
mare parte a măsurătorilor și îmbunătățirilor prezentate au tinut cont de
arhitectura și specificitățile BitTorrent. Analiza mesajelor de protocol
BitTorrent, parametrii de protocol, actualizările pentru streaming,
overlay-urile de tracker, automatizarea acțiunior clienților au reprezentat
principalul set de activități folosite în această lucrare.

\section{Contribuții}
\label{sec:conclusion:contributions}

Activitățile descrise în această teză furnizează contribuții semnificative în
tehnologia Internet și în tehnologia Peer-to-Peer. Activitățile de cercetare
și ingierie descrise furnizează actualizări ale metodelor de măsurare și
evluare a protocolelor existente în sisteme Peer-to-Peer. Cu accent pe
comportament specific clientului și pe evaluarea performanței, contribuțiile
descrise mai jos cumulează la un progres valoros în analiza și îmbunătățirea
protocoalelor și aplicațiilor Peer-to-Peer.

Contribuțiile din cadrul acestei teze variază de la construirea și
automatizarea proiectării și implementării de protocoale până la experimente
și măsurători legate de streaming. Cele mai importante contribuții sunt:

\begin{itemize}
  \item dezvoltarea unei infrastructuri scalabile și automatizate pentru
  experimente Peer-to-Peer;
  \item folosirea tehnologiilor de virtualizare în contextul sistemelor
  Peer-to-Peer;
  \item folosirea scalabilă a soluțiilor de OpenVZ pentru a furniza o
  infrastructură realistă;
  \item investigarea unui set de abordări care să simuleze deconectările
  (intenționate sau neintenționate) în medii BitTorrent;
  \item propunerea unui set de metrici de evaluare a soluțiilor de
  virtualizare;
  \item proiectarea, dezvoltarea și încorporarea unei biblioteci de
  jurnalizare BitTorrent;
  \item crearea de actualizări și patch-uri de cod sursă pentru clienți
  BitTorrent existenți;
  \item izolarea și definirea de tipuri de mesaje de protocol și parametri de
  protocol specifici sistemelor Peer-to-Peer;
  \item proiectarea și implementarea unui framework automat de extragere,
  prelucrare și analiză a informațiilor de protocol din cadrul implementărilor
  BitTorrent;
  \item dezvoltarea unui engine de parsare și analiză a parametrilor de
  protocol care să furnizeze o privire atentă la informațiile furnizate de
  fișiere de tip jurnale ale clienților;
  \item propunerea unui context formal de evaluare a clienților BitTorrent
  bazat pe parametri măsurați de tip protocol;
  \item proiectarea și implementarea unui nou protocol de unificare a
  swarm-ului în mediul BitTorrent;
  \item proiectarea și implementarea unui protocol multiparty în nucleul
  Linux;
  \item crearea unei implementări în spațiul utilizator și a unei suite de
  test pentru protocolul multiparty;
  \item furnizarea suportului de streaming în cadrul libtorrent-rasterbar
  (implementare de BitTorrent);
  \item furnizarea de experimente centrate pe utilizatori legate de tehnologia
  Peer-to-Peer;
  \item analiza și monitorizarea streaming-ului BitTorrent.
\end{itemize}

\section{Dezvoltări ulteriorare}
\label{sec:conclusion:future}

Așa cum se întâmplă cu îmbunătățirile, actualizări și extensiile, nici un
efort nu poate fi complet. Evoluția continuă a protocoalelor, expansiunea
Internet-ului, solicitări din partea utilizator sunt elemente care necesită
actualizări continue. În același timp, abordările curente trebuie să fie
promovate în cadrul unui public mai larg, către evaluări formale de calitate
înaltă și fiabilitate cât mai bună.

Proiectarea și implementarea protocolului multiparty la nivelul nucleului
Linux trebuie să fie definivitată, testată și evaluată. Scopul curent este
transmiterea codului curent pentru o integrare în ramura principală de
dezvoltare a nucleului Linux, ținând cont, în același timp, de efortul de
standardizare. Implementarea prezentată, împreună cu swift, reprezintă un
draft care trebuie să fie submis pentru standardizare către IETF, în forma
unei propuneri PPSP (\textit{Peer-to-Peer Streaming Protocol}).

Evaluarea formală a parametrilor Peer-to-Peer, așa cum este indicat în partea
finală a Capitolului~\ref{chapter:proto-measure} trebuie îmbunățățită. În
momentul de fața, modelul propus furnizează informații reduse referitoare la
punctele problematice pe care experimentatorul trebuie să insiste și pe care
să-și accentueze efortul.

În conexiune cu evaluarea formală a parametrilor Peer-to-Peer se regăsește
activitatea de comparare a streaming-ului Peer-to-Peer cu distribuția clasică
BitTorrent, descrisă în Captolul~\ref{chapter:multimedia-dist}. Accentul
LivingLab-ului este de a furniza comparații de sunete între două actualizări
și, pe baza acesteia, să furnizeze sfaturi referitoare la îmbunătățirea
extensiilor de streaming.

Contextul efervescent și evolutiv al sistemelor Peer-to-Peer oferă o mulțime
de posibilități de dezvoltări ulterioare de măsurare și îmbunătățire a
protocoalelor. Deși accentul principal este acela de a întări eforturile
curent, noi direcții de cercetare pot fi susținute.

\section{Publicații}
\label{sec:conclusion:publications}

\subsection{Articole}

\begin{itemize}
  \item Mircea Bardac, George Milescu, and Răzvan Deaconescu. Monitoring a
  BitTorrent Tracker for Peer-to-Peer System Analysis. In \textit{Intelligent
  Distributed Computing}, pages 203--208, 2009 (\textbf{ISI indexed})
  \item Călin-Andrei Burloiu, Răzvan Deaconescu, and Nicolae Țăpuș. Design and
  Implementation of a BitTorrent Tracker Overlay for Swarm Unification. In
  \textit{International Conference on Network Services}, 2011 (\textbf{ISI
  indexed})
  \item Răzvan Deaconescu, George Milescu, Bogdan Aurelian, Răzvan Rughiniș,
  and Nicolae Țăpuș. A Virtualized Infrastructure for Automated BitTorrent
  Performance Testing and Evaluation. \textit{International Journal on
  Advances in Systems and Measurements}, 2(2\&3):236--247, 2009
  \item Răzvan Deaconescu, George Milescu, and Nicolae Țăpuș. Simulating
  Connection Dropouts in BitTorrent Environments. In \textit{EUROCON --
  International Conference on Computer as a Tool}, 2011, IEEE, pages 1-4, 2011
  (\textbf{ISI indexed})
  \item Răzvan Deaconescu, Răzvan Rughiniș, and Nicolae Țăpuș. A BitTorrent
  Performance Evaluation Framework. \textit{Proceedings of Fifth International
  Conference of Networking and Services}, 2009, \textbf{Best Paper Award}
  (\textbf{ISI Indexed})
  \item Răzvan Deaconescu, Răzvan Rughiniș, and Nicolae Țăpuș. A Virtualized
  Testing Environment for BitTorrent Applications. \textit{Proceedings of
  CSCS'17}, 2009
  \item Răzvan Deaconescu, Marius Sandu-Popa, Adriana Drăghici, and Nicolae
  Țăpuș. Using Enhanced Logging for BitTorrent Swarm Analysis. In
  \textit{Proceedings of the 9th RoEduNet IEEE International Conference},
  Sibiu, 2010 (\textbf{ISI Indexed})
  \item Răzvan Deaconescu, Marius Sandu-Popa, Adriana Drăghici, and Nicolae
  Țăpuș. BitTorrent Swarm Analysis through Automation and Enhanced Logging.
  \textit{International Journal of Computer Networks \& Communications},
  3(1):53--65, 2011
  \item Andreea Leța, Răzvan Deaconescu and Răzvan Rughiniș. Extending Packet
  Altering Capacities in Simulated Large Networks. In \textit{Proceedings of
  the 17th International Conference on Control Systems and Computer Science
  (CSCS17)}, Bucharest, 2009
  \item Marius Sandu-Popa, Adriana Drăghici, Răzvan Deaconescu, and Nicolae
  Țăpuș. A Peer-to-Peer Swarm Creation and Management Framewor. In
  \textit{Proceedings of the 1st Workshop on Software Services: Frameworks and
  Platforms}, Timișoara, Romania, 2010 (\textbf{ISI indexed})
  \item George Milescu, Răzvan Deaconescu, and Nicolae Țăpuș. Versatile
  Configuration and Deployment of Realistic Peer-to-Peer Scenarios. In
  \textit{International Conference on Network Services}, 2011 (\textbf{ISI
  indexed})
  \item Răzvan Rughiniș and Deaconescu, Răzvan. Analysis of a QoS-based
  Traffic Engineering Solution in GMPLS Grid Networks. \textit{17th
  International Conference on Control Systems and Computer Science},
  Bucharest, 2009
  \item Răzvan Rughiniș and Răzvan Deaconescu. Optimization Strategies in MPLS
  Traffic Engineering. \textit{UPB Scientific Bulletin, Series C}, 1/ 2009,
  ISSN 1454-234x, pp. 91-102
  \item Răzvan Rughiniș and Răzvan Deaconescu. Methods of Adjusting MPLS
  Network Policies. \textit{UPB Scientific Bulletin, Series C}, 3/2009, ISSN
  1454-234x, pp. 121-132
  \item Mircea Bardac, Răzvan Deaconescu and Adina Magda Florea. Scaling
  Peer-to-Peer testing using Linux Containers. \textit{9th RoEduNet IEEE
  International Conference}, 2010, June 24-26, 2010, Sibiu, Romania, pag.
  287-292, ISSN: 2068-1038, ISBN 978-1-4244-7335-9 (\textbf{ISI indexed})
  \item Laura Gheorghe, Răzvan Rughiniș, Răzvan Deaconescu, and Nicolae Țăpuș.
  Authentication and Anti-replay Security Protocol for Wireless Sensor
  Networks. \textit{The Fifth International Conference on Systems and Networks
  Communications}, 2010 (\textbf{ISI indexed})
  \item Laura Gheorghe, Răzvan Rughiniș, Răzvan Deaconescu and Nicolae Țăpuș.
  Reliable Authentication and Anti-replay Security Protocol for Wireless
  Sensor Networks. \textit{The Second International Conferences on Advanced
  Service Computing}, 2010 (\textbf{ISI indexed})
  \item Laura Gheorghe, Răzvan Rughiniș, Răzvan Deaconescui and Nicolae Țăpuș.
  Adaptive Trust Management Protocol Based on Fault Detection for Wireless
  Sensor Networks. \textit{The Second International Conferences on Advanced
  Service Computing}, 2010 (\textbf{ISI indexed}):
\end{itemize}

\subsection{Cărți}

\begin{itemize}
  \item Răzvan Rughiniș, Răzvan Deaconescu, George Milescu and Mircea Bardac.
  Introducere în sisteme de operare. Editura Printech, 2009, Bucharest. ISBN:
  978-606-521-386-9
  \item Răzvan Rughiniș, Răzvan Deaconescu, Andrei Ciorba and Bogdan Doinea.
  Rețele locale. Editura Printech, 2008, Bucharest. ISBN: 978-606-521-092-9
\end{itemize}

\subsection{Prezentări și workshop-uri}

\begin{itemize}
  \item Performance of P2P Implementations. \textit{P2P'08 Workshop}. Aachen,
  September 2008
  \item Peer-to-Peer Systems. Evolution and Challenges. \textit{Ixia HiTech
  Presentations}. Bucharest, April 2011
\end{itemize}
