% vim: set tw=78 tabstop=4 shiftwidth=4 aw ai:

\chapter{Concluzie}
\label{chapter:conclusion}

În contexul evoluției continue a tehnlogiilor de Internet și a tehnologiilor
Peer-to-Peer, această lucrare prezintă abordările pentru măsurarea
parametrilor de protocol și furnizarea de îmbunătățiri cu accent particular pe
performanță. Luând în considerare evoluția tehnologiei Peer-to-Peer din
ultimul deceniu, am folosit lecțiile expuse în ultimii ani pentru a augmenta
protocoale existente sau pentru a proiecta și implementa protocoale noi, ținând
cont de folosirea unui set de scenarii realiste și de colectarea și
intepretarea parametrilor de protocol.

Analiza măsurătorilor noastre a fost centrată în jurul nivelului protocol, cu
accent pe mesaje specifice clienților. Nu am insistat pe furnizarea unei
vederi de ansamblu a swarm-ului, ci mai degrabă pe o analiză centrată pe
client. Informația furnizată de clienți a fost supusă analizei -- parametrii de
protocol precum viteza de download, viteza de upload, conexiunile între peeri,
evenimentele de tip protocol se găsesc în centrul analizei. O abordare
,,individualistă'', centrată pe client, a fost preferată în fața uneia centrată
pe swarm.

Ținând cont de impactul potențial imens al protocolului BitTorrrent, cea mai
mare parte a măsurătorilor și îmbunătățirilor prezentate au ținut cont de
arhitectura și particularitățile BitTorrent. Analiza mesajelor de protocol
BitTorrent, parametrii de protocol, actualizările pentru streaming,
overlay-urile de tracker, automatizarea acțiunior clienților au reprezentat
principalul set de activități folosite în această lucrare.

Abordarea folosită în această lucrare este descrisă și de cronologia
capitolelor. Am urmat o secvență de forma lansează, rulează, analizează,
evaluează și îmbunătățește legată de clienți Peer-to-Peer și mesaje de tip
protocol. O infrastructură este creată și constituie structura pe care rulează
nodurile dintr-un swarm. Automatizarea a contribuit la configurarea facilă si
a rularea diverselor implementări. Informația este adunată și interpretată sub
formă de parametri de protocol, care sunt ulterior supuși evaluării. Ținând
cont de evaluare și de sfaturi, îmbunătățil sunt propuse și implementate.

Infrastructura a fost folosită ca baza pentru experimente și scenarii;
virtualizarea este folosită pentru motive de eficiență și continuare facilă.
Diverse swarm-uri și topologii au fost lansate deasupra infastructurii.
Abordarea centrată pe client folosită în această lucrare necesita date de la
peeri; aceste date au fost prelucrate în parametri de tip protocol care, la
rândul lor, au fost stocate și apoi folosite pentru analiză, evaluare și
comparație. S-a ținut cont și de alte abordări legate de colectare a
datelor și de procesare a acestora.

Ținând cont de accentul pe actualizările de streaming în sistemele
Peer-to-Peer, au fost folosite și măsurători, analiză și evaluare a acestora.
LivingLab-ul local, ca parte a proiectului P2P-Next, a fost centrul de test
pentru scenarii live, cuprinzând utilizatori și tehnologii de streaming.
Întrucât tehnologia NextShare furnizează un bun suport atât pentru streaming
VoD cât și live, a fost aleasă ca etalon al evaluării.

Îmbunătățirile în sisteme Peer-to-Peer au fost furnizate în lumina a două noi
protocoale. Primul este un protocol de tip overlay rulând deasupra unui swarm
BitTorrent existent. Celălalt este rezultatul proiectării unui protocol
multiparty la nivelul Transport în stiva de rețea a nucleului Linux --
depinzând de implementarea anterioară (\textit{swift}). Aceste actualizări
furnizează facilități supimentare și performanță îmbunătățită sistemelor
Peer-to-Peer.

Cea mai mare parte a acestei lucrări s-a desfășurat în contextul proiectului
P2P-Next. Autorul mulțumește și apreciează membrii echipei P2P-Next, care au
lucrat necontenit și entuziast pentru furnizarea noii generații de platforme
Peer-to-Peer de livrare de conținut.

\section{Contribuții}
\label{sec:conclusion:contributions}

Activitățile descrise în această teză furnizează contribuții în
tehnologia Internet și în tehnologia Peer-to-Peer. Activitățile de cercetare
și inginerie descrise furnizează actualizări ale metodelor de măsurare și
evaluare a protocoalelor existente în sisteme Peer-to-Peer. Cu accent pe
comportament specific clientului și pe evaluarea performanței, contribuțiile
descrise mai jos cumulează la un progres valoros în analiza și îmbunătățirea
protocoalelor și aplicațiilor Peer-to-Peer.

Contribuțiile din cadrul acestei teze variază de la construirea și
automatizarea de sisteme de test, la proiectarea și implementărea de
protocoale, și până la experimente și măsurători legate de streaming. Cele mai
importante contribuții sunt:

\begin{itemize}
  \item dezvoltarea unei infrastructuri scalabile și automatizate pentru
  experimente Peer-to-Peer;
  \item folosirea tehnologiilor de virtualizare în contextul sistemelor
  Peer-to-Peer;
  \item folosirea scalabilă a soluțiilor de tip OpenVZ pentru a furniza o
  infrastructură realistă;
  \item investigarea unui set de abordări care să simuleze deconectările
  (intenționate sau neintenționate) în medii BitTorrent;
  \item propunerea unui set de metrici de evaluare a soluțiilor de
  virtualizare;
  \item proiectarea, dezvoltarea și încorporarea unei biblioteci de
  jurnalizare BitTorrent;
  \item crearea de actualizări și patch-uri de cod sursă pentru clienți
  BitTorrent existenți;
  \item izolarea și definirea de tipuri de mesaje de protocol și parametri de
  protocol specifici sistemelor Peer-to-Peer;
  \item proiectarea și implementarea unui framework automat de extragere,
  prelucrare și analiză a informațiilor de protocol din cadrul implementărilor
  BitTorrent;
  \item dezvoltarea unui engine de parsare și analiză a parametrilor de
  protocol care să furnizeze o privire atentă la informațiile furnizate de
  fișiere de tip jurnale ale clienților;
  \item propunerea unui context formal de evaluare a clienților BitTorrent
  bazat pe parametri de tip protocol;
  \item proiectarea și implementarea unui nou protocol de unificare a
  swarm-urilor în mediul BitTorrent;
  \item proiectarea și implementarea unui protocol multiparty în nucleul
  Linux;
  \item crearea unei implementări în spațiul utilizator și a unei suite de
  test pentru protocolul multiparty;
  \item furnizarea suportului de streaming în cadrul libtorrent-rasterbar
  (implementare de BitTorrent);
  \item furnizarea de experimente centrate pe utilizatori legate de tehnologia
  Peer-to-Peer;
  \item analiza și monitorizarea streaming-ului BitTorrent.
\end{itemize}

%Crearea unei \textbf{infrastructuri scalabile și automate} stă la baza
%scenariilor și experimentelor care cuprind sisteme Peer-to-Peer.
%Infrastructura este capabilă să integreze o diversitate de clienți
%Peer-to-Peer și să ofere o vedere adâncă a parametrilor de comportament și
%performanță.
%
%Folosirea \textbf{tehnologiei de virtualizare în contextul sistemelor
%Peer-to-Peer} oferă o abordare originală pentru furnizarea de scenarii de test
%și experimentale în scurt timp. Virtualizarea oferă beneficiul creării unui
%swarm de dimensiune medie (câteva sute de noduri) deasupra unui numări mult
%mai mic (chiar 10) de sisteme hardware de uz general. În același timp
%furnizează o interfață ușor de folosit care asigură extensibilitate,
%configurabilitate și automatizare.
%
%Am demonstrat \textbf{folosirea scalabilă a soluției OpenVZ} pentru furnizarea
%unei infrastructuri realiste. Folosirea OpenVZ, o soluție de virtualizare cu
%impact reduse, a însemnat că am putut crea, în mod facil, un număr mare de
%isteme virtualizate și, deci, noduri Peer-to-Peer. Soluțiile de virtualizare
%la nivelul sistemului de operare sunt soluții excelente pentru situațiile în
%care scalabilitatea și realismul sunt importante.
%
%Avantajele și dezavantajele soluțiilor de virtualizare au fost explorate și
%formalizate într-un set de \textbf{metrici de virtualizare}. Aceste metrici
%furnizează o comparație între diferite soluții de virtualizare. Deși un
%formalism cu elemente complet cuantificabile este dificil de creat, furnizăm o
%vedere intimă a modului în care o soluție de virtualizare se compară cu o altă
%soluție de virtualizare și care sunt factorii majori de influență a
%performanței.
%
%Am lucrat la un set de abordări care să \textbf{simuleze căderile de conexiune
%în medii BitTorrent}. Am considerat câteva abordări pentru simularea căderilor
%de conexiune, precum oprirea/suspendarea procesului, terminarea procesului,
%filtrarea conexiunii prin firewall. Am tras câteva concluzii, prezentând
%avantajele și dezavantajele fiecăreia dintre abordările analizate.
%
%In procesul realizării de evenimente realiste și relevante pentru clienții
%Peer-to-Peer existenți, \textbf{actualizări și patch-uri au fost aplicate
%diverșilor clienți BitTorrent}. Cea mai mare parte a acestora au fost aplicate
%clienților cel mai des folosiți în scenariile noastre, spre exemplu
%Tribler/NextShare și hrktorrent/libtorrent-rasterbar. Actualizările au fost
%transmise dezvoltatorilor și integrate în ramurile principale de dezvoltare.
%Instrumentarea a ținut, de asemenea, cont de interfețe ascunse sau neevident
%ale clienților, cel mai adesea în procesul de colectare a informațiilor de
%jurnalizare.
%
%Am izolat și definit \textbf{parametri de protocol și tipuri de mesaje de
%protocol} specifice sistemelor Peer-to-Peer. Am definit două tipui de mesaje:
%mesaje de stare și mesaje verbose; aceste mesaje constau din informație
%valoroasă care este tradusă de un engine de jurnalizare în parametri de
%protocol precum rata de download, rata de upload, numărul de conexiuni de
%peer, evenimente de tip protocol și altele.
%
%A fost implementată o \textbf{bibliotecă de jurnalizare generică} ce
%furnizează un API care poate fi legat în cadrul clienților BitTorrent
%existenți. Biblioteca furnizează un API atât pentru informații de stare cât și
%pentru informații verbose. A fost integrată și testată în clienții rtorrent și
%Transmission. Oferă configurabilitatea API-ului de folosit și a tipurilor de
%mesaje de jurnalizare furnizate (fie text, fie format XML).
%
%\textbf{Configurația centrată pe client și actualizările de jurnalizare} au
%format baza celei mai folosite abordări de colectare a informațiilor de
%protocol pentru clienți. Facilitatea de jurnalizare, împreună cu
%infrastructura de virtualizare, instrumentează și/sau configurează clienții
%BitTorrent pentru a furniza informații de jurnalizare extinse. Datele sunt
%colectate și supuse analizei.
%
%A fost dezvoltat \textbf{un engine de parsare și analiză} care să furnizeze o
%privire adâncă a informației furnizată de fișierele de jurnalizare a
%clienților. Acest engine constă din câteva componente precum parsere, elemente
%de stocare și de rendering. Parserele pot fi \textbf{parsere de post-procesare
%sau parsere în timp real}, cu posibilitatea celor din urmă să furnizeze
%monitorizarea clienților și swarm-urilor. Engine-urile de stocare, în mod
%tipic baze de date, sunt ideale pentru a furniza o interfață ușor de accesat
%la parametrii de protocol. Engine-ul de rendering furnizează o interfață
%grafică a mesajelor de protocol.
%
%În cadrul activităților de măsurare a protocoalelor, am definit \textbf{un
%context formal de evaluare a performanței} protocoalelor Peer-to-Peer. Ține
%cont de parametrii extrași din mesajele de jurnalizare și construiește un
%model pentru analiză și interpretare. Acest formalism este centrat în jurul
%performanței, în principal viteza de transfer și, astfel, timpul de transfer.
%
%În timpul investigației technologiilor Peer-to-Peer de streaming, am furnizat
%\textbf{suport de streaming pentru implementarea libtorrent}. Câțiva algoritmi
%de streaming au fost folosiți; algoritmul de selecție a pieselor din
%libtorrent a trebuit să fie în mod semnificativ actualizat pentru a furniza o
%experiență de streaming realistă.
%
%În contextul proiectului P2P-Next, dezvoltarea și menținerea unui LivingLab
%local a fost esențială pentru \textbf{furnizarea de experimente live centrate
%pe utilizatori} legat de tehnologia Peer-to-Peer. LivingLab oferă acces la o
%diveritate de fișiere vide livrate prin intermediul technologiilor
%Peer-to-Peer, în particular NextShare, în formă unui plugin de navigator, ușor
%de folosit de utilizatori. Diverse experimente au avut loc și au implicat
%utilizatorii, permițând obținerea de feedback legat de tehnologia curentă și,
%astfel, posibilitatea de a furniza sugestii.
%
%O componentă semnificativă a activităților LivingLab a fost și este dedicată
%\textbf{analizei și monitorizării streaming-ului BitTorrent}. Am creat mediul
%pentru scenarii de testare care implică utilizatorii și tehnologii de
%streaming Peer-to-Peer. Acesta a furnizat rezultate preliminare în analiza
%streaming-ului BitTorrent și furnizarea unei comparații între distribuția
%clasică și cea de streaming.
%
%Am prezentat un nou protocol de overlay care rulează peste BitTorrent, țintit
%la integrarea peer-ilor din diverse swarm-uri. Denumit TSUP (\textit{Tracker
%Swarm Unification Protocol}), protocolul este folosit pentru \textbf{crearea
%și menținerea unei rețele de trackere care să permită peerilor dintr-un swarm
%să conveargă într-un singur swarm}. Fiecare swarm inițial este controlat de
%câte un tracker, trackerele folosesc protocolul de tip overlay pentru a
%comunica unul cu altul și, astfel, pentru a lua parte în swarm-ul mai mare.
%Implementarea a fost testată în cadrul implementării XBT Tracker.
%
%Protocolul TSUP a fost supus \textbf{experimentării în cadrul tracker-ului
%XBT}. Aceste experimente au folosit o diversitate de topologii de test pentru
%a testa protocolul: de la unele simple la unele complexe care conțineau mai
%multe trackere. S-a observat overhead-ul redus introdus și comportamentul de
%ansamblu al noului swarm.
%
%Am propus și proiectat o optimizare a protocolului curent \textit{swift}.
%\textbf{Integrarea în spațiul kernel în forma unui protocol multiparty} care
%este responsabil de transferul datelor și îmbunătățește performanța
%protocolului. Asigură o eficiență mai mare a transferului de date prin
%descreșterea numărului de schimbări dintre spațiul utilizator și kernel și
%eliminarea unora dintre penalităților de performanță. Pe moment în faza draft,
%există un efort de strandardizare pentru a impune swift și, astfel,
%implementarea multiparty la nivelul nucleului, ca un stadard IETF.
%
%Pentru a furniza un mediu un mediu rapid de dezvoltare, am creat o
%\textbf{implementare user-space și o suită de teste pentru protocolul
%multiparty}. Considerând mediul dificil de dezvoltare la nivelul nucleului,
%această implementare permite o dezvoltare, implementare și testare rapidă.
%După ce caracteristicile sunt testate în cadrul implementării în spațiul
%utilizator, vor fi ,,portate'' în cadrul implementării de kernel-space.

\section{Dezvoltări ulteriorare}
\label{sec:conclusion:future}

După cum este de așteptat când vine vorba de îmbunătățiri, actualizări și
extensii, nici un efort nu poate fi complet. Evoluția continuă a
protocoalelor, expansiunea Internet-ului, solicitări din partea utilizator
sunt elemente care necesită actualizări continue. În același timp, abordările
curente trebuie să fie promovate în cadrul unui public mai larg, către
evaluări formale de calitate înaltă și fiabilitate cât mai bună.

Proiectarea și implementarea protocolului multiparty la nivelul nucleului
Linux trebuie să fie definivitată, testată și evaluată. Scopul curent este
transmiterea codului curent pentru o integrare în ramura principală de
dezvoltare a nucleului Linux, ținând cont, în același timp, de efortul de
standardizare. Implementarea prezentată, împreună cu swift, reprezintă un
draft submis pentru standardizare către IETF, în forma unei propuneri PPSP
(\textit{Peer-to-Peer Streaming Protocol}).

Evaluarea formală a parametrilor Peer-to-Peer, așa cum este indicat în partea
finală a Capitolului~\ref{chapter:proto-measure} trebuie îmbunățățită. În
momentul de față, modelul propus furnizează informații reduse referitoare la
punctele problematice pe care experimentatorul trebuie să insiste și pe care
să-și accentueze efortul.

Legat de evaluarea formală a parametrilor Peer-to-Peer se regăsește
activitatea de comparare a streaming-ului Peer-to-Peer cu distribuția clasică
BitTorrent, descrisă în Captolul~\ref{chapter:multimedia-dist}. Accentul
LivingLab-ului este de a furniza comparații de performanță între două
actualizări și, pe baza acesteia, să furnizeze sfaturi referitoare la
îmbunătățirea extensiilor de streaming.

Contextul efervescent și evolutiv al sistemelor Peer-to-Peer oferă o mulțime
de posibilități de dezvoltări ulterioare de măsurare și îmbunătățire a
protocoalelor. Deși accentul principal este acela de a întări eforturile
curente, noi direcții de cercetare pot fi susținute, în jurul ideilor de
creștere a performanței la nivelul protocoalelor Peer-to-Peer.

\section{Publicații}
\label{sec:conclusion:publications}

\subsection{Articole}

\begin{itemize}
  \item Mircea Bardac, George Milescu, and Răzvan Deaconescu. Monitoring a
  BitTorrent Tracker for Peer-to-Peer System Analysis. In \textit{Intelligent
  Distributed Computing}, pages 203--208, 2009 (\textbf{ISI indexed})
  \item Călin-Andrei Burloiu, Răzvan Deaconescu, and Nicolae Țăpuș. Design and
  Implementation of a BitTorrent Tracker Overlay for Swarm Unification. In
  \textit{International Conference on Network Services}, 2011 (\textbf{ISI
  indexed})
  \item Răzvan Deaconescu, George Milescu, Bogdan Aurelian, Răzvan Rughiniș,
  and Nicolae Țăpuș. A Virtualized Infrastructure for Automated BitTorrent
  Performance Testing and Evaluation. \textit{International Journal on
  Advances in Systems and Measurements}, 2(2\&3):236--247, 2009
  \item Răzvan Deaconescu, George Milescu, and Nicolae Țăpuș. Simulating
  Connection Dropouts in BitTorrent Environments. In \textit{EUROCON --
  International Conference on Computer as a Tool}, 2011, IEEE, pages 1-4, 2011
  (\textbf{ISI indexed})
  \item Răzvan Deaconescu, Răzvan Rughiniș, and Nicolae Țăpuș. A BitTorrent
  Performance Evaluation Framework. \textit{Proceedings of Fifth International
  Conference of Networking and Services}, 2009, \textbf{Best Paper Award}
  (\textbf{ISI Indexed})
  \item Răzvan Deaconescu, Răzvan Rughiniș, and Nicolae Țăpuș. A Virtualized
  Testing Environment for BitTorrent Applications. \textit{Proceedings of
  CSCS'17}, 2009
  \item Răzvan Deaconescu, Marius Sandu-Popa, Adriana Drăghici, and Nicolae
  Țăpuș. Using Enhanced Logging for BitTorrent Swarm Analysis. In
  \textit{Proceedings of the 9th RoEduNet IEEE International Conference},
  Sibiu, 2010 (\textbf{ISI Indexed})
  \item Răzvan Deaconescu, Marius Sandu-Popa, Adriana Drăghici, and Nicolae
  Țăpuș. BitTorrent Swarm Analysis through Automation and Enhanced Logging.
  \textit{International Journal of Computer Networks \& Communications},
  3(1):53--65, 2011
  \item Andreea Leța, Răzvan Deaconescu and Răzvan Rughiniș. Extending Packet
  Altering Capacities in Simulated Large Networks. In \textit{Proceedings of
  the 17th International Conference on Control Systems and Computer Science
  (CSCS17)}, Bucharest, 2009
  \item Marius Sandu-Popa, Adriana Drăghici, Răzvan Deaconescu, and Nicolae
  Țăpuș. A Peer-to-Peer Swarm Creation and Management Framewor. In
  \textit{Proceedings of the 1st Workshop on Software Services: Frameworks and
  Platforms}, Timișoara, Romania, 2010 (\textbf{ISI indexed})
  \item George Milescu, Răzvan Deaconescu, and Nicolae Țăpuș. Versatile
  Configuration and Deployment of Realistic Peer-to-Peer Scenarios. In
  \textit{International Conference on Network Services}, 2011 (\textbf{ISI
  indexed})
  \item Răzvan Rughiniș and Deaconescu, Răzvan. Analysis of a QoS-based
  Traffic Engineering Solution in GMPLS Grid Networks. \textit{17th
  International Conference on Control Systems and Computer Science},
  Bucharest, 2009
  \item Răzvan Rughiniș and Răzvan Deaconescu. Optimization Strategies in MPLS
  Traffic Engineering. \textit{UPB Scientific Bulletin, Series C}, 1/ 2009,
  ISSN 1454-234x, pp. 91-102
  \item Răzvan Rughiniș and Răzvan Deaconescu. Methods of Adjusting MPLS
  Network Policies. \textit{UPB Scientific Bulletin, Series C}, 3/2009, ISSN
  1454-234x, pp. 121-132
  \item Mircea Bardac, Răzvan Deaconescu and Adina Magda Florea. Scaling
  Peer-to-Peer testing using Linux Containers. \textit{9th RoEduNet IEEE
  International Conference}, 2010, June 24-26, 2010, Sibiu, Romania, pag.
  287-292, ISSN: 2068-1038, ISBN 978-1-4244-7335-9 (\textbf{ISI indexed})
  \item Laura Gheorghe, Răzvan Rughiniș, Răzvan Deaconescu, and Nicolae Țăpuș.
  Authentication and Anti-replay Security Protocol for Wireless Sensor
  Networks. \textit{The Fifth International Conference on Systems and Networks
  Communications}, 2010 (\textbf{ISI indexed})
  \item Laura Gheorghe, Răzvan Rughiniș, Răzvan Deaconescu and Nicolae Țăpuș.
  Reliable Authentication and Anti-replay Security Protocol for Wireless
  Sensor Networks. \textit{The Second International Conferences on Advanced
  Service Computing}, 2010 (\textbf{ISI indexed})
  \item Laura Gheorghe, Răzvan Rughiniș, Răzvan Deaconescui and Nicolae Țăpuș.
  Adaptive Trust Management Protocol Based on Fault Detection for Wireless
  Sensor Networks. \textit{The Second International Conferences on Advanced
  Service Computing}, 2010 (\textbf{ISI indexed}):
\end{itemize}

\subsection{Cărți}

\begin{itemize}
  \item Răzvan Rughiniș, Răzvan Deaconescu, George Milescu and Mircea Bardac.
  Introducere în sisteme de operare. Editura Printech, 2009, Bucharest. ISBN:
  978-606-521-386-9
  \item Răzvan Rughiniș, Răzvan Deaconescu, Andrei Ciorba and Bogdan Doinea.
  Rețele locale. Editura Printech, 2008, Bucharest. ISBN: 978-606-521-092-9
\end{itemize}

\subsection{Prezentări și workshop-uri}

\begin{itemize}
  \item Performance of P2P Implementations. \textit{P2P'08 Workshop}. Aachen,
  September 2008
  \item Peer-to-Peer Systems. Evolution and Challenges. \textit{Ixia HiTech
  Presentations}. Bucharest, April 2011
\end{itemize}
