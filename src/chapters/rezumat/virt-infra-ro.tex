% vim: set tw=78 tabstop=4 shiftwidth=4 aw ai:

\chapter{Implementarea infrastructurilor rețelelor Peer-to-Peer cu ajutorul
virtualizării}
\label{chapter:virt-infra}

Considerând protocoalele și aplicațiile de rețea, două tipuri de medii
sunt în general folosite pentru testare, măsurare și analiză: configurații
de laborator și experimente aplicate în lumea reală. Configurațiile
de laborator sau experimentale sunt configurații adaptate evaluării
unor protocoale; în acest caz, experimentatorul deține controlul
complet asupra mediului. Acesta este de obicei cazul testării
funcționalității sau experimente în care experimentatorul necesită controlul
complet asupra procesului. Experimentatorul poate folosi o infrastructură
de laborator, una bazată pe sisteme cluster/cloud sau o infrastructură
oferită de către comunitate, cum ar fi
PlanetLab\footnote{\url{http://www.planet-lab.org/}}. Rias~\cite{rias} este
un exemplu de topologie overlay creată peste o infrastructură PlanetLab.

Experimentele în lumea reală oferă avantajul realismului dar și dezavantajul
lipsei de control. Informațiile furnizate de clienți sau alte entități sunt
limitate. În general, aceste experimente sunt folosite pentru a obține
informații relevante statistic, întrucât sunt angrenați participanți într-un
număr semnificativ mai mare decât în cazul unui test de laborator.

\section{Soluții de virtualizare}
\label{sec:virt-infra:openvz}

Un ,,buzzword'' important în ultimul deceniu, tehnologia de virtualizare
a evoluat de la bazele teoretice puse la sfârșitul anilor '70 către o
diversitate de implementări mature în ziua de astăzi. Odată cu creșterea
capacității hard-disk-urilor, a memoriei și a puterii de procesare
(în cea mai mare parte în număr de core-uri), soluțiile de virtualizare
au ajuns să furnizeze cel mai bun mod de alocare a resurselor. Noi
tehnologii precum virtualizarea hardware, îmbunătățirile I/O sau migrarea
în timp real au sporit cererea pentru soluții de virtualizare eficiente
care consolidează resursele hardware din prezent și viitor.

Soluțiile de virtualizare și-au găsit locul în utilizările de zi cu zi
datorită faptului că aduc beneficii importante, în special în ceea ce
privește costurile. Un sistem fizic dat poate în acest caz rula diverse
instanțe de sistem de operare, în timp ce în absența virtualizării ar fi
necesare mai multe sisteme.

Se pot identifica~\cite{best-damn-virt} trei beneficii importante ale
virtualizării:

\begin{itemize}
  \item consolidare;
  \item fiabilitate;
  \item securitate.
\end{itemize}

\textbf{Consolidarea}, un aspect important în special în cadrul mediului
de afaceri, permite ,,unificarea'' resurselor deasupra unui număr mic
de platforme fizice. Viteza sporită a procesoarelor și a subsistemului I/O
și capacitatea memoriei aflată în continuă creștere, toate acestea pot
fi acum folosite pentru a furniza suficiente resurse pentru multiple
sisteme de operare virtualizate.

Soluțiile de virtualizare oferă \textbf{fiabilitate} prin izolare. Un
eveniment de eșec pe o mașină virtuală nu va afecta o altă mașină virtuală.
,,Partiționarea'' întrebuințată de către soluțiile de virtualizare se
referă la faptul că fiecare mașină virtuală rulează pe un hardware simulat
dedicat și specializat.

Izolarea oferită de către tehnologiile de virtualizare este un aspect cheie
în ceea ce privește asigurarea securității mașinilor virtuale. În cazul
în care o mașină virtuală eșuează sau este compromisă, acest fapt nu va
afecta alte sisteme și, la nevoie, poate fi dezactivată rapid. Acest
rezultat ar fi foarte dificil de obținut pe o infrastructură fizică.

Datorită faptului că virtualizarea la nivelul sistemului de operare s-a
dovedit a fi cea mai bună bază pentru construirea unei infrastructuri
Peer-to-Peer, a trebuit să alegem între soluții existente cum ar fi
OpenVZ, LXC sau V-Server. Datorită experienței cu Linux, soluția a trebuit
să fie una bazată pe acesta, astfel că a trebuit să alegem între OpenVZ,
LXC și V-Server. Deși LXC are în prezent avantajul de a fi activ în versiunea
principală a nucleului, în momentul în care am început construirea
infrastructurii acesta era încă într-un stadiu incipient de dezvoltare.
În plus, resursele pentru documentare sunt reduse. Între OpenVZ și V-Server
am ales OpenVZ datorită documentației foarte bune din punct de vedere
calitativ și în urma recomandărilor și a experienței altor utilizatori.

Într-un mod similar cu alte soluții de virtualizare la nivelul sistemului
de operare, OpenVZ este o serie de patch-uri pentru nucleul Linux,
acesta îmbunătățind izolarea resurselor și a proceselor între așa-zisele
containere. Distribuțiile Linux importante vin de obicei cu o versiune
de nucleu care poate rula OpenVZ și care este ușor de instalat. O
serie de utilitare din spațiul utilizator permit instalarea și configurarea
mașinilor virtuale.

Fiind o soluție la nivelul sistemului de operare, în care se pot realiza
ușor configurările rețelei și având o interfață flexibilă de configurare
a limitării de resurse, OpenVZ a fost considerată a fi cea mai potrivită
alegere pentru implementarea unei infrastructuri de rețea virtualizată.
Capitolele următoare vor oferi detalii legate de procesul și utilitarele
folosite pentru scenariile referitoare la crearea infrastructurii și
respectiv crearea de scenarii de test cu swarm-uri Peer-to-Peer.

\section{Configurarea infrastructurii Peer-to-Peer virtualizate}
\label{sec:virt-infra:setup}

Crearea unui mediu virtualizat necesită noduri hardware unde vor fi
plasate mașinile virtuale, infrastructura de rețea, un set de șabloane
OpenVZ pentru instalare și un framework care permite comandarea clienților
aflați în mașini virtuale. Fiecare mașină virtuală rulează o aplicație
BitTorrent care a fost instrumentată pentru a folosi o interfață în linie
de comandă ușor de automatizat.

În ciuda limitărilor sale, OpenVZ este cea mai bună alegere pentru crearea
unui mediu virtualizat folosit la evaluarea clienților BitTorrent. 
Penalizările minime de performanță impuse de acesta și consumul mic de 
resurse asigură rularea a zeci de mașini virtuale pe același nod fizic.

În urma experimentelor am ajuns la concluzia că un mediu de testare
virtualizat bazat pe OpenVZ ar oferi capabilități de testare similare cu
acelea ale unui cluster non-virtualizat, la cel mult 10\% din cost.
Configurația noastră experimentală formată din doar 10 calculatoare este
capabilă de a rula cel puțin 100 de medii virtuale cu pierderi minime
de performanță.

\begin{figure}
  \begin{center}
    \includegraphics[width=0.7\textwidth]{src/img/virt-infra/virt-infra-overview}
  \end{center}
  \caption{Infrastructura de testare}
  \label{fig:virt-infra:infrastructure-overview}
\end{figure}

Figura~\ref{fig:virt-infra:infrastructure-overview} oferă o privire
de ansamblu asupra infrastructurii de testare BitTorrent.

Infrastructura este proiectată pentru a rula pe sisteme hardware de
consum. Fiecare sistem fizic folosește o implementare de mașină virtuală
OpenVZ pentru a rula sisteme virtuale multiple peste același nod hardware.
Fiecare mașină virtuală conține utilitarele de bază pentru rularea și compilarea
clienților BitTorrent. Implementările BitTorrent au fost instrumentate pentru
comenzi automatizate și de asemenea pentru a oferi către ieșire informații
de stare și jurnalizare necesare pentru analiza ulterioară și interpretarea
rezultatelor. Datorită faptului că infrastructura se dorește a fi
independentă de diversele implementări, adăugarea unui nou client BitTorrent
se referă la adăugarea script-urilor și instrumentației necesare.

\section{Evaluarea virtualizării}

În contextul unei multitudini de soluții de virtualizare, o infrastructură
pentru implementarea scenariilor Peer-to-Peer trebuie să ia în considerare
cât de adecvată este fiecare soluție. În urma unor studii empirice am ales
OpenVZ drept soluție potrivită cu nevoile noastre, dar căreia îi lipsește 
o metodă formală de evaluare.

Având în vedere utilizarea soluțiilor de virtualizare pentru scopul nostru,
luăm în considerare trei dimensiuni importante:

\begin{itemize}
  \item \textbf{eficiența} (scalabilitatea) -- cât de multe mașini
  virtuale/containere pot fi implementate pe o gazdă virtuală, permițând
  simularea adecvată a unui mediu;
  \item \textbf{izolarea} -- cât de bine sunt separate resursele mașinilor
  virtuale;
  \item \textbf{fiabilitate} -- cât de multe defecte software au loc pentru
  o soluție dată; acestea pot fi o consecință a implementării sau a
  suprautilizării/abuzului unei resurse.
\end{itemize}

Wood~et~al.~\cite{virt-prof-model} au investigat VMware ESX și Xen și au
creat un model I/O și un profil.

O abordare similară cu a noastră a fost aleasă de 
Soltesz~et~al.~\cite{virt-doppel}. Aceștia au considerat Xen și Linux
Vserver ca fiind reprezentative pentru para-, respectiv pan-virtualizare.
Metricile acestora includ performanța, izolarea și scalabilitatea, cu
semnificații similare cu cele enumerate mai sus.

\textbf{Eficiența} este o măsură legată de implementarea unui număr mare
de mașini virtuale și containere deasupra unui sistem fizic dat.
Padala~et~al.~\cite{eval-virt-performance} au arătat că OpenVZ duce la
o penalizare de performanță mai mică în comparație cu Xen. Cu toate acestea,
nu au fost folosite metode formale iar studiul nu ia în considerare alte
soluții de virtualizare la nivelul SO.

O măsurătoare formală a eficienței/performanței virtualizării trebuie să
ia în calcul trei aspecte:

\begin{itemize}
  \item resursele hardware;
  \item implementarea software-ului (în cazul nostru, implementarea
  clientului BitTorrent);
  \item soluția de virtualizare.
\end{itemize}

Astfel, formalizăm eficiența drept o funcție a celor trei metrici de mai sus:

\begin{align}
Eff & = f(HW, SW, VS)\\
Eff & = f(RAM, HDD, CPU, NET, OS, PS, BT, VS, NVM)\\
Eff &= \frac{VMB}{HNB}
\end{align}

unde:

\begin{multicols}{2}
    \begin{itemize}
      \item \texttt{Eff}: eficiența/performanța
      \item \texttt{HW}: resursele hardware
      \item \texttt{SW}: implementarea software-ului
      \item \texttt{VS}: soluția de virtualizare
      \item \texttt{RAM}: capacitatea memoriei RAM a sistemului
      \item \texttt{HDD}: capacitatea dispozitivului I/O
      \item \texttt{CPU}: viteza procesorului
      \item \texttt{NET}: aspecte legate de rețea
      \item \texttt{OS}: implementarea sistemului de operare
      \item \texttt{PS}: procese de bază ale container-ului
      \item \texttt{BT}: implementarea BitTorrent
      \item \texttt{NVM}: numărul de mașini virtuale
      \item \texttt{VMB}: comportamentul mașinii virtuale
      \item \texttt{HNB}: comportamentul mașinii fizice
    \end{itemize}
\end{multicols}

\textbf{Izolarea} reprezintă un mod de a determina cât de bine este o
mașină virtuală separată în raport cu o altă mașină virtuală și cu sistemul
de bază. Datorită faptului că VE-urile OpenVZ folosesc același nucleu,
este evident că măsura izolării este mai mică decât aceea a Xen sau KVM.
Izolarea servește drept o dimensiune importantă pentru măsurarea a cât
de adecvată este o soluție de virtualizare dată. Abilitatea soluției
în a izola complet o mașină virtuală și a-i specifica utilizarea
resurselor rezultă într-o abstractizare mai bună a mașinii fizice.

Deși formalizarea izolării ne este dificilă, considerăm că este realizabilă
o ,,scară a izolării virtualizării'', în cadrul căreia soluțiile de
virtualizare pot fi comparate, astfel că valorile metricii sunt mai degrabă
relative decât absolute. Astfel, considerăm că se poate afirma cu 
siguranță următorul fapt:

\begin{align}
Iso(normal processes) < Iso(chroot) < Iso(OpenVZ, LXC) < Iso(Xen,KVM)
\end{align}

\textbf{Fiabilitatea} trebuie luată în considerare în contextul unei
infrastructuri solicitate intens atunci când se face evaluarea unor soluții
de virtualizare multiple. În cadrul experienței noastre am întâmpinat
diverse probleme software, precum inconsistența la nivelul discului, erori
la nivelul sistemului de operare și timp crescut de răspuns cauzate de
utilizarea intensă/abuzul resurselor hardware. Infrastructura noastră
bazată pe OpenVZ trebuie evaluată în acest context. Din fericire, poate fi
luat în considerare un set măsuri bine definite și analizate în detaliu,
cum ar fi \textit{rata de eșec} sau \textit{distanța medie între eșecuri}.
Acestea trebuie să ia în calcul mediul de evaluare, la fel ca în cazul
măsurii pentru eficiență:

\begin{align}
Rel & = f(HW, SW, VS)\\
Rel & = f(RAM, HDD, CPU, OS (filesystem), PS, BT, VS, NVM)
\end{align}

E posibil să nu fie fezabilă (și nici posibilă) extragerea unei
formule care ia în calcul toate datele de intrare. Mai degrabă se consideră
o valoare numerică pentru fiecare metrică și se asigură o comparație între
aceste valori în așa fel încât afectarea unei variabile de intrare într-o
direcție dată (mărire sau micșorare) va avea un impact anume asupra
metricii. O abordare similară este oferită de
Ismail~et~al.~\cite{virt-metrics}.

Am experimentat cu medii OpenVZ, LXC și Xen pentru a crea un scenariu de test
experimental potrivit pentru metricile de mai sus.

În cazul OpenVZ, studiul nostru se concentrează pe eficiență și scalabilitate.
Containerele active care rulează clientul hrktorrent folosesc între 70MB și
170MB de RAM. Această oferă o estimare grosieră de circa $[20MB; 40MB]$ de
memorie consumată per VE. Sistemul de bază consumă cel mult 120MB de RAM.

Din perspectiva hard disk-ului, sistemul de bază poate fi configurat să
folosească 20GB de spațiu fără constrângeri majore pentru configurația
software. Fiecare container VE care poate rula clienți BitTorrent, folosește
1,7GB de spațiu. În același timp, 1GB de spațiu vor trebui păstrați pe fiecare
sistem pentru testare și jurnalizare, și 5GB pentru transfer și stocare. Acest
lucru înseamnă că circa 8GB de spațiu ar trebui să fie rezervați pe fiecare
container.

Valorile de mai sus sunt orientative. Un sistem configurat corespunzătoare
poate folosi mai puține resurse. Totuși, vrem să demonstrăm că, și în cazul
acestor valori înalte, un PC poate susține o cantitate semnificativă de
containere cu puțin overhead și penalizare de resurse.

Tabelul~\ref{table:virt-infra:openvz} prezintă o limită scăzută pentru numărul
de containere OpenVZ care pot rula pe un sistem de bază. Font cursiv înseamnă
limitare datorată capacității memoriei RAM, în vreme ce font aldin înseamnă
limitare datorată spațiului de disc.

\begin{table}[ht]
  \centering
  \begin{tabular}{|r|r|r|r|r|r|}
    \hline
     & \multicolumn{5}{|c|}{\textbf{Memory}} \\
    \hline
    \textbf{HDD} & \textbf{1GB} & \textbf{2GB} & \textbf{4GB} & \textbf{8GB} &
    \textbf{16GB} \\
    \hline
    80GB & \textbf{7} & \textbf{7} & \textbf{7} & \textbf{7} &
    \textbf{7} \\
    \hline
    120GB & \textbf{12} & \textbf{12} & \textbf{12} & \textbf{12} &
    \textbf{12} \\
    \hline
    200GB & \textbf{22} & \textbf{22} & \textbf{22} & \textbf{22} &
    \textbf{22} \\
    \hline
    300GB & \textit{22} & \textbf{35} & \textbf{35} & \textbf{35} &
    \textbf{35} \\
    \hline
    500GB & \textit{22} & \textit{47} & \textbf{60} & \textbf{60} &
    \textbf{60} \\
    \hline
    750GB & \textit{22} & \textit{47} & \textbf{91} & \textbf{91} &
    \textbf{91} \\
    \hline
    1TB & \textit{22} & \textit{47} & \textit{97} & \textbf{122} &
    \textbf{122} \\
    \hline
  \end{tabular}
  \caption{Scalability in OpenVZ (Number of Containers)}
  \label{table:virt-infra:openvz}
\end{table}

În cazul LXC, am rulat un scenariu care include folosirea comenxii Unix gzip,
folosită pentru comprimarea datelor. Aceasta a permis să măsurăt atât valori
legate de procesor cât și unele legate de hard-disk, pe aspecte de
scabilitate. Tabelele prezentate indică folosirea procesorului și procentajul
de operațiile de intrare/ieșire.

\begin{table}[ht]
  \centering
  \begin{tabular}{@{}rrrrr@{}}
    \toprule
    \textbf{No. containers} & \textbf{Average} & \textbf{Maximum} &
    \textbf{Minimum} & \textbf{Time} \\
    & \textbf{(\%)} & \textbf{(\%)} & \textbf{(\%)} &\textbf{(s)} \\
    \midrule
    2 & 19.2 & 21.3 & 18.7 & 2626 \\
    4 & 39.0 & 44.0 & 35.4 & 2786 \\
    8 & 82.3 & 95.6 & 70.6 & 2401 \\
    \bottomrule
  \end{tabular}
  \caption{LXC CPU Usage for gzip}
  \label{table:virt-infra:lxc-cpu}
\end{table}

\begin{table}[ht]
  \centering
  \begin{tabular}{@{}rrrrr@{}}
    \toprule
    \textbf{No. containers} & \textbf{Average} & \textbf{Maximum} &
    \textbf{Minimum} & \textbf{Time} \\
    & \textbf{(ops/s)} & \textbf{(ops/s)} & \textbf{(ops/s)} &\textbf{(s)} \\
    \midrule
    2 & 41.5 & 104.4 & 17.0 & 2626 \\
    4 & 39.0 & 174.2 & 29.2 & 2786 \\
    8 & 176.6 & 275.4 & 84.6 & 2401 \\
    \bottomrule
  \end{tabular}
  \caption{LXC I/O operatinons for gzip}
  \label{table:virt-infra:lxc-io}
\end{table}

\begin{table}[ht]
  \centering
  \begin{tabular}{@{}rrrrr@{}}
    \toprule
    \textbf{No. containers} & \textbf{Average} & \textbf{Maximum} &
    \textbf{Minimum} & \textbf{Time} \\
    & \textbf{(\%)} & \textbf{(\%)} & \textbf{(\%)} &\textbf{(s)} \\
    \midrule
    2 & 11.8 & 32.3 & 5.3 & 2626 \\
    4 & 26.1 & 65.9 & 12.4 & 2786 \\
    8 & 61.3 & 85.4 & 36.2 & 2401 \\
    \bottomrule
  \end{tabular}
  \caption{LXC Disk Busy for gzip}
  \label{table:virt-infra:lxc-disk}
\end{table}

Experimentarea cu LXC prezintă o creștere liniară a utilizării procesorului și
a discului. Spațiul de memorie este întotdeauna consumat la valoarea maximă
disponibilă astfel încât nu este avut în vedere. LXC poate scala liniar cu o
aplicație încărcată puternic și mai mult decât liniar dacă unele containere nu
sunt supraîncărcate în acel timp.

În cazul Xen am folosit patru maxini virtuale, fiecare folosind 256MB RAM și
care rulează pe un singur core pe sistemul de bază. Aceste mașini virtuale au
fost \texttt{ffmpeg}, o aplicație pentru conversii video. Rezultatele
utilizării procesului și discului sunt prezentate în
Tabelul~\ref{table:virt-infra:xen-metrics}.

\begin{table}[ht]
  \centering
  \begin{tabular}{@{}lrrr@{}}
    \toprule
    \textbf{Metric} & \textbf{1 VM} & \textbf{2 VMs} & \textbf{3 VMs} \\
    \midrule
    CPU Usage (\%) & 18.49 & 22.07 & 26.85 \\
    Disk writes (bytes/s) & 1008.11 & 1820.96 & 5074.22 \\
    Running Time (s) & 483 & 908 & 1808 \\
    \bottomrule
  \end{tabular}
  \caption{Xen Metrics for ffmpeg}
  \label{table:virt-infra:xen-metrics}
\end{table}

Evoluția diverselor metrici este puțin supra-liniara în cazul utilizării
spațiului de disc și aproape liniară în cazul timpului scurs. Xen scalează
similar cu LXC în cazul rulării de aplicații consumatoare de resurse (precum
\textit{ffmpeg}). Totuși, fiecare mașină virtuală folosește o valoare
predefinită pentru memorie care este, în general, mai mare decât a unei
soluții bazate pe virrtualizare la nivelul sistemului de operare -- lucru
datorat prezenței unui nou nucleu de sistem de operare și a hypervisor-ului.

\section{Automatizarea implementării și gestiunii clienților Peer-to-Peer}
\label{sec:virt-infra:auto-deploy}

Deasupra infrastructurii virtualizate, am dezvoltat un framework pentru
rularea, comandarea și gestiunea ,,swarm''-urilor BitTorrent. Scopul
este acela de a avea acces la un sistem ușor de folosit, pentru implementarea
unor scenarii de complexitate variabilă, efectuarea de măsurători ample și
colectarea și analiza informației oferite de swarm (precum mesaje specifice
protocolului, viteza de transfer sau numărul de parteneri conectați).

Framework-ul de gestiune a swarm-ului~\cite{swarm-management}
este o infrastructură bazată pe servicii care permite configurarea facilă
și comandarea clienților BitTorrent de pe o varietate de sisteme. O aplicație
client (\textit{commander}) este folosită pentru a trimite comenzi/cereri
către toate stațiile care rulează un anumit client de BitTorrent. Fiecare
stație rulează un \textit{serviciu dedicat} care interpretează cererile
și gestionează clientul local de BitTorrent în mod corespunzător.

Cu ajutorul automatizării și al instrumentării clienților, framework-ul
de gestiune permite colectarea rapidă a informației de stare și jurnalizare
de la clienții BitTorrent. Avantajele importante ale framework-ului sunt:

\begin{itemize}
  \item \textit{automatizarea} -- interacțiunea cu utilizatorul este
  necesară doar pentru pornirea clienților și pentru investigarea stării
  acestora;
  \item \textit{control complet} -- framework-ul de gestiune a swarm-ului
  permite utilizatorului/experimentatorului să specifice caracteristicile
  swarm-ului și ale clienților și să definească contextul/mediul unde
  este implementat scenariul;
  \item \textit{informație completă despre clienți} -- clienții instrumentați
  oferă informații detaliate în ceea ce privește implementarea internă a
  protocolului și evoluția transferului; informația este adunată de la
  toți clienții și folosită pentru analiză ulterioară.
\end{itemize}

\section{Simularea căderilor de conexiune}
\label{sec:virt-infra:dropouts}

Pentru a reproduce un comportament realist al unui swarm, experimentatorul
trebuie să ia în considerare felul în care conexiunile sunt create și
apoi distruse; cu alte cuvinte, trebuie luată în considerare dinamica
swarm-ului (perturbații sau ,,churning''). Definim simularea churn-ului
drept \textit{simularea căderilor de conexiune}~\cite{simulating-dropouts}.
Aceste tehnici trebuie să fie luate în considerare pentru a oferi actualizări
valoroase infrastructurii rețelei virtualizate.

Vom analiza două soluții pentru reproducerea realistă a căderilor din rețea
în medii simulate și le vom compara cu un eșec indus în rețea. Luând în
considerare infrastructurile unde există control complet asupra procesele
și mașinile clienților, conexiunile partenerilor la swarm sunt întrerupte
și reluate prin oprirea și repornirea clientului, suspendarea și reluarea
acestuia și dezactivarea și reactivarea interfeței de rețea. Comportamentul
clientului în primele două cazuri este comparat cu al treilea caz din
punctul de vedere al swarm-ului.

Din punctul de vedere al swarm-ului, o cădere în conexiunea clientului
este echivalentă cu părăsirea bruscă a sistemului de către client.
În acest caz nici conexiunile BitTorrent și nici legăturile TCP nu sunt
încheiate într-un mod ,,grațios'' și partenerii swarm-ului vor trece
prin timeout-uri multiple înainte de a declara conexiunile încheiate.

Acest comportament poate fi reprodus folosind trei soluții: oprirea
și repornirea clienților, suspendarea lor și dezactivarea interfeței
de rețea.

Soluțiile pentru separarea clientului de swarm sunt diferite din
punctul de vedere al gestiunii conexiunii TCP: clienții opriți au toate
conexiunile oprite, în timp ce clienții suspendați îți pot reporni
conexiunile în cazul în care nu s-a ajuns la un timeout în momentul
repornirii.

Căderea conexiunii clientului poate fi indusă prin dezactivarea interfeței
de rețea. Folosind această abordare, clientul va continua să ruleze fără
nici o altă intervenție din partea sistemului de operare, în timp ce
conexiunile sale TCP vor fi închise. Dezactivarea interfeței de rețea
reproduce îndeaproape căderile de rețea care au loc în Internet.

Infrastructura virtualizată și framework-ul scriptat au fost folosite
drept platformă de testare pentru scenariile de cădere ale rețelei.
Suita de evaluare folosește containere virtualizate pentru a crea
perechi ,,leecher-seeder''.

Fiecare pereche este folosită pentru a transfera un fișier de 100MB
de la un ,,seeder'' inițial către un ,,leecher''. Un tracker BitTorrent
este de asemenea pornit în același container cu seeder-ul pentru a media
comunicarea. Leecher-ul folosește o lățime de bandă limitată la 100KB/s.
Atât informația de la leecher cât și cea de la seeder sunt colectate
sub forma unor fișiere de log și parsate ulterior.

Scopul principal al experimentelor efectuate este acela de măsurare și
comparație a timpului de recuperare după fiecare cădere de conexiune
pentru fiecare din cele trei soluții propuse. Prin folosirea diverselor
intervale de timeout, punem în evidență asemănări și diferențe între
clienții suspendați, cei terminați și cazul în care sunt dezactivate
interfețele de rețea pentru a simula o cadere de conexiune.

% ifdown, mean and rsd
\def \meanifa {0}
\def \rsdifa {0}
\def \meanifb {12.10}
\def \rsdifb {10.88}
\def \meanifc {23.84}
\def \rsdifc {16.73}
\def \meanifd {47.16}
\def \rsdifd {4.50}
\def \meanife {79.66}
\def \rsdife {13.4}
\def \meaniff {146.37}
\def \rsdiff {8.05}
\def \meanifg {288.58}
\def \rsdifg {2.15}
\def \meanifh {535.42}
\def \rsdifh {0.09}

% suspend, mean and rsd
\def \meansuspenda {0}
\def \rsdsuspenda {0}
\def \meansuspendb {9.80}
\def \rsdsuspendb {4.30}
\def \meansuspendc {17.12}
\def \rsdsuspendc {2.33}
\def \meansuspendd {32.33}
\def \rsdsuspendd {8.87}
\def \meansuspende {72.16}
\def \rsdsuspende {26.83}
\def \meansuspendf {142.11}
\def \rsdsuspendf {11.56}
\def \meansuspendg {259.90}
\def \rsdsuspendg {1.56}
\def \meansuspendh {532.42}
\def \rsdsuspendh {2.11}

% stop, mean and rsd
\def \meanstopa {0}
\def \rsdstopa {0}
\def \meanstopb {12.21}
\def \rsdstopb {11.61}
\def \meanstopc {19.95}
\def \rsdstopc {9.04}
\def \meanstopd {36.36}
\def \rsdstopd {4.11}
\def \meanstope {67.61}
\def \rsdstope {2.45}
\def \meanstopf {131.80}
\def \rsdstopf {1.14}
\def \meanstopg {260.26}
\def \rsdstopg {0.69}
\def \meanstoph {515.81}
\def \rsdstoph {0.33}


%\begin{sidewaystable}
\begin{table}
  \centering
  \caption{Recovery Timeout for Different Scenarios}
  \label{tab:virt-infra:recovery}
  \begin{tabular}{@{}rrrrrrr@{}}
    \toprule
      & \multicolumn{2}{c}{\textbf{ifdown}} & \multicolumn{2}{c}{\textbf{suspend}} & \multicolumn{2}{c}{\textbf{stop}} \\
    \cmidrule{2-3} \cmidrule{4-5} \cmidrule{6-7}
      \textit{pause(s)} & \textit{mean(s)} & \textit{rsd(\%)} & \textit{mean(s)} & \textit{rsd(\%)} & \textit{mean(s)} & \textit{rsd(\%)} \\
    \midrule
%      4 & \meanifa & \rsdifa & \meansuspenda & \rsdsuspenda & \meanstopa &
%      \rsdstopa \\
      8 & \meanifb & \rsdifb & \meansuspendb & \rsdsuspendb & \meanstopb &
      \rsdstopb \\
      16 & \meanifc & \rsdifc & \meansuspendc & \rsdsuspendc & \meanstopc &
      \rsdstopc \\
      32 & \meanifd & \rsdifd & \meansuspendd & \rsdsuspendd & \meanstopd &
      \rsdstopd \\
      64 & \meanife & \rsdife & \meansuspende & \rsdsuspende & \meanstope &
      \rsdstope \\
      128 & \meaniff & \rsdiff & \meansuspendf & \rsdsuspendf & \meanstopf &
      \rsdstopf \\
      256 & \meanifg & \rsdifg & \meansuspendg & \rsdsuspendg & \meanstopg &
      \rsdstopg \\
      512 & \meanifh & \rsdifh & \meansuspendh & \rsdsuspendh & \meanstoph &
      \rsdstoph \\
    \bottomrule
  \end{tabular}
\end{table}
%\end{sidewaystable}


Tabelul~\ref{tab:virt-infra:recovery} face un sumar al rezultatelor
experimentelor efectuate. Cele trei metode principale folosite pentru
simularea căderilor de conexiune pot fi identificate prin \textbf{ifdown},
\textbf{suspend} și respectiv \textbf{stop}. Pentru fiecare metodă au
fost calculate valoarea medie și abaterea pătratică medie a timpului de 
recuperare. Coloana marcată cu \textit{pause} are semnificația timeout-ului
unui proces dat. Coloana \textit{mean} reprezintă media valorilor măsurate
și se măsoară în secunde, în timp ce coloana \textit{rsd} este abaterea
pătratică medie, exprimată în procente.

Remarcăm, drept concluzie, că prima soluție (\textbf{ifdown}) dezactivează
interfața de rețea a peer-ului pentru a simula sfârșitul conexiunii.
Rezultatele au arătat că deși timpul de recuperare are același ordin de
mărime ca în celelalte cazuri, acesta este mai mare. În concluzie, această
metodă este cea mai potrivită pentru a simula căderile de conexiune datorate
eșecului rețelei.

A doua soluție (\textbf{suspend}) suspendă clienții în timpul căderii
simulate (folosind SIGSTOP) și îi repornește ulterior (folosind SIGCONT).
Deși suspendarea clienților nu este o acțiune uzuală, rezultatele au arătat
că această metodă este similară cu metoda \textbf{stop}, raportat la timpul
de recuperare. Într-un mediu unde suspendarea peer-ilor este mai ușor de
obținut decât oprirea lor, o astfel de soluție se poate dovedi a fi
potrivită și poate oferi rezultate realiste.

A treia soluție (\textbf{stop}) constă în oprirea (folosind SIGKILL) și
repornirea clienților. Această soluție, deși agresivă, este
considerată a fi cea mai bună aproximare a unui comportament realist al
unei căderi de conexiuni, deoarece peer-ii au în mod normal o dinamică
ridicată a intrării și ieșirii dintr-un swarm.

\section{Configurația implementată și scenariile experimentale}
\label{sec:virt-infra:setup-scenarios}

Infrastructura completă, folosind automatizare, virtualizare și mecanismul
de simulare a căderilor, permite implementarea automată și gestiunea
unei varietăți mari de scenarii Peer-to-Peer. Fiecare gazdă virtualizată
OpenVZ rulează un singur client BitTorrent (sau tracker) și colectează
informații relevante pentru o analiză ulterioară. Stația \textit{commander}
definește configurația noului scenariu și apoi îl rulează deasupra
infrastructurii virtualizate. Limitările de lățime de bandă, tipurile
de client, porturile folosite, rata de churn și fișierele torrent sunt
configurate pentru scenariul dat.

Configurația experimentală a folosit în acest caz doar șase noduri fizice
din cadrul infrastructurii. Cea mai mare parte a experimentelor au constat
în sesiuni simultane de descărcare. Fiecare sistem a rulat câte un client
anume, toate VE-urile rulând simultan și în aceleași condiții. Rezultatele
și datele de jurnalizare au fost colectate după ce fiecare client și-a
încheiat download-ul.

\begin{table}[ht]
  \centering
  \begin{tabular}{@{}lrrrr@{}}
    \toprule
    & \textbf{Test1} & \textbf{Test2} & \textbf{Test3} &
    \textbf{Test4} \\
    \midrule
    mărimea fișierului & 908MB & 4.1GB & 1.09GB & 1.09GB	\\
    seederi & 2900 & 761 & 521 & 496	\\
    leecheri & 2700 & 117 & 49 & 51	\\
    \midrule
    \textbf{Client} & \multicolumn{4}{c}{Timp de descărcare (secunde)} \\
    \midrule
    aria2c & 4620 & 3233 & 580 & 623 \\
    azureus & 1961 & 2313 & N/A & 420 \\
    bittorrent & 17580 & 3639 & 1560 & 840 \\
    libtorrent & \textbf{581} & \textbf{913} & \textbf{150} & \textbf{134} \\
    transmission & 2446 & 3180 & 420 & 300 \\
    tribler & 2040 & 1260 & N/A & N/A \\
    \bottomrule
  \end{tabular}
  \caption{Rezultatele swarm-urilor test}
  \label{table:virt-infra:testsw}
\end{table}

Tabelul~\ref{table:virt-infra:testsw} prezintă o comparație a clienților
BitTorrent testați în patru scenarii diferite. Fiecare scenariu se referă
la un swarm diferit. Deși a fost colectată o cantitate mare de date, doar
timpul total de descărcare este prezent în tabel. Se poate observa
superioritatea implementării libtorrent, dovedită superioară în fața altor
implementări.
