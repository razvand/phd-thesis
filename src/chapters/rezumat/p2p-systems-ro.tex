% vim: set tw=78 tabstop=4 shiftwidth=4 aw ai:

\chapter{Sisteme și protocoale peer-to-peer}
\label{chapter:p2p-systems}

Sistemele peer-to-peer (P2P) au devenit un ,,buzzword'' la începutul
secolului 21 și sunt astăzi una dintre cele mai comune tehnologii folosite
în Internet. Deși până în anul 2000 cea mai des întâlnită paradigmă
era cea client-server, apariția serviciului Napster
\footnote{\url{http://www.napster.com/}} a făcut ca sistemele peer-to-peer
să devină una dintre cele mai folosite (și controversate) tehnologii.

\section{Paradigma peer-to-peer}
\label{sec:p2p-systems:paragigm}

După cum am menționat, caracteristicile rețelelor peer-to-peer sunt 
partajarea și distribuția de resurse, descentralizarea și autonomia.

Partajarea și distribuția resurselor se referă la funcționalitatea
furnizată de fiecare nod al rețelei peer-to-peer. Un nod poate funcționa
ca un server sau ca un client și poate avea rol de producător sau
consumator de resurse și servicii (informații, fișiere, lățime de bandă,
spațiu de stocare, cicli de procesor). Aceste noduri sunt numite și
,,servanți''.

Descentralizarea se referă la absența unei autorități centrale de coordonare
care să se ocupe cu organizarea rețelei, cu distribuția resurselor sau
comunicația între noduri. Nodurile rețelei comunică direct unele cu altele.
Se face deseori distincție între rețele P2P pure și hibride. În rețelele P2P
hibride, unele funcții (cum ar fi indexarea sau autentificarea) sunt alocate
unui subset de noduri care au rol de coordonare.

Autonomia înseamnă că fiecare nod decide în mod independent ce resurse
va furniza celorlalte noduri.

În comunitățile științifice și în cele IT, aplicațiile P2P sunt deseori
clasificate în categorii: mesagerie instant, partajare de fișiere, procesare
și colaborare într-un grid.  Această clasificare a fost elaborată de-a lungul
timpului și nu reușește să facă o diferențiere clară între aplicații. Astăzi,
există multe situații în care un număr de categorii pot fi considerate
echivalente. Din acest motiv, o altă clasificare ar putea lua în considerare
probleme legate de consumul de resurse în sisteme P2P.

\section{Sisteme peer-to-peer implementate în Internet}
\label{sec:p2p-systems:p2p-internet}

Încă de la începutul anilor 2000 au apărut o varietate de soluții peer-to-peer.
Deși popularitatea lor se datorează diversității de soluții peer-to-peer pentru
partajare de fișiere, alte utilizări, cum ar fi software-ul colaborativ
(JXTA), streaming video și chat (Jabber) au apărut în Internet.

Fiecare tip de aplicație posedă trăsături și abordări specifice pentru
comunicația peer-to-peer. Napster, în mod similar cu DirectConnect,
pune la dispoziție un server central de indexare, în timp ce comunicația
propriu-zisă (partajare) are loc între nodurile rețelei. Gnutela folosește
o abordare complet descentralizată. Conexiunile și interacțiunea
între parteneri nu depind de un server central, în afară de ,,supernoduri''
care sunt necesare pentru a porni rețeaua. BitTorrent, unul dintre cele
mai de succes protocoale, izolează o sesiune de partajare în cadrul unui
swarm, descris el însuși de un fișier de metadate de tip \textit{.torrent}.

Napster\footnote{\url{http://www.napster.com/}} a fost primul exemplu
reușit de sistem peer-to-peer. Napster a fost închis definitiv în iunie 2001,
un eveniment care a marcat sfârșitul unei perioade remarcabile, cu 65 de
milioane de utilizatori în doar 20 de luni. Modelul central de control
al aplicației Napster a fost posibil datorită unei abordări centrate pe
utilizator.

Într-o arhitectură obișnuită, comunicația între componentele Napster-ului
este mediată de către server. Clienții se conectează la un \textbf{meta-server}
cunoscut. Meta-serverul alege un server cu încărcare redusă din cadrul
unuia dintre clustere. Fiecare cluster are 5 servere, iar fiecare poate
controla aproximativ 15.000 utilizatori.

Inițial, Gnutella\footnote{\url{http://rfc-gnutella.sourceforge.net/}}
a fost numele unui prototip de client dezvoltat în doar câteva săptămâni
în Martie 2000 de către aceeași echipă care a creat WinAmp. La lansarea
versiunii beta, aproape toată lumea a considerat Gnutella un competitor
pentru Napster, proiectat pentru a rezolva multe dintre neajunsurile sale.
Cu toate acestea, AOL, care cumpărase recent Nullsoft, a decis să oprească
dezvoltarea clientului. Povestea s-ar fi sfârșit acolo dacă n-ar fi fost
Bryan Mayland, care a dedus protocolul Gnutella și i-a publicat specificația
pe Internet. Astfel a început dezvoltarea unui proiect Gnutella open-source.

Astăzi, Gnutella este un termen generic cu diferite sensuri: protocolul,
tehnologia open-source și rețeaua folosită. Există un număr de clienți pentru
protocolul Gnutella și, deși majoritatea se axează pe partajarea și
căutarea de fișiere, pot fi activate multe alte operații.

Una din cele mai noi și inovative aplicații în arhitecturile peer-to-peer este
rețeaua FastTrack. Ea a apărut ca o soluție la problemele observate la Napster
și Gnutella. Arhitectura FastTrack este, prin natura ei, hibridă, ceea ce
presupune o intersecție a două sau mai multe topologii de rețea de bază. În
FastTrack discutăm despre intersecția între topologii centralizate și
descentralizate.

BitTorrent\footnote{\url{http://www.bittorrent.com/}} este un protocol nou,
proiectat pentru a soluționa problema fenomenului ,,free riding'', deși nu
în totalitate~\cite{free-riding}. Ideea din spatele protocolului este simplă:
nodul care joacă rolul de server sparge fișierul în fragmente. Dacă fișierul
este cerut de mai mulți clienți simultan, fiecare client va obține un
alt fragment. Când un client descarcă complet primul fragment, va permite
altor clienți să îl descarce la rândul lor, în timp ce el îl obține pe al
doilea. Cu alte cuvinte, nodul are simultan rol de client și server după ce
a primit complet primul fragment. Procesul continuă până când descărcarea
fișierului se termină.

Protocolul BitTorrent este potrivit pentru descărcarea fișierelor de dimensiuni
mari, proces care durează un timp îndelungat. În consecință, mai mulți
utilizatori care joacă rol de server sunt disponibili pentru o perioadă mai
mare. Creșterea numărului de clienți nu va determina scăderea performanței
rețelei, pentru că încărcarea va fi distribuită între ei.

Să considerăm o situație în care serverul (seeder) partajează un fișier
împărțit în patru fragmente. Fiecare client (leecher) cere unul dintre
fragmente de la server. Se observă cum clientul \#1 obține toate fragmentele,
dar de la noduri diferite. Ordinea în care fragmentele ajung la client nu
trebuie să corespundă secvenței inițiale. Algoritmul BitTorrent va încerca
să maximizeze numărul de fragmente disponibile pentru descărcare la un anumit
moment de timp -- cu alte cuvinte, să echilibreze disponibilitatea fragmentelor
între utilizatorii dintr-o rețea.

Pentru a descărca un fișier într-o rețea BitTorrent, un utilizator execută
următorii pași:

\begin{itemize}
  \item instaliează o aplicație BitTorrent
  \item caută pe web resursa dorită
  \item găsește un fișier pe server
  \item selectează locația unde fișierul va fi descărcat
  \item așteaptă terminarea procesului
  \item închide aplicația (fișierul va continua să fie partajat
                până când utilizatorul închide aplicația).
\end{itemize}

\section{Distribuția conținutului în rețele peer-to-peer}
\label{sec:p2p-systems:streaming}

În mod tradițional, soluțiile peer-to-peer au fost utilizate pentru partajarea
de fișiere între utilizatorii din Internet. În general, transferul de fișiere
are loc în secțiuni mici (denumite și fragmente), care sunt transferate de
la un client la altul. Două fragmente pot fi transferate de la doi clienți
diferiți, în funcție de disponibilitatea lor. Transferul fragmentelor poate fi
(și în general este) nesecvențial; un fragment este descărcat în funcție de
disponibilitatea lui, de acordul utilizatorului care îl deține și de
implementarea protocolului peer-to-peer.

Deoarece conținutul multimedia alcătuiește o mare parte din datele distribuite
în Internet, în ultimii ani s-a observat dezvoltarea soluțiilor de streaming
și integrarea lor cu soluțiile peer-to-peer. Atât streamingul în timp real,
cât și la cerere, au fost introduse în soluțiile peer-to-peer existente sau
implementate ca aplicații noi. Vom introduce principalele metode pentru
furnizarea de soluții de streaming pentru sisteme P2P, cum ar fi sisteme
bazate pe arbori, multi-arbori și rețele.

Într-o rețea P2P, utilizatorul poate nu doar să descarce un fișier video,
ci și să-l încarce. Astfel, un utilizator poate deveni un participant activ
la procesul de streaming. În ultimii ani au apărut aplicații de streaming video
bazate pe soluții P2P.

Soluțiile de streaming bazate pe P2P ~\cite{p2p-streaming-survey} creează o
topologie overlay pentru livrarea conținutului (o topologie de rețea virtuală
suprapusă peste una fizică). De obicei, topologia overlay are o structură
de arbore (\textit{tree-based}) sau rețea (\textit{mesh-based}).

Într-o topoogie cu structură de arbore, datele sunt transportate de la nodul
rădăcină (\textit{root node}) către nodurile fii, și așa mai departe.
Principalul dezavantaj este deconectarea utilizatorilor (peer churn). Atunci
când un părinte se deconectează din rețea, livrarea conținutului video către
fiii acestuia va fi întreruptă temporar.

Într-o topologie cu structură de rețea, utilizatorii pot comunica între ei
fără să adere la o topologie statică. Relațiile sunt stabilite în funcție de
disponibilitatea datelor și lățimea de bandă. Periodic, utilizatorii inițiază
conexiuni, fac schimb de informații cu privire la disponibilitatea datelor și
transferă date de la vecini. Acest tip de structură are avantajul robusteții
în cazul în care utilizatorii se deconectează. Totuși, calitatea imaginii poate
suferi dacă nu este găsită o cale de distribuție a datelor.

\section{Probleme și provocări în sistemele peer-to-peer}
\label{sec:p2p-systems:issues}

Multitudinea soluțiilor peer-to-peer din Internet face dificilă alegerea
soluției potrivite pentru o problemă dată. Fiecare aplicație sau tip de
aplicație este specializată într-un anumit tip de activitate; faptul că
există variante ale fiecărui tip de aplicație poate reprezenta o piedică în
calea unei alegeri optime. Pentru a face o clasificare formală a soluțiilor
este nevoie de măsurători, metrici și evaluări.

Deși streamingul (atât la cerere cât și în timp real) a fost integrat în multe
soluții peer-to-peer, performanța acestora este sub aceea a unui server HTTP
cu încărcare moderată și mult mai mică decât o rețea de livrare de conținut
clasică (CDN). O mai bună utilizare a lățimii de bandă, motivarea
utilizatorilor, crearea de noi modele și infrastructuri hibride P2P/CDN ar putea
crește performanța aplicațiilor peer-to-peer care răspund unor cereri de
streaming video.

O evaluare a diferitelor soluții peer-to-peer poate fi obținută fie
experimental, fie printr-o analiză formală. În timp ce analiza formală se
poate face cu creionul și hârtia, cea experimentală necesită un număr mare de
resurse și alegerea unor puncte de monitorizare potrivite pentru a obține
informațiile necesare evaluării. O abordare des întâlnită este plasarea de
,,sonde'' în swarmuri peer-to-peer existente~\cite{corr-overlay} și captarea
de informații în acele puncte. O alta implică dezvoltarea unei infrastructuri
~\cite{bt-vi} capabile să realizeze o simulare realistă a unei rețele
peer-to-peer; această abordare ar necesita un număr mare de sisteme, apropiat
de dimensiunea medie a unui swarm, pentru a asigura un mediu de test realist.

Chiar și după 10 ani de dezvoltare continuă, sistemele peer-to-peer dau naștere
unor probleme interesante și creează dificultăți semnificative pentru
inginerii și oamenii de știință din domeniul Internetului. Analiza
amănunțită și evaluarea, dublate de soluții inovative și eficiente, reprezintă
cheia îmbunătățirii aplicațiilor peer-to-peer și a gradului de utilizare
a acestora în Internet.
