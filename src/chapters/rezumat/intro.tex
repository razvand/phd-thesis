% vim: set tw=78 tabstop=4 shiftwidth=4 aw ai:

\chapter{Introduction}
\label{chapter:intro}

\textit{``Make it run, then make it right, then make it fast.'' -- Kent Beck}

Among the myriad of inventions, discoveries and contributions to the
development of human civilization, the Internet stands out as one of the
most impressive and extraordinary. With only a few decades of existence, the
Internet has managed to provide the means for uniting the entire planet, for
ensuring access to knowledge, information and digital resources. The Internet
is also a catalyst for all major technological advances in many other fields
of study where researchers and engineers alike have been challenged to provide
better services, better connectivity, better user experience, better
performance and increased diversity.

On the frontier between two millennia, when the Internet had become a
mainstream technology and de facto service, a new technology emerged,
fulfilling the need for data exchange among edge users: Peer-to-Peer Systems.

The exploding evolution of the Internet had provided users everywhere with
higher and higher bandwidth. The similarly explosive evolution of computer
technology ensured faster CPUs, larger hard disks, more memory. As even fringe
users had access to a good quality resources, the need to easily interconnect
and share information became preeminent. Peer-to-Peer Systems thus filled the
vacuum technological space enabled by the ever progressing Internet
technology.

Peer-to-Peer technology is as early as the Internet. In the beginnings of the
Internet, all stations had limited power and limited bandwidth and had to
cooperate to achieve a common goal. A prime example of this is the email
service. Peer-to-Peer have reemerged in late 90s when
Napster\footnote{\url{http://www.napster.com/}}
became a true powerful service in use by multiple users. Through more than 10
years of research, engineering and updates, Peer-to-Peer concepts,
applications and protocols have managed nowadays to become common knowledge.

The BitTorrent protocol is, in this thesis author's opinion, an example of a
protocol that had been made to work, and then has been made to work right,
though, as every protocol, has its own downsides~\cite{bittorrent-trade-offs}.
Simple, yet powerful, BitTorrent shines when used for large content
distribution and has subsequently been adapted for other features such as
streaming~\cite{bittorrent-streaming}.

It is around the BitTorrent protocol that this thesis is centered around. Our
aim, as highlighted in the thesis, is to take the protocol, run measurements,
analyze it and make it better: make it faster, enhance its performance, its
usability, broaden its scope. We do not aim to fill every possible spot of
improvement, but rather signal relevant aspects to be taken into account to
ensure increased performance.

\section{Thesis Objective}
\label{sec:intro:objective}

When involved in Peer-to-Peer systems, updates, enhancements and improvements
have to take into account the rules, diversity and particularities of such
systems. Carefully crafted experiments, sensible measurements, reasonable
conclusions, useful proposals are important components of research activities
in this domain.

The objective of this thesis is providing a series of improvements to
Peer-to-Peer systems at protocol level, either through updating and enhancing
existing protocols or designing and implementing new ones. Formal and
experimental evaluations are employed to provide arguments regarding the
advantages of these updates. Providing improvements is based on careful
trial deployment and analysis of Peer-to-Peer protocols. The use of testing
environments and the employment of measurements and evaluation techniques are
critical to the successful providing of protocol updates and improvements. As
a secondary objective, the thesis highlights the methods and mechanisms for
creating realistic, scalable and automated trial environments and the
approaches available for collecting, measuring and interpreting Peer-to-Peer
protocol parameters.

\section{Thesis Scope}
\label{sec:intro:scope}

In accordance with the stated objective of providing improvements and updates
to Peer-to-Peer systems (as also stated in the title), this thesis primarily
defines its scope in the field of Peer-to-Peer systems, with an overal view
present in Chapter~\ref{chapter:p2p-systems}. In particular, most of
the research and development of this work is centered around the BitTorrent
protocol, one of the most popular protocols in the Internet.

Motivated by the desire to provide new and useful updates to existing
applications and creating new ones, the approach employed is one that
carefully deploys and monitors Peer-to-Peer applications, collects protocol
parameters, puts them under scrutiny and analysis and provides the input
required for improvements, extensions and updates.

While multiple approaches may be considered for client, protocol, network and
swarm analysis, the approach chosen relies on low level information,
considered to be the basis of protocol parameters. We are not (directly)
concerned with the overall view of the Peer-to-Peer system, but rather with an
in depth view of the client behavior. Hook points in Peer-to-Peer client
implementations, client logging or network traffic analysis provide the means
to gather the valuable low level Peer-to-Peer protocol information.

Protocol messages contain valuable information such as client download rate,
client upload rate, number of connected peers, protocol events. These
parameters are analysed and conclusions are drawn allowing for protocol
improvements to be proposed, analyzed and then put to use. An important
approach is using collected parameters for comparison between various
implementations, protocols or behavior in diverse situations.

Protocol improvements may result in updates to existing protocols and
solutions as is the case with the tracker overlay presented in
Chapter~\ref{chapter:unified-tracker} or the design of a novel protocol as is
the case with the multiparty protocol implementation in the Linux kernel
detailed in Chapter~\ref{chapter:multiparty}.

\section{Contents}
\label{sec:intro:contents}

Thesis contents follow both a chronological order, when considering the work
employed, and a logical order as latter chapters rely on information presented
by former chapters. Forward and backward references are provided where
necessary, while always keeping an eye out for providing the initial
information in a previous chapter.

Chapter~\ref{chapter:p2p-systems} -- \textbf{Peer-to-Peer Systems} presents a
state of the art of Peer-to-Peer systems and implementations. It provides
insight on the birth of P2P systems, their evolution and current challenges.
Particular focus is attributed to the BitTorrent Protocol and content
distribution as they collectively form the context of this work. We insist on
streaming features for P2P systems as it is a relevant aspect presented in
this thesis. Issues and challenges are also identified and described.

The backbone of most of the experiments and trials involved is the
virtualized network infrastructure and features presented in
Chapter~\ref{chapter:virt-infra}. We argument the benefits of using
virtualization for creating P2P environments and highlight the characteristics
of the employed infrastructure. Thorough information is provided about
automating the deployment of the infrastructure, Peer-to-Peer clients and
gathering results.

We have designed, developed and deployed a novel virtualization infrastructure
providing such features as: scalability, extensibility, automation. Based on
OpenVZ and using a plethora of tools, the infrastructure has been used to
create experimental setups for scenarios including performance measurements,
implementation comparison, streaming versus classical distribution. We propose
a set of metrics that provide performance information about various
virtualization solutions, including OpenVZ, Xen and LXC.

Chapter~\ref{chapter:proto-measure} is concerned with methods and approaches
of gathering information from Peer-to-Peer clients, parsing that information
in protocol parameters and subjecting it to analysis. Two approaches used for
collecting logging information are the implementation of a generic library
whose API is ``hooked'' in existing applications and the usage of client
logging data. A custom processing framework is provided to interpret data
provided by clients. Information may be analyzed either subsequent to the log
collection activity (post-processing) or in the same time (live/real-time).
The latter approach may be used to provide monitoring.

The generic library has been integrated in the libtorrent-rakshasa and
Transmission BitTorrent clients, while the custom processing framework has
been integrated with a variety of clients, due to its less intrusive nature.
Storage, parsing and analysis components allow analysis and measurements of
BitTorrent swarms. We propose a formal analysis framework based on client
speed parameters; it is focused on providing peer and swarm performance
information.

A first improvement to BitTorrent swarms is the design, implementation,
testing and evaluation of a novel protocol that allows swarm unification, as
described in Chapter~\ref{chapter:unified-tracker}. The protocol, dubbed TSUP
(\textit{Tracker Swarm Unification Protocol}) allows swarms that part the same
.torrent file, yet are different, to be unified, such that peers would be able
to communicate one with the other even if initially in different swarms.
Intermediation is enabled by trackers and the new protocol forms a tracker
overlay network -- each tracker is initially part of the single swarm.

TSUP has been designed from scratch and has been implemented into the XBT
tracker. The implementation has been evaluated both formally and
experimentally on top of the virtualized infrastructure. The experimental
setup made use of diverse swarms with respect to number of trackers, seeders
and leechers.

Chapter~\ref{chapter:multiparty} presents the design and initial
implementation of a multiparty protocol in the Linux kernel. Heavily based on
swift this implementation aims to provided a real Transport layer of a
multiparty protocol directly within the Linux networking stack. An
intermediate approach, using raw sockets, is employed to make use of the
favorable user space environment while still providing the same interface.

The streaming part of this work and description of trials and results in the
local LivingLab are presented in Chapter~\ref{chapter:multimedia-dist}. The
addition of streaming facilities in the popular libtorrent-rasterbar
implementation\footnote{\url{http://www.rasterbar.com/products/libtorrent/}}
is an important component, highlighting the steps required to provide
streaming. The major part of the chapter is dedicated to the deployment,
analysis and evaluation of the NextShare
technology\footnote{\url{http://www.p2p-next.org}}, an advanced Peer-to-Peer
implementation based on BitTorrent.

Results and analysis of Peer-to-Peer streaming deployment in the LivingLab
form an important contribution of this work. Both feedback from users and
collected information from clients are used to provide an insight on the
performance of video-on-demand streaming in a Peer-to-Peer network.

The last chapter (\textbf{Conclusion}) highlights the major contributions and
results of this work and also points out what should be undertaken in the near
future. It rounds up the scientific relevance of this work and sets the ground
for further actions on the explored directions.
