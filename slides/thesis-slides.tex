% vim: set tw=78 tabstop=8 shiftwidth=2 softtabstop=2 aw ai:
\documentclass{beamer}

\usepackage[utf8x]{inputenc}            % diacritice
\usepackage[english]{babel}
\usepackage{color}                      % highlight
\usepackage{alltt}                      % highlight
%\usepackage{code/highlight}            % highlight
\usepackage{hyperref}                   % folosiți \url{http://...}
                                        % sau \href{http://...}{Nume Link}
\usepackage{verbatim}
\usepackage{subfigure}
\usepackage{booktabs}

% Show contents at every section beginning. Ripped off from manual.
\AtBeginSection[] % Do nothing for \section*
{
  \begin{frame}<beamer>
    \frametitle{Outline}
  \tableofcontents[currentsection]
    \end{frame}
}

\mode<presentation>
{ \usetheme{Berlin} }

% Încărcăm simbolurilor Unicode românești în titlu și primele pagini
\PrerenderUnicode{aâîțșĂÎÂȚȘ}

\title[Protocol Measurements and Improvements in Peer-to-Peer
Systems]{Protocol Measurements and Improvements in Peer-to-Peer Systems}
\subtitle{PhD Thesis}
\institute[CSE, ACS, UPB]{University POLITEHNICA of Bucharest}
\author[Răzvan Deaconescu]{Author: drd. Răzvan Deaconescu\\
Supervisor: prof. dr. ing. Nicolae Țăpuș}
\date{September 28, 2011}

\begin{document}

% Slide-urile cu mai multe părți sunt marcate cu textul (cont.)
\setbeamertemplate{frametitle continuation}[from second]

% Arătăm numărul frame-ului
% \setbeamertemplate{footline}[frame number]

\frame{\titlepage}

\section{Context}

\begin{frame}{TODO}
  \begin{itemize}
    \item TODO
  \end{itemize}
\end{frame}

\section{Objective and Description}

\begin{frame}{TODO}
  \begin{itemize}
    \item TODO
  \end{itemize}
\end{frame}

\section{Virtualized Network Infrastructure}

\begin{frame}{TODO}
  \begin{itemize}
    \item TODO
  \end{itemize}
\end{frame}

\section{Questions}

\end{document}
